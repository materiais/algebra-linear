\documentclass[12pt,a4paper]{article}
\usepackage{cmap} % Makes the PDF copiable. See http://tex.stackexchange.com/a/64198/25761
\usepackage[T1]{fontenc}
\usepackage[brazil]{babel}
\usepackage[utf8]{inputenc}
\usepackage{amsmath}
\usepackage{amsfonts}
\usepackage{amssymb}
\usepackage{amsthm}
\usepackage{textcomp} % \degree
\usepackage{gensymb} % \degree
\usepackage[usenames,svgnames,dvipsnames]{xcolor}
\usepackage{hyperref}
\usepackage{multicol}
\usepackage{graphicx}
\usepackage[margin=2cm]{geometry}
\usepackage{systeme}

\hypersetup{
    colorlinks = true,
    allcolors = {blue}
}

% TODO: Consider using exsheets
% http://linorg.usp.br/CTAN/macros/latex/contrib/exsheets/exsheets_en.pdf
%
% http://ctan.org/tex-archive/macros/latex/contrib/exercise/
% Options: answerdelayed,lastexercise,noanswer
\usepackage[answerdelayed,lastexercise]{exercise}

\addto\captionsbrazil{%
\def\listexercisename{Lista de exerc\'icios}%
\def\ExerciseName{Exerc\'icio}%
\def\AnswerName{Solu\c{c}\~ao do exerc\'icio}%
\def\ExerciseListName{Ex.}%
\def\AnswerListName{Solu\c{c}\~ao}%
\def\ExePartName{Parte}%
\def\ArticleOf{de\ }%
}

\renewcommand{\ExerciseHeaderTitle}{(\ExerciseTitle)\ }
\renewcommand{\ExerciseListHeader}{%\ExerciseHeaderDifficulty%
\textbf{%\ExerciseListName\
\ExerciseHeaderNB.\ %
%\ --- \
\ExerciseHeaderTitle}%
%\ExerciseHeaderOrigin
\ignorespaces}
\renewcommand{\AnswerListHeader}{\textbf{\ExerciseHeaderNB.\ (\AnswerListName)\ }}

\newcommand*\R{\mathbb{R}}

% Loop Space / CC BY-SA-3.0 / https://tex.stackexchange.com/a/2238/25761
\newenvironment{amatrix}[1]{%
  \left[\begin{array}{@{}*{#1}{c}|c@{}}
}{%
  \end{array}\right]
}

% Loop Space / CC BY-SA-3.0 / https://tex.stackexchange.com/a/3164/25761
%--------grstep
% For denoting a Gauss' reduction step.
% Use as: \grstep{\rho_1+\rho_3} or \grstep[2\rho_5 \\ 3\rho_6]{\rho_1+\rho_3}
\newcommand{\grstep}[2][\relax]{%
   \ensuremath{\mathrel{
       {\mathop{\longrightarrow}\limits^{#2\mathstrut}_{
                                     \begin{subarray}{l} #1 \end{subarray}}}}}}

\renewcommand{\theenumi}{\alph{enumi}}
\renewcommand\labelenumi{(\theenumi) }

\newcommand*\tipo{Prova III}
\newcommand*\turma{CIV122-02U}
\newcommand*\disciplina{ALI0001}
\newcommand*\eu{Helder G. G. de Lima}
\newcommand*\data{05/11/2018}

\author{\eu}
\title{\tipo - \disciplina}
\date{\data}

\begin{document}
\thispagestyle{empty}
\newgeometry{margin=2cm,bottom=0.5cm}
\begin{center}
\includegraphics[width=9.0cm]{marca} \\
\textbf{\tipo\ (\disciplina / \turma)} \\
Prof. \eu\footnote{
Este é um material de acesso livre distribuído sob os termos da licença \href{https://creativecommons.org/licenses/by-sa/4.0/deed.pt_BR}{Creative Commons BY-SA 4.0}}
\end{center}

\noindent Nome do(a) aluno(a): \underline{\hspace{9,7cm}} Data: \underline{\data}

%\section*{Instruções}
\begin{center}\fbox{
\begin{minipage}{14cm}

\begin{footnotesize}
\begin{itemize}
\renewcommand{\theenumi}{\Roman{enumi}}
\item Identifique-se em todas as folhas.
\item Mantenha o celular e os demais equipamentos eletrônicos desligados durante a prova.
\item Resolva (integralmente) apenas os itens de que precisar para somar 10,0 pontos.
\end{itemize}
\end{footnotesize}

\end{minipage}
}
\end{center}

\section*{Questões}
\begin{ExerciseList}
\Exercise[title={2,0}] A função $f: \R^2 \to \R$ definida por $f(x,y) = \sqrt[3]{x^3+y^3}$ é linear? Em caso afirmativo, justifique com todos os detalhes necessários. Caso contrário, dê um exemplo numérico mostrando claramente que uma das propriedades das transformações lineares não é satisfeita.
\Answer Para que $f$ fosse uma transformação linear, deveria ser verdade que $f(u+v) = f(u)+f(v)$, para quaisquer $u$ e $v$ em $\R^2$. No entanto, se $u = (1,0)$ e $v=(0,1)$, então $u+v = (1,1)$ e
\[
f(u)+f(v)
= \sqrt[3]{1^3+0^3} + \sqrt[3]{0^3+1^3}
= 1 + 1
\neq \sqrt[3]{2}
= \sqrt[3]{1^3+1^3}
= f(u+v).
\]
Portanto, $f$ não é uma transformação linear.


\Exercise[title={2,0}] Seja
$\alpha = \{(2, 0, 0), (0, 4, 4), (0,0,8)\}$ uma base de $\R^3$ e $\beta = \{(1, 2),(-1, 0)\}$ uma base de $\R^2$. Se $T:\R^3 \to \R^2$ é uma transformação linear e $[T]_\beta^\alpha =
\begin{bmatrix}
 0 &  6 & 4\\
-6 & 10 & 4
\end{bmatrix}
$, encontre $T(x,y,z)$.
\Answer Como as colunas de $[T]_\beta^\alpha$ são formadas pelas coordenadas de $T(v_i)$, para cada $v_i$ da base $\alpha$, tem-se:
\begin{align*}
T(2,0,0) & = 0 \cdot (1,2) - 6 \cdot (-1,0) = (6,0),\\
T(0,4,4) & = 6 \cdot (1,2) +10 \cdot (-1,0) = (-4, 12),\\
T(0,0,8) & = 4 \cdot (1,2) + 4 \cdot (-1,0) = (0, 8).
\end{align*}
Além disso, como $\alpha$ é uma base, dado qualquer $v = (x,y,z) \in \R^3$, existem $c_1, c_2, c_3 \in \R$ tais que
\[
(x,y,z) = c_1 (2,0,0) + c_2(0,4,4) + c_3(0,0,8) = (2c_1, 4c_2,4c_2+8c_3).
\]
De fato, $c_1 = \frac{x}{2}$, $c_2 = \frac{y}{4}$ e $c_3 = \frac{z-y}{8}$. Assim, pode-se obter $T$ do seguinte modo:
\begin{align*}
T(x,y,z)
& = T( c_1 (2,0,0) + c_2(0,4,4) + c_3(0,0,8) )\\
& = c_1 T(2,0,0) + c_2T(0,4,4) + c_3T(0,0,8)\\
& = \frac{x}{2} (6,0)
  + \frac{y}{4}(-4, 12)
  + \frac{z-y}{8}(0, 8)\\
& = (3x-y,2y+z).
\end{align*}

\textbf{Solução alternativa} Como as coordenadas de $(x,y,z)$ em relação à base $\alpha$ foram calculadas acima, tem-se:
\[
[T(x,y,z)]_\beta
= [T]_\beta^\alpha \cdot [(x,y,z)]_\alpha
= \begin{bmatrix}
 0 &  6 & 4\\
-6 & 10 & 4
\end{bmatrix}
\cdot
\begin{bmatrix}
\frac{x}{2} \\ \frac{y}{4} \\ \frac{z-y}{8}
\end{bmatrix}
=
\begin{bmatrix}
y+\frac{z}{2}\\
-3x+\frac{z}{2}+2y
\end{bmatrix}.
\]
Assim, combinando linearmente os vetores de $\beta$ com estas coordenadas, conclui-se que:
\[
T(x,y,z)
= \left(y+\frac{z}{2}\right)\cdot (1,2) + \left(-3x+\frac{z}{2}+2y\right)\cdot (-1,0)
= (3x-y,2y+z).
\]

\Exercise[title={2,0}] Seja $P_2 = \{ q \mid q(x) = ax^2+bx+c\}$ o espaço vetorial formado pelos polinômios de grau no máximo dois e considere o operador linear $T:P_2 \to P_2$ definido por $T(q) = q - q^\prime$, para cada $q \in P_2$. Verifique se $T$ é um isomorfismo, justificando todas as suas afirmações.
\Answer Como o domínio e o contradomínio de $T$ são iguais, e portanto de mesma dimensão, basta que $T$ seja injetora para que também seja bijetora, isto é, um isomorfismo.

Para checar a injetividade de $T$, basta conferir se o seu núcleo é o espaço nulo:

Por definição, para todo polinômio $q \in P_2$, com $q(x) = ax^2+bx+c$, tem-se
\[
T(q)
= T(ax^2+bx+c)
= (ax^2+bx+c)-(2ax+b)
= ax^2+(b-2a)x+(c-b)
\]
Além disso, para checar a injetividade de $T$, basta conferir se o seu núcleo é o espaço nulo:
\[
T(q) = \vec{0}
\Leftrightarrow ax^2+(b-2a)x+(c-b) = 0x^2+0x+0, \forall x \in \R\\
\Leftrightarrow
\begin{cases}
a = 0\\
b-2a = 0\\
c-b =0
\end{cases}
\]
Como o sistema anterior é possível determinado, $a=b=c=0$, e conclui-se que $N(T) = \{ \vec{0} \}$, ou seja, que $T$ é injetora. Portanto, $T$ é um isomorfismo.


\Exercise[title={2,0}] Sejam $V$ e $W$ espaços vetoriais, e $T: V \to W$ uma transformação linear bijetora e $T^{-1}: W \to V$ a inversa de $T$, isto é, a função definida por
\[
T^{-1}(w) = v \Leftrightarrow T(v) = w.
\]
Demonstre que $T^{-1}$ também é uma transformação linear.
\Answer Para concluir que $T^{-1}$ é uma transformação linear, basta observar o seguinte:
\begin{enumerate}
\item Sejam $w_1$ e $w_2$ vetores arbitrários de $W$. Então
$w_1 = T(v_1)$ e $w_2 = T(v_2)$, para certos vetores $v_1$ e $v_2$ de $V$. Neste caso,
\[
T^{-1}(w_1 + w_2)
\stackrel{*}{=} T^{-1}(T(v_1) + T(v_2))
= T^{-1}(T(v_1+v_2)))
= v_1 + v_2
= T^{-1}(w_1)+T^{-1}(w_1).
\]

\item Se $w$ é um vetor arbitrário de $W$ e $\alpha \in \R$, então $w = T(v)$, para algum $v\in V$. Neste caso,
\[
T^{-1}(\alpha \cdot w)
= T^{-1}(\alpha \cdot T(v))
\stackrel{*}{=} T^{-1}(T(\alpha \cdot v))
= \alpha \cdot v
= \alpha \cdot T^{-1}(w).
\]
\end{enumerate}
Nos itens anteriores, as igualdades marcadas com um asterisco (*) decorrem fato de $T$ ser linear.

\Exercise
Sejam $\alpha = \left\{ (-4,6), (-3,17)\right\}$ e $\beta = \{ (-5,15), (0,-10) \}$ bases de $\R^2$. Demonstre, baseando-se em cálculos ou em argumentos teóricos (o que for mais apropriado), que:
\begin{enumerate}
\item \textbf{(2,0)} Se $[v]_{\alpha} =
\begin{bmatrix}
7\\9
\end{bmatrix}$, então $[v]_{\beta} =
\begin{bmatrix}
11\\-3
\end{bmatrix}$.
\item \textbf{(2,0)} As matrizes $[I]_{\beta}^{\alpha}$ e $[I]_{\alpha}^{\beta}$, neste caso, são iguais.
\end{enumerate}
\Answer
\begin{enumerate}
\item Se $[v]_{\alpha} =
\begin{bmatrix}
7\\9
\end{bmatrix}$, então, pela definição de coordenadas, tem-se
\[
v = 7\cdot (-4,6)+9\cdot (-3,17) = (-55, 195).
\]
Por outro lado,
\[
(-55, 195) = 11\cdot (-5,15) + (-3)\cdot (0,-10),
\]
então conclui-se que $[v]_{\beta} =
\begin{bmatrix}
11\\-3
\end{bmatrix}$.
\item
Observando que
\begin{align*}
(-4, 6)& =\frac{4}{5}(-5,15)+\frac{3}{5}(0,-10)\\
(-3,17)& =\frac{3}{5}(-5,15)-\frac{4}{5}(0,-10)
\end{align*}
conclui-se que $[I]_{\beta}^{\alpha}
=\begin{bmatrix}
\frac{4}{5} & \frac{3}{5} \\
\frac{3}{5} & -\frac{4}{5}
\end{bmatrix}$. Consequentemente,
\[
[I]_{\alpha}^{\beta}
= \left([I]_{\beta}^{\alpha}\right)^{-1}
= \left(
\begin{bmatrix}
\frac{4}{5} & \frac{3}{5} \\
\frac{3}{5} & -\frac{4}{5}
\end{bmatrix}
\right)^{-1}
=
\frac{1}{\frac{4}{5} \cdot \left(-\frac{4}{5}\right) - \frac{3}{5} \cdot \frac{3}{5}}
\begin{bmatrix}
-\frac{4}{5} & -\frac{3}{5} \\
-\frac{3}{5} & \frac{4}{5}
\end{bmatrix}
=
\frac{1}{-1}
\begin{bmatrix}
-\frac{4}{5} & -\frac{3}{5} \\
-\frac{3}{5} & \frac{4}{5}
\end{bmatrix}
=[I]_{\beta}^{\alpha}.
\]

\textbf{Solução alternativa:}
Considerando que
\[
[I]_{\alpha}^{\beta} = ([I]_{\beta}^{\alpha})^{-1},
\Leftrightarrow
[I]_{\alpha}^{\beta} \cdot [I]_{\beta}^{\alpha} = I_{2\times 2},
\]
a igualdade $[I]_{\beta}^{\alpha} = [I]_{\alpha}^{\beta}$ só será verdadeira se $[I]_{\beta}^{\alpha} \cdot [I]_{\beta}^{\alpha} = {([I]_{\beta}^{\alpha})}^2 = I_{2\times 2}$, o que é verdade:
\[
{([I]_{\beta}^{\alpha})}^2
=
\begin{bmatrix}
\frac{4}{5} & \frac{3}{5} \\
\frac{3}{5} & -\frac{4}{5}
\end{bmatrix}
\cdot
\begin{bmatrix}
\frac{4}{5} & \frac{3}{5} \\
\frac{3}{5} & -\frac{4}{5}
\end{bmatrix}
=
\begin{bmatrix}
1&0\\0&1
\end{bmatrix}.
\]
\end{enumerate}
\end{ExerciseList}

\vfill
\begin{center}
BOA PROVA!
\end{center}

\newpage
\restoregeometry
\section*{Respostas}
\shipoutAnswer
\end{document}
