\documentclass[12pt,a4paper]{article}
\usepackage{cmap} % Makes the PDF copiable. See http://tex.stackexchange.com/a/64198/25761
\usepackage[T1]{fontenc}
\usepackage[brazil]{babel}
\usepackage[utf8]{inputenc}
\usepackage{amsmath}
\usepackage{amsfonts}
\usepackage{amssymb}
\usepackage{amsthm}
\usepackage{textcomp} % \degree
\usepackage{gensymb} % \degree
\usepackage[usenames,svgnames,dvipsnames]{xcolor}
\usepackage{hyperref}
\usepackage{multicol}
\usepackage{graphicx}
\usepackage[margin=2cm]{geometry}
\usepackage{systeme}

\hypersetup{
    colorlinks = true,
    allcolors = {blue}
}

% TODO: Consider using exsheets
% http://linorg.usp.br/CTAN/macros/latex/contrib/exsheets/exsheets_en.pdf
%
% http://ctan.org/tex-archive/macros/latex/contrib/exercise/
% Options: answerdelayed,lastexercise,noanswer
\usepackage[answerdelayed,lastexercise]{exercise}

\addto\captionsbrazil{%
\def\listexercisename{Lista de exerc\'icios}%
\def\ExerciseName{Exerc\'icio}%
\def\AnswerName{Solu\c{c}\~ao do exerc\'icio}%
\def\ExerciseListName{Ex.}%
\def\AnswerListName{Solu\c{c}\~ao}%
\def\ExePartName{Parte}%
\def\ArticleOf{de\ }%
}

\renewcommand{\ExerciseHeaderTitle}{(\ExerciseTitle)\ }
\renewcommand{\ExerciseListHeader}{%\ExerciseHeaderDifficulty%
\textbf{%\ExerciseListName\
\ExerciseHeaderNB.\ %
%\ --- \
\ExerciseHeaderTitle}%
%\ExerciseHeaderOrigin
\ignorespaces}
\renewcommand{\AnswerListHeader}{\textbf{\ExerciseHeaderNB.\ (\AnswerListName)\ }}

\newcommand*\ger[1]{\operatorname{ger}\left\{#1\right\}}
\newcommand*\R{\mathbb{R}}

% Loop Space / CC BY-SA-3.0 / https://tex.stackexchange.com/a/2238/25761
\newenvironment{amatrix}[1]{%
  \left[\begin{array}{@{}*{#1}{c}|c@{}}
}{%
  \end{array}\right]
}

% Loop Space / CC BY-SA-3.0 / https://tex.stackexchange.com/a/3164/25761
%--------grstep
% For denoting a Gauss' reduction step.
% Use as: \grstep{\rho_1+\rho_3} or \grstep[2\rho_5 \\ 3\rho_6]{\rho_1+\rho_3}
\newcommand{\grstep}[2][\relax]{%
   \ensuremath{\mathrel{
       {\mathop{\longrightarrow}\limits^{#2\mathstrut}_{
                                     \begin{subarray}{l} #1 \end{subarray}}}}}}

\renewcommand{\theenumi}{\alph{enumi}}
\renewcommand\labelenumi{(\theenumi) }

\newcommand*\tipo{Prova IV}
\newcommand*\turma{PRO112-02U}
\newcommand*\disciplina{ALI0001}
\newcommand*\eu{Helder G. G. de Lima}
\newcommand*\data{05/12/2018}

\author{\eu}
\title{\tipo - \disciplina}
\date{\data}

\begin{document}
\thispagestyle{empty}
\newgeometry{margin=2cm,bottom=0.5cm}
\begin{center}
\includegraphics[width=9.0cm]{marca} \\
\textbf{\tipo\ (\disciplina / \turma)} \\
Prof. \eu\footnote{
Este é um material de acesso livre distribuído sob os termos da licença \href{https://creativecommons.org/licenses/by-sa/4.0/deed.pt_BR}{Creative Commons BY-SA 4.0}}
\end{center}

\noindent Nome do(a) aluno(a): \underline{\hspace{9,7cm}} Data: \underline{\data}

%\section*{Instruções}
\begin{center}\fbox{
\begin{minipage}{14cm}
\begin{footnotesize}
\begin{itemize}
\renewcommand{\theenumi}{\Roman{enumi}}
\item Identifique-se em todas as folhas.
\item Mantenha o celular e os demais equipamentos eletrônicos desligados durante a prova.
\item Justifique cada resposta com cálculos ou argumentos baseados na teoria estudada.
\item Resolva (integralmente) apenas os itens de que precisar para somar 10,0 pontos.
\end{itemize}
\end{footnotesize}
\end{minipage}
}
\end{center}

\section*{Questões}
\begin{ExerciseList}
\Exercise[title={2,0}]
Demonstre a(s) afirmação(ões) verdadeira(s) e dê um contra-exemplo para a(s) falsa(s):
\begin{enumerate}
\item Se $\alpha \neq 7$ então a matriz $A
=
\begin{bmatrix}
1 & 2 \\ 3 & 4
\end{bmatrix}
$ não é semelhante à matriz $B=
\begin{bmatrix}
-2 & 4 \\ -3 & \alpha
\end{bmatrix}$.
\item Se uma matriz $C \in \R^{2 \times 2}$ tem um único autovalor então $C$ não é diagonalizável.
\end{enumerate}
\Answer
\color{red}
\ldots


\Exercise[title={2,0}] Mostre que se $w$ é um autovetor do operador linear $T: V \to V$, então $c \cdot w$ também é um autovetor de $T$, contanto que $c \neq 0$.
\Answer
\color{red}
\ldots


\Exercise[title={2,0}] Seja $\beta = \{ v_1, v_2, v_3 \}$ uma base de $\R^3$, em que $v_1 = (1,2,3)$, $v_2 = (3,2,1)$ e $v_3 = (0,0,1)$. Se $T: \R^3 \to \R^3$ é o operador linear tal que $T(v_1) = v_2$, $T(v_2) = v_1$ e $T(v_3) = v_3$, quais são os autovetores e autovalores de $T$? (cuidado com os índices acima!)
\Answer
\color{red}
\ldots


\Exercise[title={2,0}] Seja $T: P_2 \to P_2$, dado por $T(q) = q^\prime$ (ou seja, $T$ é o operador derivação). Verifique se $T$ é diagonalizável, e justifique sua resposta.
\Answer
\color{red}
\ldots


\Exercise[title={2,0}]
Mostre que existe alguma base $\beta$ de $\R^3$ formada por autovetores do operador linear
\[
T(x,y,z) = (x - 6 y, -y, -4 x + 12 y - z).
\]
\Answer
\color{red}
\ldots


\Exercise[title={2,0}]
Seja $A = 
\begin{bmatrix}
2 & 6 \\ 0 & -1
\end{bmatrix}$. Utilize a diagonalização de $A$ para calcular $A^9$.
\Answer \ldots
\color{red}
\ldots
\end{ExerciseList}

\vfill
\begin{center}
BOA PROVA E BOAS FÉRIAS!
\end{center}

%\newpage
%\restoregeometry
%\section*{Respostas}
%\shipoutAnswer
\end{document}
