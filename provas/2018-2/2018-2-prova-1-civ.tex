\documentclass[12pt,a4paper]{article}
\usepackage{cmap} % Makes the PDF copiable. See http://tex.stackexchange.com/a/64198/25761
\usepackage[T1]{fontenc}
\usepackage[brazil]{babel}
\usepackage[utf8]{inputenc}
\usepackage{amsmath}
\usepackage{amsfonts}
\usepackage{amssymb}
\usepackage{amsthm}
\usepackage{textcomp} % \degree
\usepackage{gensymb} % \degree
\usepackage[usenames,svgnames,dvipsnames]{xcolor}
\usepackage{hyperref}
\usepackage{multicol}
\usepackage{graphicx}
\usepackage[margin=2cm]{geometry}
\usepackage{systeme}

\hypersetup{
    colorlinks = true,
    allcolors = {blue}
}

% TODO: Consider using exsheets
% http://linorg.usp.br/CTAN/macros/latex/contrib/exsheets/exsheets_en.pdf
%
% http://ctan.org/tex-archive/macros/latex/contrib/exercise/
% Options: answerdelayed,lastexercise,noanswer
\usepackage[answerdelayed,lastexercise]{exercise}

\addto\captionsbrazil{%
\def\listexercisename{Lista de exerc\'icios}%
\def\ExerciseName{Exerc\'icio}%
\def\AnswerName{Solu\c{c}\~ao do exerc\'icio}%
\def\ExerciseListName{Ex.}%
\def\AnswerListName{Solu\c{c}\~ao}%
\def\ExePartName{Parte}%
\def\ArticleOf{de\ }%
}

\renewcommand{\ExerciseHeaderTitle}{(\ExerciseTitle)\ }
\renewcommand{\ExerciseListHeader}{%\ExerciseHeaderDifficulty%
\textbf{%\ExerciseListName\
\ExerciseHeaderNB.\ %
%\ --- \
\ExerciseHeaderTitle}%
%\ExerciseHeaderOrigin
\ignorespaces}
\renewcommand{\AnswerListHeader}{\textbf{\ExerciseHeaderNB.\ (\AnswerListName)\ }}

\newcommand*\R{\mathbb{R}}

% Loop Space / CC BY-SA-3.0 / https://tex.stackexchange.com/a/2238/25761
\newenvironment{amatrix}[1]{%
  \left[\begin{array}{@{}*{#1}{c}|c@{}}
}{%
  \end{array}\right]
}

% Loop Space / CC BY-SA-3.0 / https://tex.stackexchange.com/a/3164/25761
%--------grstep
% For denoting a Gauss' reduction step.
% Use as: \grstep{\rho_1+\rho_3} or \grstep[2\rho_5 \\ 3\rho_6]{\rho_1+\rho_3}
\newcommand{\grstep}[2][\relax]{%
   \ensuremath{\mathrel{
       {\mathop{\longrightarrow}\limits^{#2\mathstrut}_{
                                     \begin{subarray}{l} #1 \end{subarray}}}}}}

\renewcommand{\theenumi}{\alph{enumi}}
\renewcommand\labelenumi{(\theenumi) }

\newcommand*\tipo{Prova I}
\newcommand*\turma{CIV122-02U}
\newcommand*\disciplina{ALI0001}
\newcommand*\eu{Helder G. G. de Lima}
\newcommand*\data{03/09/2018}

\author{\eu}
\title{\tipo - \disciplina}
\date{\data}

\begin{document}
\thispagestyle{empty}
\newgeometry{margin=2cm,bottom=0.5cm}
\begin{center}
\includegraphics[width=9.0cm]{marca} \\
\textbf{\tipo\ (\disciplina / \turma)} \\
Prof. \eu \footnote{
Este é um material de acesso livre distribuído sob os termos da licença \href{https://creativecommons.org/licenses/by-sa/4.0/deed.pt_BR}{Creative Commons BY-SA 4.0}}
\end{center}

\noindent Nome do(a) aluno(a): \underline{\hspace{9,7cm}} Data: \underline{\data}

%\section*{Instruções}
\begin{center}\fbox{
\begin{minipage}{14cm}

\begin{footnotesize}
\begin{itemize}
\renewcommand{\theenumi}{\Roman{enumi}}
\item Identifique-se em todas as folhas.
\item Mantenha o celular e os demais equipamentos eletrônicos desligados durante a prova.
\item Resolva (integralmente) apenas os itens de que precisar para somar 10,0 pontos.
\end{itemize}
\end{footnotesize}

\end{minipage}
}
\end{center}

\section*{Questões}
\begin{ExerciseList}
\Exercise[title={2,0}] Sejam
$A = \begin{bmatrix}
0\\-3\\5
\end{bmatrix}$,
$B = \begin{bmatrix}
-1 & 2 & 0
\end{bmatrix}$,
$C = \begin{bmatrix}
-2 & 0 \\
 4 & 0 \\
 0 & 0
\end{bmatrix}$
e
$D = \begin{bmatrix}
0 & 3 & -5 \\
\pi & \sqrt{2} & 2018
\end{bmatrix}$. Determine:
\begin{enumerate}
\item O resultado, se possível, da operação matricial $2AB+CD$
\item Todas as propriedades, quando possível, das matrizes: $AB$, $CD$ e $2AB+CD$
\end{enumerate}
\Answer Ao realizar as operações indicadas, resulta que:
\[
AB =
\begin{bmatrix}
 0 & 0 & 0 \\
 3 & -6 & 0 \\
-5 & 10 & 0
\end{bmatrix},\quad
CD =
\begin{bmatrix}
0 & -6 &  10 \\
0 & 12 & -20 \\
0 &  0 &   0
\end{bmatrix},\quad \text{ e }
2AB+CD
=\begin{bmatrix}
  0 & -6 &  10\\
  6 &  0 & -20\\
-10 & 20 &   0
\end{bmatrix}.
\]
\begin{itemize}
\item A matriz $AB$ é quadrada, triangular inferior e singular.
\item A matriz $CD$ é quadrada, triangular superior e singular.
\item A matriz $2AB+CD$ é quadrada, antissimétrica e singular.
\end{itemize}

\Exercise[title={2,0}] Obtenha todas as soluções (se existirem) do seguinte sistema linear:
\[
\systeme[xyzw]{
 x + y + z + w = 1,
 x - y + z - w  = 1,
2x + y + z +2w = 1
}
\]
\Answer As soluções podem ser obtidas aplicando-se a eliminação gaussiana à matriz ampliada do sistema:
\begin{align*}
& \begin{bmatrix}
1 &  1 & 1 &  1 & 1\\
1 & -1 & 1 & -1 & 1\\
2 &  1 & 1 &  2 & 1
\end{bmatrix}
\rightarrow
\begin{bmatrix}
1 &  1 & 1 &  1 & 1\\
0 & -2 & 0 & -2 & 0\\
0 & -1 & -1 & 0 & -1
\end{bmatrix}
\rightarrow
\begin{bmatrix}
1 &  1 & 1 &  1 & 1\\
0 & 1 & 0 & 1 & 0\\
0 & -1 & -1 & 0 & -1
\end{bmatrix} \\
& \rightarrow
\begin{bmatrix}
1 &  1 & 1 &  1 & 1\\
0 & 1 & 0 & 1 & 0\\
0 & 0 & -1 & 1 & -1
\end{bmatrix}
\rightarrow
\begin{bmatrix}
1 &  1 & 1 &  1 & 1\\
0 & 1 & 0 & 1 & 0\\
0 & 0 & 1 & -1 & 1
\end{bmatrix}
\rightarrow
\begin{bmatrix}
1 &  1 & 0 &  2 & 0\\
0 & 1 & 0 & 1 & 0\\
0 & 0 & 1 & -1 & 1
\end{bmatrix}
\rightarrow
\begin{bmatrix}
1 & 0 & 0 & 1 & 0\\
0 & 1 & 0 & 1 & 0\\
0 & 0 & 1 & -1 & 1
\end{bmatrix}
\end{align*}
Logo, o sistema linear dado é equivalente ao seguinte:
\[
\systeme[xyzw]{
x+w=0,
y+w=0,
z-w=1.
}
\]
Portanto, toda solução tem a forma $(x,y,z,w) = (-w,-w,1+w,w)$, para algum $w \in \R$ e o conjunto solução é $S = \{(-w,-w,1+w,w)\mid w \in \R\}$.

\Exercise[title={2,0}] Mostre que $P = \begin{bmatrix}
0 & 0 & 0 \\
4 & 8 & 1 \\
3 & 6 & 1 \\
1 & 2 & 0
\end{bmatrix}$ e $Q = \begin{bmatrix}
1 & 2 & 0 \\
3 & 6 & 4 \\
3 & 6 & 2 \\
2 & 4 & 0
\end{bmatrix}$ têm a mesma forma escalonada reduzida.
\Answer Pelo processo de eliminação gaussiana, chega-se às seguintes matrizes:
\begin{align*}
P =
\begin{bmatrix}
0 & 0 & 0 \\
4 & 8 & 1 \\
3 & 6 & 1 \\
1 & 2 & 0
\end{bmatrix}
\rightarrow
\begin{bmatrix}
1 & 2 & 0 \\
4 & 8 & 1 \\
3 & 6 & 1 \\
0 & 0 & 0
\end{bmatrix}
\rightarrow
\begin{bmatrix}
1 & 2 & 0 \\
0 & 0 & 1 \\
0 & 0 & 1 \\
0 & 0 & 0
\end{bmatrix}
\rightarrow
\begin{bmatrix}
1 & 2 & 0 \\
0 & 0 & 1 \\
0 & 0 & 0 \\
0 & 0 & 0
\end{bmatrix}
\end{align*}
\begin{align*}
Q =
\begin{bmatrix}
1 & 2 & 0 \\
3 & 6 & 4 \\
3 & 6 & 2 \\
2 & 4 & 0
\end{bmatrix}
\rightarrow
\begin{bmatrix}
1 & 2 & 0 \\
0 & 0 & 4 \\
0 & 0 & 2 \\
0 & 0 & 0
\end{bmatrix}
\rightarrow
\begin{bmatrix}
1 & 2 & 0 \\
0 & 0 & 1 \\
0 & 0 & 2 \\
0 & 0 & 0
\end{bmatrix}
\rightarrow
\begin{bmatrix}
1 & 2 & 0 \\
0 & 0 & 1 \\
0 & 0 & 0 \\
0 & 0 & 0
\end{bmatrix}
\end{align*}
Portanto, $P$ e $Q$ têm a mesma forma escalonada reduzida.

\Exercise[title={2,0}] Se $N \in \R^{2 \times 3}$, qual deve ser a ordem de uma matriz $X$ para que ocorra $X\cdot N \in \R^{3 \times 3}$?  Encontre, se existirem, todas as matrizes $X$ tais que $X \cdot N = I_{3\times 3}$, supondo que $N = \begin{bmatrix}
0 & 1 & 0 \\
2 & 3 & 1
\end{bmatrix}$.
\Answer Para que uma matriz $X \in \R^{m \times n}$ possa ser multiplicada à esquerda de $N \in \R^{2 \times 3}$, é preciso que $n = 2$, e neste caso, $X \cdot N$ terá ordem $m \times 3$. Assim, para que ocorra $X\cdot N \in \R^{3 \times 3}$, é preciso que $m=3$. Supondo que $
X = \begin{bmatrix}
x & y \\ z & w \\ r & s
\end{bmatrix}
\in \R^{3 \times 2}$
fosse tal que $X \cdot N = I_{3\times 3}$, teríamos:
\[
X \cdot N
= \begin{bmatrix}
x & y \\ z & w \\ r & s
\end{bmatrix}
\cdot
\begin{bmatrix}
0 & 1 & 0 \\
2 & 3 & 1
\end{bmatrix}
=
\begin{bmatrix}
2y & x+3y & y \\
2w & z+3w & w \\
2s & r+3s & s
\end{bmatrix}
=
\begin{bmatrix}
1 & 0 & 0 \\
0 & 1 & 0 \\
0 & 0 & 1
\end{bmatrix}
\]
Comparando as matrizes termo a termo, conclui-se, em particular, que $y \in \R$ deveria satisfazer simultaneamente $2y = 1$ e $y=0$, o que é impossível. Logo, não existe uma matriz $X$ nas condições propostas.

\Exercise[title={2,0}] Suponha que $A \in \R^{4 \times 4}$ seja uma matriz tal que $\det(A) = 6$ e que $B =
\begin{bmatrix}
1 & 2 & 0 & 2\\
2 & 0 & 2 & 0\\
0 & 3 & 0 & 3\\
0 & 2 & 4 & 0
\end{bmatrix}$. Calcule $\det(A^{-1} \cdot B^T)$.
\Answer Pelas propriedades do determinante, sabe-se que
\[
\det(A^{-1} \cdot B^T)
=\det(A^{-1}) \cdot \det(B^T)
=\frac{1}{\det(A)} \cdot \det(B).
\]
Além disso, subtraindo o dobro da primeira linha da segunda linha, e expandindo o determinante da matriz resultante pela primeira coluna, resulta que:
\[
\det(B) =
\begin{vmatrix}
1 & 2 & 0 & 2\\
2 & 0 & 2 & 0\\
0 & 3 & 0 & 3\\
0 & 2 & 4 & 0
\end{vmatrix}
=
\begin{vmatrix}
1 & 2 & 0 & 2\\
0 & -4 & 2 & -4\\
0 & 3 & 0 & 3\\
0 & 2 & 4 & 0
\end{vmatrix}
=
1 \cdot (-1)^{1+1}
\begin{vmatrix}
-4 & 2 & -4\\
3 & 0 & 3\\
2 & 4 & 0
\end{vmatrix}
=12.
\]
Portanto, $\det(A^{-1} \cdot B^T)
=\dfrac{\det(B)}{\det(A)}
=\dfrac{12}{6}
=2$.


\Exercise[title={2,0}] Prove ou dê um contra-exemplo:
\begin{enumerate}
\item Toda matriz que não é simétrica é, obrigatoriamente, antissimétrica.
\item Toda matriz antissimétrica é singular (ou seja, não é inversível).
\end{enumerate}
\Answer
\begin{enumerate}
\item A afirmação é falsa pois, por exemplo, a matriz
$A =
\begin{bmatrix}
1 & 2 \\ 3 & 4
\end{bmatrix}$
não é simétrica, mas também não é antissimétrica.

\item A afirmação é falsa pois, por exemplo, a matriz
$B =
\begin{bmatrix}
0 & 1 \\ -1 & 0
\end{bmatrix}$
é antissimétrica e, apesar disso, $\det(B) = 1 \neq 0$, ou seja, $B$ é inversível.

\end{enumerate}
\end{ExerciseList}

\vfill
\begin{center}
BOA PROVA!
\end{center}

\newpage
\restoregeometry
\section*{Respostas}
\shipoutAnswer
\end{document}
