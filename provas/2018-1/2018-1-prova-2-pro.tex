\documentclass[12pt,a4paper]{article}
\usepackage{cmap} % Makes the PDF copiable. See http://tex.stackexchange.com/a/64198/25761
\usepackage[T1]{fontenc}
\usepackage[brazil]{babel}
\usepackage[utf8]{inputenc}
\usepackage{amsmath}
\usepackage{amsfonts}
\usepackage{amssymb}
\usepackage{amsthm}
\usepackage{textcomp} % \degree
\usepackage{gensymb} % \degree
\usepackage[usenames,svgnames,dvipsnames]{xcolor}
\usepackage{hyperref}
\usepackage{multicol}
\usepackage{graphicx}
\usepackage[margin=2cm]{geometry}
\usepackage{systeme}

\hypersetup{
    colorlinks = true,
    allcolors = {blue}
}

% TODO: Consider using exsheets
% http://linorg.usp.br/CTAN/macros/latex/contrib/exsheets/exsheets_en.pdf
%
% http://ctan.org/tex-archive/macros/latex/contrib/exercise/
% Options: answerdelayed,lastexercise,noanswer
\usepackage[answerdelayed,lastexercise]{exercise}

\addto\captionsbrazil{%
\def\listexercisename{Lista de exerc\'icios}%
\def\ExerciseName{Exerc\'icio}%
\def\AnswerName{Solu\c{c}\~ao do exerc\'icio}%
\def\ExerciseListName{Ex.}%
\def\AnswerListName{Solu\c{c}\~ao}%
\def\ExePartName{Parte}%
\def\ArticleOf{de\ }%
}

\renewcommand{\ExerciseHeaderTitle}{(\ExerciseTitle)\ }
\renewcommand{\ExerciseListHeader}{%\ExerciseHeaderDifficulty%
\textbf{%\ExerciseListName\
\ExerciseHeaderNB.\ %
%\ --- \
\ExerciseHeaderTitle}%
%\ExerciseHeaderOrigin
\ignorespaces}
\renewcommand{\AnswerListHeader}{\textbf{\ExerciseHeaderNB.\ (\AnswerListName)\ }}

\newcommand*\ger[1]{\operatorname{ger}\left\{#1\right\}}
\newcommand*\R{\mathbb{R}}

% Loop Space / CC BY-SA-3.0 / https://tex.stackexchange.com/a/2238/25761
\newenvironment{amatrix}[1]{%
  \left[\begin{array}{@{}*{#1}{c}|c@{}}
}{%
  \end{array}\right]
}

% Loop Space / CC BY-SA-3.0 / https://tex.stackexchange.com/a/3164/25761
%--------grstep
% For denoting a Gauss' reduction step.
% Use as: \grstep{\rho_1+\rho_3} or \grstep[2\rho_5 \\ 3\rho_6]{\rho_1+\rho_3}
\newcommand{\grstep}[2][\relax]{%
   \ensuremath{\mathrel{
       {\mathop{\longrightarrow}\limits^{#2\mathstrut}_{
                                     \begin{subarray}{l} #1 \end{subarray}}}}}}

\renewcommand{\theenumi}{\alph{enumi}}
\renewcommand\labelenumi{(\theenumi) }

\newcommand*\tipo{Prova II}
\newcommand*\turma{PRO112-02U}
\newcommand*\disciplina{ALI0001}
\newcommand*\eu{Helder G. G. de Lima}
\newcommand*\data{23/04/2018}

\author{\eu}
\title{\tipo - \disciplina}
\date{\data}

\begin{document}
\thispagestyle{empty}
\newgeometry{margin=2cm,bottom=0.5cm}
\begin{center}
\includegraphics[width=9.0cm]{marca} \\
\textbf{\tipo\ (\disciplina / \turma)} \\
Prof. \eu \footnote{
Este é um material de acesso livre distribuído sob os termos da licença \href{https://creativecommons.org/licenses/by-sa/4.0/deed.pt_BR}{Creative Commons Atribuição-CompartilhaIgual 4.0 Internacional}}
\end{center}

\noindent Nome do(a) aluno(a): \underline{\hspace{9,7cm}} Data: \underline{\data}

%\section*{Instruções}
\begin{center}\fbox{
\begin{minipage}{14cm}

{\footnotesize
\begin{itemize}
\renewcommand{\theenumi}{\Roman{enumi}}
\item Identifique-se em todas as folhas.
\item Mantenha o celular e os demais equipamentos eletrônicos desligados durante a prova.
\item Resolva (integralmente) apenas os itens de que precisar para somar 10,0 pontos.
\end{itemize}
}

\end{minipage}
}
\end{center}

\section*{Questões}
\begin{ExerciseList}
\Exercise[title={2,0}] Utilize os axiomas ou propriedades de espaço vetorial para mostrar que:
\begin{center}
``Para todo espaço vetorial $V$, e todo vetor $u \in V$, vale $0 \cdot u = \vec{0}$.''
\end{center}
\Answer \textbf{Solução 1}: Dado um vetor $u$ de um espaço vetorial $V$, tem-se:
\[
0 \cdot u
= (0 + 0) \cdot u
= 0 \cdot u + 0 \cdot u.
\]
Então, somando o vetor $-(0 \cdot u)$ em ambos os membros, conclui-se que:
\[
\vec{0}
= -(0 \cdot u) + 0 \cdot u
= -(0 \cdot u) + [0 \cdot u + 0 \cdot u]
= [-(0 \cdot u) + 0 \cdot u] + 0 \cdot u
= \vec{0} + 0 \cdot u
= 0 \cdot u.
\]

\textbf{Solução 2}: Dado um vetor $u$ de um espaço vetorial $V$, tem-se:
\[
0 \cdot u
= (1 + (-1)) \cdot u
= 1 \cdot u + (-1) u
= u + (- u)
= \vec{0}.
\]

\textbf{Solução 3}: Como o vetor nulo é o único vetor que somado com um vetor $u$ resulta em $u$, basta mostrar que $u + (0 \cdot u) = u$ e então obrigatoriamente $0 \cdot u = \vec{0}$. E, de fato, tem-se:
\[
u + 0 \cdot u
= 1 \cdot u + 0 \cdot u
= (1 + 0) \cdot u
= 1 \cdot u
= u.
\]


\Exercise[title={2,0}] Determine quais dos conjuntos $U$ a seguir são subespaços do espaço vetorial $V$. Justifique suas respostas.
\begin{enumerate}
\item $U = \{ (x,y,z) \in \R^3 \mid x^2 + y^2 = 0 \}$ e $V = \R^3$.
\item $U = \{ M \in \R^{2 \times 2} \mid M^TM = I_{2\times2} \}$ e $V = \R^{2\times 2}$ (espaço das matrizes $2 \times 2$).
\end{enumerate}
\Answer
\begin{enumerate}
\item Como $x^2 + y^2 = 0$ se, e somente se, $x = y = 0$, pode-se dizer que $U = \{ (0,0,z) \in \R^3 \}$. Assim, se $u_1 = (0,0,z_1)$ e $u_2 = (0,0,z_2)$ então $u_1 + u_2 = (0,0,z_1) + (0,0,z_2) = (0,0,z_1+z_2) \in U$, o que significa que $U$ é fechado para a soma. Do mesmo modo, se $k \in \R$ então $k u_1 = k (0,0,z_1) = (0,0,kz_1) \in U$, ou seja, $U$ é fechado para a multiplicação por escalar. Assim, $U$ é um subespaço de $\R^3$.
\item Como ${0_{2\times 2}}^T 0_{2\times 2} = 0_{2\times 2} \neq I_{2 \times 2}$, conclui-se que $0_{2\times 2} \not\in M$. Portanto, $U$ não é um subespaço vetorial de $\R^{2 \times 2}$.
\end{enumerate}

\Exercise[title={2,0}] A afirmação a seguir é verdadeira ou falsa? Justifique detalhadamente a sua conclusão.
\begin{center}
``Existe um subespaço de $P_2$ que contém os vetores $q_1(x) = x+ 1$ e $q_2(x) = 2x^2 + 2$,\\ mas não contém $q_3(x) = 2x^2 + x + 3$.''
\end{center}
\Answer Observando que $q_1(x) + q_2(x) = (x+ 1) + (2x^2 + 2) = 2x^2 + x + 3 = q_3(x)$ e que todo subespaço deve ser fechado para a soma, conclui-se que a afirmação é falsa, pois todo subespaço de $P_2$ que contenha os vetores $q_1$ e $q_2$ deve, obrigatoriamente, conter sua soma $q_3$.

\Exercise[title={2,0}] Sejam
$W_1 = \{ (x,y,x,x) \in \R^4 \mid x \in \R, y \in \R \}$
e
$W_2 = \{ (r,s,t,u) \in \R^4 \mid t = u = -r\}$.
Mostre que $\R^4$ não é soma direta de $W_1$ e $W_2$. Justifique cada uma de suas afirmações.
\Answer Pela definição de soma direta, basta mostrar que $W_1 \cap W_2 \neq \{ \vec{0} \}$ ou que $\R^4$ é diferente de $W_1 + W_2$.

\textbf{Solução 1}: Por inspeção, verifica-se que $v=(0,1,0,0)$ pertence tanto a $W_1$ quanto a $W_2$. Portanto, $W_1 \cap W_2 \neq \{ \vec{0} \}$ e $\R^4$ não é soma direta de $W_1$ e $W_2$.

\textbf{Solução 2}: Uma alternativa à solução anterior é calcular a interseção explicitamente $W_1 \cap W_2$, e mostrar que não é trivial. Para isso, suponha que um vetor $v = (x,y,x,x)$ de $W_1$ também esteja em $W_2$. Então, pela definição de $W_2$, tem-se $x = -x$, o que implica que $x = 0$. Consequentemente, $v = (0,y,0,0) = y (0,1,0,0)$ e resulta que $W_1 \cap W_2 = \ger{(0,1,0,0)} \neq \{ \vec{0} \}$. Logo, $\R^4$ não é soma direta de $W_1$ e $W_2$.

\textbf{Solução 3}: Basta observar que o vetor $u = (0,0,0,1) \in \R^4$ não pertence à soma $W_1 + W_2$, pois neste caso $\R^4 \neq W_1 + W_2$. De fato, se $u \in W_1 + W_2$ então existiriam vetores $w_1 = (x,y,x,x) \in W_1$ e $w_2 = (r,s,-r,-r) \in W_2$ tais que $u = w_1 + w_2$, isto é, $(0,0,0,1) = (x+r,y+s,x-r,x-r)$. Mas isso resultaria em uma contradição, pois ocorreria $0 = x-r = 1$. Logo, $\R^4 \neq W_1 + W_2$.

\Exercise[title={2,0}] Se $S$ é o espaço das matrizes $A = (a_{ij}) \in \R^{2\times 2}$ que são simétricas e tais que $a_{11} = 3a_{22}$, qual é a dimensão de $S$? Justifique todas as etapas necessárias para chegar ao resultado.
\Answer Sendo $S = \left\{
\begin{bmatrix}
a_{11} & a_{12} \\ a_{21} & a_{22}
\end{bmatrix} \in \R^{2 \times 2} \mid a_{12} = a_{21} \text{ e }
a_{11} = 3a_{22}
\right\}$,
tem-se $v \in S$ se, e somente se,
\[
v
=\begin{bmatrix}
3a_{22} & a_{21} \\ a_{21} & a_{22}
\end{bmatrix}
=
a_{22}
\begin{bmatrix}
3 & 0 \\ 0 & 1
\end{bmatrix}
+
a_{21}
\begin{bmatrix}
0 & 1 \\ 1 & 0
\end{bmatrix}.
\]
Então $S = \ger{
\begin{bmatrix}
3 & 0 \\ 0 & 1
\end{bmatrix},
\begin{bmatrix}
0 & 1 \\ 1 & 0
\end{bmatrix}
}$. Como estas matrizes não são múltiplas uma da outra, elas são linearmente independentes e formam uma base de $S$. Portanto, $\operatorname{dim} (S) = 2$.

\Exercise[title={2,0}] Para que valor(es) de $\mathbf{k} \in \R$ o conjunto $C \subset \R^{2 \times 2}$ a seguir é linearmente dependente?
\[
C = \left\{
\begin{bmatrix}
1 & 2 \\ 3 & \mathbf{k}
\end{bmatrix},
\begin{bmatrix}
4 & 3 \\ 2 & \mathbf{k-1}
\end{bmatrix},
\begin{bmatrix}
6 & 7 \\ 8 & 9
\end{bmatrix}
\right\}.
\]
Justifique cuidadosamente sua resposta.
\Answer \textbf{Solução 1}: As matrizes dadas serão linearmente dependentes se, e somente se, existir mais de uma solução para a equação
\[
x
\begin{bmatrix}
1 & 2 \\ 3 & \mathbf{k}
\end{bmatrix}
+y
\begin{bmatrix}
4 & 3 \\ 2 & \mathbf{k-1}
\end{bmatrix}
+z
\begin{bmatrix}
6 & 7 \\ 8 & 9
\end{bmatrix}
=
\begin{bmatrix}
0 & 0 \\ 0 & 0
\end{bmatrix},
\]
ou seja, se o seguinte sistema linear for possível e indeterminado:
\[
\begin{cases}
x +4y+6z &=0,\\
2x+3y+7z &=0,\\
3x+2y+8z &=0,\\
\mathbf{k}x+(\mathbf{k-1})y+9z &=0
\end{cases}
\]
Escalonando a matriz de coeficientes do sistema, obtém-se:
\[
\begin{bmatrix}
1 & 4 & 6\\
2 & 3 & 7\\
3 & 2 & 8\\
k & k-1 & 9\\
\end{bmatrix}
\rightarrow
\begin{bmatrix}
1 & 4 & 6\\
0 & -5 & -5\\
0 & -10 & -10\\
0 & -3k-1 & 9-6k\\
\end{bmatrix}
\rightarrow
\begin{bmatrix}
1 & 4 & 6\\
0 & 1 & 1\\
0 & -10 & -10\\
0 & -3k-1 & 9-6k\\
\end{bmatrix}
\rightarrow
\begin{bmatrix}
1 & 4 & 6\\
0 & 1 & 1\\
0 & 0 & 0\\
0 & 0 & 10-3k\\
\end{bmatrix}
\]

Assim, se $k=10/3$, o posto da matriz de coeficientes é $2$ e o sistema é possível e indeterminado, ou seja, $C$ é um conjunto /linearmente independente de matrizes.

\textbf{Solução 2}: As matrizes dadas serão linearmente dependentes se, e somente se, a terceira puder ser escrita como combinação linear das demais (por quê?), isto é, se existirem $a,b \in \R$ tais que
\[
a
\begin{bmatrix}
1 & 2 \\ 3 & \mathbf{k}
\end{bmatrix}
+b
\begin{bmatrix}
4 & 3 \\ 2 & \mathbf{k-1}
\end{bmatrix}
=
\begin{bmatrix}
6 & 7 \\ 8 & 9
\end{bmatrix}.
\]
Comparando as entradas da primeira linha, conclui-se que
\[
\begin{cases}
a +4b &=6,\\
2a+3b &=7
\end{cases}
\]
ou seja, que $a = 2$ e $b = 1$. Assim, $k$ deve satisfazer:

\[
2
\begin{bmatrix}
1 & 2 \\ 3 & \mathbf{k}
\end{bmatrix}
+
\begin{bmatrix}
4 & 3 \\ 2 & \mathbf{k-1}
\end{bmatrix}
=
\begin{bmatrix}
6 & 7 \\ 8 & 9
\end{bmatrix},
\]
ou seja, $2k + (k-1) = 9$. Portanto, $k = 10/3$.
\end{ExerciseList}

\begin{center}
BOA PROVA!
\end{center}

\newpage
\restoregeometry
\section*{Respostas}
\shipoutAnswer
\end{document}
