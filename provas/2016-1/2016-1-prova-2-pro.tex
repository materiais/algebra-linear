\documentclass[12pt,a4paper]{article}
\usepackage{cmap} % Makes the PDF copiable. See http://tex.stackexchange.com/a/64198/25761
\usepackage[T1]{fontenc}
\usepackage[brazil]{babel}
\usepackage[utf8]{inputenc}
\usepackage{amsmath}
\usepackage{amsfonts}
\usepackage{amssymb}
\usepackage{amsthm}
\usepackage{textcomp} % \degree
\usepackage{gensymb} % \degree
\usepackage[usenames,svgnames,dvipsnames]{xcolor}
\usepackage{hyperref}
\usepackage{multicol}
\usepackage{graphicx}
\usepackage[margin=2cm]{geometry}
\usepackage{systeme}

\hypersetup{
    colorlinks = true,
    allcolors = {blue}
}

% TODO: Consider using exsheets
% http://linorg.usp.br/CTAN/macros/latex/contrib/exsheets/exsheets_en.pdf
%
% http://ctan.org/tex-archive/macros/latex/contrib/exercise/
% Options: answerdelayed,lastexercise,noanswer
\usepackage[answerdelayed,lastexercise]{exercise}

\addto\captionsbrazil{%
\def\listexercisename{Lista de exerc\'icios}%
\def\ExerciseName{Exerc\'icio}%
\def\AnswerName{Solu\c{c}\~ao do exerc\'icio}%
\def\ExerciseListName{Ex.}%
\def\AnswerListName{Solu\c{c}\~ao}%
\def\ExePartName{Parte}%
\def\ArticleOf{de\ }%
}

\renewcommand{\ExerciseHeaderTitle}{(\ExerciseTitle)\ }
\renewcommand{\ExerciseListHeader}{%\ExerciseHeaderDifficulty%
\textbf{%\ExerciseListName\
\ExerciseHeaderNB.\ %
%\ --- \
\ExerciseHeaderTitle}%
%\ExerciseHeaderOrigin
\ignorespaces}
\renewcommand{\AnswerListHeader}{\textbf{\ExerciseHeaderNB.\ (\AnswerListName)\ }}

\newcommand{\fixme}{{\color{red}(...)}}
\newcommand*\ger[1]{\operatorname{ger}\left\{#1\right\}}
\newcommand*\R{\mathbb{R}}

% Loop Space / CC BY-SA-3.0 / https://tex.stackexchange.com/a/2238/25761
\newenvironment{amatrix}[1]{%
  \left[\begin{array}{@{}*{#1}{c}|c@{}}
}{%
  \end{array}\right]
}

% Loop Space / CC BY-SA-3.0 / https://tex.stackexchange.com/a/3164/25761
%--------grstep
% For denoting a Gauss' reduction step.
% Use as: \grstep{\rho_1+\rho_3} or \grstep[2\rho_5 \\ 3\rho_6]{\rho_1+\rho_3}
\newcommand{\grstep}[2][\relax]{%
   \ensuremath{\mathrel{
       {\mathop{\longrightarrow}\limits^{#2\mathstrut}_{
                                     \begin{subarray}{l} #1 \end{subarray}}}}}}

\renewcommand{\theenumi}{\alph{enumi}}
\renewcommand\labelenumi{(\theenumi) }

\newcommand*\tipo{Prova II}
\newcommand*\turma{PRO112-02A}
\newcommand*\disciplina{ALI0001}
\newcommand*\eu{Helder G. G. de Lima}
\newcommand*\data{26/04/2016}

\author{\eu}
\title{\tipo - \disciplina}
\date{\data}

\begin{document}
\thispagestyle{empty}
\newgeometry{margin=2cm,bottom=0.5cm}
\begin{center}
\includegraphics[width=9.0cm]{marca} \\
\textbf{\tipo\ (\disciplina / \turma)} \\
Prof. \eu\footnote{
Este é um material de acesso livre distribuído sob os termos da licença \href{https://creativecommons.org/licenses/by-sa/4.0/deed.pt_BR}{Creative Commons Atribuição-CompartilhaIgual 4.0 Internacional}}
\end{center}

\noindent Nome do(a) aluno(a): \underline{\hspace{9,7cm}} Data: \underline{\data}

%\section*{Instruções}
\begin{center}\fbox{
\begin{minipage}{14cm}

{\footnotesize
\begin{itemize}
\renewcommand{\theenumi}{\Roman{enumi}}
\item Identifique-se em todas as folhas.
\item Mantenha o celular e os demais equipamentos eletrônicos desligados durante a prova.
\item Anule \textsc{\textbf{uma}} das 5 questões (apenas 4 serão corrigidas): \framebox(30,10){}
\end{itemize}
}

\end{minipage}
}
\end{center}

%\section*{Questões}
\begin{ExerciseList}
\Exercise[title={2,5}]
Analise as afirmações a seguir e decida quais são verdadeiras.
\begin{itemize}
\item Se for \textbf{verdadeira}, explique-a com um argumento lógico convincente, e os devidos cálculos.
\item Se for \textbf{falsa}, forneça um exemplo simples que mostre por que é falsa.
\end{itemize}
\begin{enumerate}
\item (0,8) Nenhuma matriz quadrada $A$ que satisfaça $A^{3} = 0$ pode ser inversível.
\item (0,8) As funções constantes são um subespaço do espaço vetorial das funções de $\R$ em $\R$.
\item (0,9) Se $A = \begin{bmatrix}
1 & -2  & -2 & -1\\
2 &  1  &  1 & 0
\end{bmatrix} \in M_{2 \times 4}(\R)$ então $\det(A^T A) = \det(A A^T)$
\end{enumerate}
\Answer
\begin{enumerate}
\item \textbf{Verdadeiro}. Se $A^3 = 0$ Então $\det(A^3) = \det( 0 ) = 0$, isto é, $ \det(A)^3 = 0$ e portanto $\det(A) = 0$, o que significa que $A$ não é inversível.
\item \textbf{Verdadeiro}. A soma de funções constantes é uma função constante, e ao multiplicar uma função constante por um escalar qualquer, o resultado também é uma função constante.
\item \textbf{Falso}. Como $A A^T =  \begin{bmatrix}
10 & -2\\
-2 & 6
\end{bmatrix}$, tem-se $\det(AA^T) = 56$. Por outro lado, $\det(A^TA) = 0$ pois $A^T A =  \begin{bmatrix}
 5 & 0 & 0 & -1\\
 0 & 5 & 5 &  2 \\
 0 & 5 & 5 &  2 \\
-1 & 2 & 2 &  1
\end{bmatrix}$ tem duas linhas idênticas. Assim, $\det(AA^T) \neq \det(A^TA)$.
\end{enumerate}

\Exercise[title={2,5}] Considere os seguintes subespaços vetoriais de $P_3$ (os polinômios de grau $\leq 3$):
\begin{align*}
U &= \left\{ q(x) = a + bx + cx^2 + dx^3 \in P_3 \mid q(-1) = 0 \right\},\\
W &= \left\{ q(x) = a + bx + cx^2 + dx^3 \in P_3 \mid q( 1) = 0 \right\}.
\end{align*}
\begin{enumerate}
\item (1,2) Encontre um conjunto de vetores que gera o espaço $U \cap W$
\item (1,3) Obtenha uma base (e explique por que é base) e a dimensão de $U \cap W$.
\end{enumerate}
\Answer  \fixme


\Exercise[title={2,5}]
Verifique se $W$ satisfaz as condições para ser um subespaço vetorial de $V$, sendo:
\begin{enumerate}
\item (1,3) $W = \left\{ \begin{bmatrix}
x & y \\
z & w
\end{bmatrix} \in M_{2\times 2} \mid \begin{bmatrix}
2 & 6 \\
-2 & -6
\end{bmatrix} \begin{bmatrix}
x & y \\
z & w
\end{bmatrix} = 0 \right\}$, sendo $V = M_{2 \times 2}$.


\item (1,2) $W = \{ (x,y,z) \in \R^3 \mid x + y + z \geq 0 \}$, sendo $V = \R^3$.
\end{enumerate}
\Answer \fixme


\Exercise[title={2,5}]
Sejam $M= \begin{bmatrix}
1 & -1 \\ -1 & 0
\end{bmatrix}$, $N= \begin{bmatrix}
1 & 1 \\ 1 & 0
\end{bmatrix}$, $P= \begin{bmatrix}
0 & 1 \\ 1 & 0
\end{bmatrix}$ e $Q= \begin{bmatrix}
0 & 3 \\ 3 & -3
\end{bmatrix}$ vetores de $M_{2 \times 2}$.
\begin{enumerate}
\item (1,0) Mostre que eles são L.D. escrevendo algum vetor como combinação linear dos demais.
\item (1,0) Encontre uma base do subespaço vetorial $W = \ger{ M, N, P, Q }$.
\item (0,5) Determine se a matriz $A = \begin{bmatrix}
1 & 1 \\ 1 & 1
\end{bmatrix}$ pertence a $W$.
\end{enumerate}
\Answer  \fixme


\Exercise[title={2,5}] Seja $A = \begin{bmatrix}
2 & 0 & 4 & -6\\
0 & 1 & 2 &  3\\
1 & 3 & 8 &  1\\
2 & 1 & 6 & -3
\end{bmatrix} \in M_{4\times 4}(\R)$. Encontre uma base e a dimensão de $V$ se:
\begin{enumerate}
\item (1,3) $V$ é o espaço linha de $A$.
\item (1,2) $V$ é o núcleo (ou espaço nulo) de $A$.
\end{enumerate}

\Answer  A forma escalonada reduzida de $A$ é $\begin{bmatrix}
1 & 0 & 2 & 0 \\
0 & 1 & 2 & 0 \\
0 & 0 & 0 & 1 \\
0 & 0 & 0 & 0
\end{bmatrix}$ então uma base para $L(A)$ é
\[
B =
\left\{
\begin{bmatrix}
1 & 0 & 2 & 0
\end{bmatrix},
\begin{bmatrix}
0 & 1 & 2 & 0
\end{bmatrix},
\begin{bmatrix}
0 & 0 & 0 & 1
\end{bmatrix}
\right\}
\]
e $\dim( L(A) ) = 3$. Por outro lado, uma base de $N(A)$ é $B_2 =
\left\{
\begin{bmatrix}
-2\\-2\\1\\0
\end{bmatrix}
\right\}$ e $\dim( N(A) ) = 1$.
\end{ExerciseList}

\begin{center}
BOA PROVA!
\end{center}

\newpage
\restoregeometry
\section*{Respostas}
\shipoutAnswer
\end{document}
