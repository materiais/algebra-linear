\documentclass[12pt,a4paper]{article}
\usepackage{cmap} % Makes the PDF copiable. See http://tex.stackexchange.com/a/64198/25761
\usepackage[T1]{fontenc}
\usepackage[brazil]{babel}
\usepackage[utf8]{inputenc}
\usepackage{amsmath}
\usepackage{amsfonts}
\usepackage{amssymb}
\usepackage{amsthm}
\usepackage{textcomp} % \degree
\usepackage{gensymb} % \degree
\usepackage[usenames,svgnames,dvipsnames]{xcolor}
\usepackage{hyperref}
\usepackage{multicol}
\usepackage{graphicx}
\usepackage[margin=2cm]{geometry}
\usepackage{systeme}

\hypersetup{
    colorlinks = true,
    allcolors = {blue}
}

% TODO: Consider using exsheets
% http://linorg.usp.br/CTAN/macros/latex/contrib/exsheets/exsheets_en.pdf
%
% http://ctan.org/tex-archive/macros/latex/contrib/exercise/
% Options: answerdelayed,lastexercise,noanswer
\usepackage[answerdelayed,lastexercise]{exercise}

\addto\captionsbrazil{%
\def\listexercisename{Lista de exerc\'icios}%
\def\ExerciseName{Exerc\'icio}%
\def\AnswerName{Solu\c{c}\~ao do exerc\'icio}%
\def\ExerciseListName{Ex.}%
\def\AnswerListName{Solu\c{c}\~ao}%
\def\ExePartName{Parte}%
\def\ArticleOf{de\ }%
}

\renewcommand{\ExerciseHeaderTitle}{(\ExerciseTitle)\ }
\renewcommand{\ExerciseListHeader}{%\ExerciseHeaderDifficulty%
\textbf{%\ExerciseListName\
\ExerciseHeaderNB.\ %
%\ --- \
\ExerciseHeaderTitle}%
%\ExerciseHeaderOrigin
\ignorespaces}
\renewcommand{\AnswerListHeader}{\textbf{\ExerciseHeaderNB.\ (\AnswerListName)\ }}

\newcommand{\fixme}{{\color{red}(...)}}
\newcommand*\im[1]{\operatorname{Im}\left(#1\right)}
\newcommand*\R{\mathbb{R}}

% Loop Space / CC BY-SA-3.0 / https://tex.stackexchange.com/a/2238/25761
\newenvironment{amatrix}[1]{%
  \left[\begin{array}{@{}*{#1}{c}|c@{}}
}{%
  \end{array}\right]
}

% Loop Space / CC BY-SA-3.0 / https://tex.stackexchange.com/a/3164/25761
%--------grstep
% For denoting a Gauss' reduction step.
% Use as: \grstep{\rho_1+\rho_3} or \grstep[2\rho_5 \\ 3\rho_6]{\rho_1+\rho_3}
\newcommand{\grstep}[2][\relax]{%
   \ensuremath{\mathrel{
       {\mathop{\longrightarrow}\limits^{#2\mathstrut}_{
                                     \begin{subarray}{l} #1 \end{subarray}}}}}}

\renewcommand{\theenumi}{\alph{enumi}}
\renewcommand\labelenumi{(\theenumi) }

\newcommand*\tipo{Prova III}
\newcommand*\turma{MAT102-03U/LEF102-02U}
\newcommand*\disciplina{ALI0001}
\newcommand*\eu{Helder G. G. de Lima}
\newcommand*\data{31/05/2016}

\author{\eu}
\title{\tipo - \disciplina}
\date{\data}

\begin{document}
\thispagestyle{empty}
\newgeometry{margin=2cm,bottom=0.5cm}
\begin{center}
\includegraphics[width=9.0cm]{marca} \\
\textbf{\tipo\ (\disciplina / \turma)} \\
Prof. \eu\footnote{
Este é um material de acesso livre distribuído sob os termos da licença \href{https://creativecommons.org/licenses/by-sa/4.0/deed.pt_BR}{Creative Commons BY-SA 4.0}}
\end{center}

\noindent Nome do(a) aluno(a): \underline{\hspace{9,7cm}} Data: \underline{\data}

%\section*{Instruções}
\begin{center}\fbox{
\begin{minipage}{14cm}

{\footnotesize
\begin{itemize}
\renewcommand{\theenumi}{\Roman{enumi}}
\item Identifique-se em todas as folhas.
\item Mantenha o celular e os demais equipamentos eletrônicos desligados durante a prova.
\item Anule \textsc{\textbf{uma}} das 5 questões (apenas 4 serão corrigidas): \framebox(30,10){}
\end{itemize}
}

\end{minipage}
}
\end{center}

%\section*{Questões}
\begin{ExerciseList}
\Exercise[title={2,5}] Sejam $B = \left\{ \begin{bmatrix}
0 & 0\\ 0 & -1
\end{bmatrix}, \begin{bmatrix}
0 & 0\\ 4 & 2
\end{bmatrix}, \begin{bmatrix}
2 & 0\\ 2 & -1
\end{bmatrix}\right\}$ e $C = \left\{ \begin{bmatrix}
0 & 0\\ 0 & 1
\end{bmatrix}, \begin{bmatrix}
0 & 0\\ 1 & 1
\end{bmatrix}, \begin{bmatrix}
1 & 0\\ 1 & 1
\end{bmatrix}\right\}$ bases do espaço vetorial $V$ das matrizes triangulares inferiores $2 \times 2$, isto é, $V = \left\{ \begin{bmatrix}
a & 0\\ b & c
\end{bmatrix} \mid a,b,c \in \R \right\}$.
\begin{enumerate}
\item (2,0) Determine as matrizes de mudança de base de $B$ para $C$ e de $C$ para $B$.
\item (0,5) Se $v \in V$ é o vetor cujas coordenadas na base $B$ são dadas por $[v]_B =
\begin{bmatrix}
3 \\
1 \\
4
\end{bmatrix}$, utilize o item anterior para encontrar as coordenadas de $v$ em relação à base $C$, ou seja, $[v]_C$.
\end{enumerate}
\Answer \fixme

\Exercise[title={2,5}]
Classifique como verdadeiro (e demonstre) ou falso (e dê um contraexemplo):
\begin{enumerate}
\item (1,2) Se $\dim{U} = \dim{V}$ então toda transformação linear $T: U \to V$ é um isomorfismo.
\item (1,3) Se a transformação linear $T: U \to V$ é um isomorfismo então $\dim{U} = \dim{V}$.
\end{enumerate}
\Answer \fixme

\Exercise[title={2,5}] Decida (e justifique) quais das seguintes funções são transformações lineares:
\begin{enumerate}
\item (1,2) $L_1: \R^3 \to \R$, definida por $L_1(a,b,c) = 2a(b-c)$.
\item (1,3) $L_2: M_{2 \times 2} \to M_{2 \times 2}$, definida por $L_2(X) = X^T + 3X$.
\end{enumerate}
\Answer \fixme


\Exercise[title={2,5}] Considere a função $L: P_2 \to \R^3$ definida por $L(a + bx +cx^2) = (2b, -4a+c, 0)$.
\begin{enumerate}
\item (1,0) Mostre que $L$ é uma transformação linear.
\item (1,5) Calcule as dimensões do núcleo e da imagem de $L$, isto é, $\dim(N(L))$ e $\dim(\im{L})$.
\end{enumerate}
\Answer \fixme



\Exercise[title={2,5}] Se $S_{2\times 2}$ é o espaço das matrizes simétricas e a transformação linear $T:S_{2\times 2} \to \R^3$ satisfaz
$
T\left( \begin{bmatrix}
1 & 0\\
0 & 0
\end{bmatrix} \right) = (1,1,1),
T\left( \begin{bmatrix}
0 & 1\\
1 & 0
\end{bmatrix} \right) = (0,1,1) \text{ e }
T\left( \begin{bmatrix}
0 & 0\\
0 & 1
\end{bmatrix} \right) = (0,1,0)$:
\begin{enumerate}
\item (0,5) Obtenha uma fórmula explicita da transformação linear $T$.
\item (1,0) Mostre que $T$ é um isomorfismo.
\item (1,0) Obtenha uma fórmula explícita para o isomorfismo inverso $T^{-1}: \R^3 \to S_{2 \times 2}$.
\end{enumerate}

\end{ExerciseList}

%\begin{center}
%BOA PROVA!
%\end{center}

%\newpage
%\restoregeometry
%\section*{Respostas}
%\shipoutAnswer
\end{document}
