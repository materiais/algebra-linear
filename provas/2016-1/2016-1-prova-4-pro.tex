\documentclass[12pt,a4paper]{article}
\usepackage{cmap} % Makes the PDF copiable. See http://tex.stackexchange.com/a/64198/25761
\usepackage[T1]{fontenc}
\usepackage[brazil]{babel}
\usepackage[utf8]{inputenc}
\usepackage{amsmath}
\usepackage{amsfonts}
\usepackage{amssymb}
\usepackage{amsthm}
\usepackage{textcomp} % \degree
\usepackage{gensymb} % \degree
\usepackage[usenames,svgnames,dvipsnames]{xcolor}
\usepackage{hyperref}
\usepackage{multicol}
\usepackage{graphicx}
\usepackage[margin=2cm]{geometry}
\usepackage{systeme}

\hypersetup{
    colorlinks = true,
    allcolors = {blue}
}

% TODO: Consider using exsheets
% http://linorg.usp.br/CTAN/macros/latex/contrib/exsheets/exsheets_en.pdf
%
% http://ctan.org/tex-archive/macros/latex/contrib/exercise/
% Options: answerdelayed,lastexercise,noanswer
\usepackage[answerdelayed,lastexercise]{exercise}

\addto\captionsbrazil{%
\def\listexercisename{Lista de exerc\'icios}%
\def\ExerciseName{Exerc\'icio}%
\def\AnswerName{Solu\c{c}\~ao do exerc\'icio}%
\def\ExerciseListName{Ex.}%
\def\AnswerListName{Solu\c{c}\~ao}%
\def\ExePartName{Parte}%
\def\ArticleOf{de\ }%
}

\renewcommand{\ExerciseHeaderTitle}{(\ExerciseTitle)\ }
\renewcommand{\ExerciseListHeader}{%\ExerciseHeaderDifficulty%
\textbf{%\ExerciseListName\
\ExerciseHeaderNB.\ %
%\ --- \
\ExerciseHeaderTitle}%
%\ExerciseHeaderOrigin
\ignorespaces}
\renewcommand{\AnswerListHeader}{\textbf{\ExerciseHeaderNB.\ (\AnswerListName)\ }}

\newcommand{\norm}[1]{\left|\left|{#1}\right|\right|}
\newcommand*\R{\mathbb{R}}

% Loop Space / CC BY-SA-3.0 / https://tex.stackexchange.com/a/2238/25761
\newenvironment{amatrix}[1]{%
  \left[\begin{array}{@{}*{#1}{c}|c@{}}
}{%
  \end{array}\right]
}

% Loop Space / CC BY-SA-3.0 / https://tex.stackexchange.com/a/3164/25761
%--------grstep
% For denoting a Gauss' reduction step.
% Use as: \grstep{\rho_1+\rho_3} or \grstep[2\rho_5 \\ 3\rho_6]{\rho_1+\rho_3}
\newcommand{\grstep}[2][\relax]{%
   \ensuremath{\mathrel{
       {\mathop{\longrightarrow}\limits^{#2\mathstrut}_{
                                     \begin{subarray}{l} #1 \end{subarray}}}}}}

\renewcommand{\theenumi}{\alph{enumi}}
\renewcommand\labelenumi{(\theenumi) }

\newcommand*\tipo{Prova IV}
\newcommand*\turma{PRO112-02A}
\newcommand*\disciplina{ALI0001}
\newcommand*\eu{Helder G. G. de Lima}
\newcommand*\data{28/06/2016}

\author{\eu}
\title{\tipo - \disciplina}
\date{\data}

\begin{document}
\thispagestyle{empty}
\newgeometry{margin=2cm,bottom=0.5cm}
\begin{center}
\includegraphics[width=9.0cm]{marca} \\
\textbf{\tipo\ (\disciplina / \turma)} \\
Prof. \eu\footnote{
Este é um material de acesso livre distribuído sob os termos da licença \href{https://creativecommons.org/licenses/by-sa/4.0/deed.pt_BR}{Creative Commons Atribuição-CompartilhaIgual 4.0 Internacional}}
\end{center}

\noindent Nome do(a) aluno(a): \underline{\hspace{9,7cm}} Data: \underline{\data}

%\section*{Instruções}
\begin{center}\fbox{
\begin{minipage}{14cm}

{\footnotesize
\begin{itemize}
\renewcommand{\theenumi}{\Roman{enumi}}
\item Identifique-se em todas as folhas.
\item Mantenha o celular e os demais equipamentos eletrônicos desligados durante a prova.
\item Anule \textsc{\textbf{uma}} das 5 questões (apenas 4 serão corrigidas): \framebox(30,10){}
\end{itemize}
}

\end{minipage}
}
\end{center}

%\section*{Questões}
\begin{ExerciseList}
\Exercise[title={2,5}] Seja $L: M_{2\times 2} \to M_{2\times 2}$ o operador linear definido por $L(X) = \frac{1}{2}(X - X^T)$.
\begin{enumerate}
\item (1,5) Obtenha os autovalores e autovetores de $L$.
\item (1,0) A matriz de $L$ em relação à base canônica de $M_{2\times 2}$ é diagonalizável? Explique.
\end{enumerate}
\Answer
\begin{enumerate}
\item Se $X =
\begin{bmatrix}
x & y \\
z & w
\end{bmatrix}$, então
$
L\left(
\begin{bmatrix}
x & y \\
z & w
\end{bmatrix}
\right)
= \frac{1}{2}\left(
\begin{bmatrix}
x & y \\
z & w
\end{bmatrix} -
\begin{bmatrix}
x & z \\
y & w
\end{bmatrix}\right)
=
\begin{bmatrix}
0 &
\frac{y-z}{2} \\
\frac{z-y}{2} & 0
\end{bmatrix}$.
\textbf{Solução 1}: Calculando $L$ nos vetores da base canônica de $M_{2\times 2}$, resulta que a matriz $A$ que representa $L$ nesta base é
\[
A = \begin{bmatrix}
0 & 0 & 0 & 0 \\
0 & 1/2 & -1/2 & 0 \\
0 & -1/2 & 1/2 & 0 \\
0 & 0 & 0 & 0
\end{bmatrix}.
\]
Então o polinômio característico de $A$ é
\[
p_A(x)
=
\begin{vmatrix}
x & 0 & 0 & 0 \\
0 & x-1/2 & 1/2 & 0 \\
0 & 1/2 & x-1/2 & 0 \\
0 & 0 & 0 & x
\end{vmatrix}
=
x
\begin{vmatrix}
x-1/2 & 1/2 & 0 \\
1/2 & x-1/2 & 0 \\
0 & 0 & x
\end{vmatrix}
=
x^2
\begin{vmatrix}
x-1/2 & 1/2 \\
1/2 & x-1/2
\end{vmatrix},
\]
ou seja, $p_A(x) = x^2( x^2 - x) = x^3(x-1)$ e os autovalores de $A$ são as raízes de $p_A(x)$, isto é, $c_1 = 0$ (com multiplicidade algébrica igual a 3) e $c_2 = 1$ (com multiplicidade 1).

As coordenadas de um autovetor $X =
\begin{bmatrix}
x & y \\
z & w
\end{bmatrix}$ associado ao autovalor $c_1 = 0$ devem ser soluções do seguinte sistema linear:
\begin{align*}
\begin{bmatrix}
0 & 0 & 0 & 0 \\
0 & -1/2 & 1/2 & 0 \\
0 & 1/2 & -1/2 & 0 \\
0 & 0 & 0 & 0
\end{bmatrix}
\begin{bmatrix}
x \\
y \\
z \\
w \\
\end{bmatrix}
=
\begin{bmatrix}
0 \\
0 \\
0 \\
0 \\
\end{bmatrix}
\Leftrightarrow
\begin{cases}
-\frac{y}{2}+\frac{z}{2} = 0\\
 \frac{y}{2}-\frac{z}{2} = 0
\end{cases}
\Leftrightarrow
y=z
\end{align*}
ou seja
\[
X =
\begin{bmatrix}
x & z \\
z & w
\end{bmatrix}
=
x
\begin{bmatrix}
1 & 0 \\
0 & 0
\end{bmatrix}
+z
\begin{bmatrix}
0 & 1 \\
1 & 0
\end{bmatrix}
+w
\begin{bmatrix}
0 & 0 \\
0 & 1
\end{bmatrix}.
\]
Já os autovetores associados ao autovalor $c_2 = 1$ devem ter coordenadas que satisfaçam
\begin{align*}
\begin{bmatrix}
1 & 0 & 0 & 0 \\
0 & 1/2 & 1/2 & 0 \\
0 & 1/2 & 1/2 & 0 \\
0 & 0 & 0 & 1
\end{bmatrix}
\begin{bmatrix}
x \\
y \\
z \\
w \\
\end{bmatrix}
=
\begin{bmatrix}
0 \\
0 \\
0 \\
0 \\
\end{bmatrix}
\Leftrightarrow
\begin{cases}
x=0\\
\frac{y}{2}+\frac{z}{2} = 0\\
\frac{y}{2}+\frac{z}{2} = 0\\
w=0
\end{cases}
\Leftrightarrow
\begin{cases}
y=-z\\
x=w=0
\end{cases}
\end{align*}
ou seja
\[
X =
\begin{bmatrix}
0 & -z \\
z & 0
\end{bmatrix}
=
z
\begin{bmatrix}
0 & -1 \\
1 & 0
\end{bmatrix}
.
\]

\textbf{Solução 2}: Para que $X$ seja um autovetor de $L$ associado ao autovalor $c \in \R$, deve ocorrer que $L(X) = cX$, que equivale ao seguinte:
\[
\frac{1}{2}
\begin{bmatrix}
0 & y-z \\
z-y & 0
\end{bmatrix}
=
c
\begin{bmatrix}
x & y \\
z & w
\end{bmatrix}
\Leftrightarrow
\begin{cases}
0 = cx \\
\frac{y-z}{2} = cy \\
\frac{z-y}{2} = cz \\
0 = cw
\end{cases}
\]
Resulta da primeira equação que $c = 0$ ou $x = 0$.
\begin{itemize}
\item Se $c = 0$ então $cy = 0 = cz$ e segue da segunda (ou da terceira) equação que $y=z$. Logo, os autovetores associados a $c = 0$ são as matrizes da forma
\[
X =
\begin{bmatrix}
x & z \\
z & w
\end{bmatrix}
=
x
\begin{bmatrix}
1 & 0 \\
0 & 0
\end{bmatrix}
+z
\begin{bmatrix}
0 & 1 \\
1 & 0
\end{bmatrix}
+w
\begin{bmatrix}
0 & 0 \\
0 & 1
\end{bmatrix}.
\]
\item Se $c \neq 0$ então $x = 0$ e (pela última equação) $w = 0$. Além disso, a segunda e a terceira equações equivalem a
\[
\begin{cases}
y-z = 2cy\\
z-y = 2cz
\end{cases}
\Leftrightarrow
\begin{cases}
y = z + 2cy\\
z = y + 2cz
\end{cases}
\Rightarrow
y = (y + 2cz) + 2cy
\Rightarrow
0 = 2c(z + y)
\Rightarrow
z=-y.
\]
Substituindo na segunda equação, segue que $(-y) = y + 2c(-y)$, ou seja $-2y=-2cy$, e portanto $c = 1$, pois $y$ não pode ser nulo (ou $z$ também seria). Assim, $X
= z
\begin{bmatrix}
0 & -1 \\
1 & 0
\end{bmatrix}$.

\end{itemize}

\item (1,0) A matriz $A$ é diagonalizável pois foi possível encontrar uma base de $M_{2 \times 2}$ formada por autovetores de $L$:
\[
B = \left\{
\begin{bmatrix}
1 & 0 \\
0 & 0
\end{bmatrix},
\begin{bmatrix}
0 & 1 \\
1 & 0
\end{bmatrix},
\begin{bmatrix}
0 & 0 \\
0 & 1
\end{bmatrix},
\begin{bmatrix}
0 & -1 \\
1 & 0
\end{bmatrix}
\right\}.
\]
Também é possível concluir que $A$ é diagonalizável observando que a multiplicidade algébrica e a multiplicidade geométrica de $0$ são iguais a $3$ e que a multiplicidade algébrica e a multiplicidade geométrica de $1$ são iguais a $1$.

\end{enumerate}

\Exercise[title={2,5}] Considere a matriz $A =
\begin{bmatrix}
 3 &  0 &  4\\
 6 & -1 &  6\\
-2 &  0 & -3
\end{bmatrix}$ e obtenha:
\begin{enumerate}
\item (0,5) O polinômio característico e os autovalores de $A$.
\item (1,0) Uma matriz $P$ que diagonaliza $A$.
\item (0,5) Uma matriz diagonal semelhante à matriz $A$.
\item (0,5) A matriz $A^{64}$, utilizando os dados anteriores.
\end{enumerate}
\Answer
\begin{enumerate}
\item O polinômio característico de $A$ é $p_A(x) = \det{(xI-A)}$, ou seja,
\begin{align*}
p_A(x)
& =
\begin{vmatrix}
 x-3 &  0  &  -4\\
  -6 & x+1 &  -6\\
   2 &  0  & x+3
\end{vmatrix}
= (x+1)
\begin{vmatrix}
x-3 &  -4\\
  2 & x+3
\end{vmatrix}
 = (x+1)[ (x-3)(x+3)+ 8 ]\\
&= (x+1)[ (x^2 - 9) + 8 ]
 = (x+1)(x^2-1)
 = (x+1)^2(x-1).
\end{align*}
Assim, os autovalores de $A$ são $-1$ (com multiplicidade algébrica igual a $2$) $1$ e (com multiplicidade $1$).
\item As colunas de $P$ são formadas com as coordenadas dos autovalores de $A$. Todo autovetor $X$ associado ao autovalor $-1$ deve ser solução do sistema linear
\begin{align*}
((-1)I - A)X = 0
\Leftrightarrow
\begin{bmatrix}
-4 & 0 & -4\\
-6 & 0 & -6\\
 2 & 0 &  2
\end{bmatrix}
\begin{bmatrix}
x\\
y\\
z
\end{bmatrix}
=
\begin{bmatrix}
0\\
0\\
0
\end{bmatrix}
\Leftrightarrow
\systeme{
-4x-4z = 0,
-6x-6z = 0,
 2x+2z = 0
}
\Leftrightarrow
x=-z.
\end{align*}
Assim, $X = \begin{bmatrix}
-z \\ y \\ z
\end{bmatrix}
=
y
\begin{bmatrix}
0 \\ 1 \\ 0
\end{bmatrix}
+z
\begin{bmatrix}
-1 \\ 0 \\ 1
\end{bmatrix}$.

Já um autovetor $X$ associado ao autovalor $1$ deve ser solução do sistema linear
\begin{align*}
(I - A)X = 0
\Leftrightarrow
\begin{bmatrix}
-2 & 0 & -4\\
-6 & 2 & -6\\
 2 & 0 &  4
\end{bmatrix}
\begin{bmatrix}
x\\
y\\
z
\end{bmatrix}
=
\begin{bmatrix}
0\\
0\\
0
\end{bmatrix}
\Leftrightarrow
\systeme{
-2x-4z = 0,
-6x+2y-6z = 0,
 2x+4z = 0
}
\Leftrightarrow
\begin{cases}
x = -2z \\
y = -3z.
\end{cases}
\end{align*}
Assim, $X = \begin{bmatrix}
-2z \\ -3z \\ z
\end{bmatrix}
=
z
\begin{bmatrix}
-2 \\ -3 \\ 1
\end{bmatrix}$. Logo,
$P = \begin{bmatrix}
0 & -1 & -2 \\
1 &  0 & -3 \\
0 &  1 &  1
\end{bmatrix}$
diagonaliza a matriz $A$.
\item Usando a matriz $P$ do item anterior, pode-se obter uma matriz diagonal $D = P^{-1} A P$, cujas entradas da diagonal são justamente os autovalores de $A$.

A inversa de $P$ é $P ^{-1} =
\begin{bmatrix}
-3 & 1 & -3 \\
 1 & 0 &  2 \\
-1 & 0 & -1
\end{bmatrix}$. Então
\[
D =
\begin{bmatrix}
-3 & 1 & -3 \\
 1 & 0 &  2 \\
-1 & 0 & -1
\end{bmatrix}
\begin{bmatrix}
 3 &  0 &  4\\
 6 & -1 &  6\\
-2 &  0 & -3
\end{bmatrix}
\begin{bmatrix}
0 & -1 & -2 \\
1 &  0 & -3 \\
0 &  1 &  1
\end{bmatrix}
=
\begin{bmatrix}
-1 &  0 & 0 \\
 0 & -1 & 0 \\
 0 &  0 & 1
\end{bmatrix}.
\]

\item Como $D = P^{-1} A P$, tem-se $A^{64} = P D^{64} P^{-1}$. Mas
\[
D^{64}
=
\begin{bmatrix}
(-1)^{64} &  0 & 0 \\
 0 & (-1)^{64} & 0 \\
 0 &  0 & 1^{64}
\end{bmatrix}
=
\begin{bmatrix}
1 & 0 & 0 \\
0 & 1 & 0 \\
0 & 0 & 1
\end{bmatrix}
= I,
\]
\end{enumerate}
então $A^{64} = P I P^{-1} = P P^{-1} = I$.

\Exercise[title={2,5}] Identifique e justifique as afirmações a seguir que forem verdadeiras. Para as demais, forneça um simples contraexemplo:
\begin{enumerate}
\item (1,3) Toda matriz diagonalizável tem que ser inversível.
\item (1,2) Em todo espaço com produto interno, sempre que $\norm{u} = 3$ e $\norm{v} = 1$, e os vetores $u$ e $v$ são ortogonais, pode-se concluir que $\norm{u+v} = \sqrt{10}$.

\end{enumerate}
\Answer
\begin{enumerate}
\item \textbf{Falsa}. Se $A$ é uma matriz diagonalizável então existe alguma matriz $P$ inversível tal que $D = P^{-1} A P$, ou equivalentemente $A = P D P^{-1}$. Sendo assim,
\[
\det{A}
= \det{P} \cdot \det{D} \cdot \det{P^{-1}}
= \det{P} \cdot \det{D} \cdot \frac{1}{\det{P}}
= \det{D},
\]
o que significa que para $A$ ser inversível é preciso que $D$ também seja.

Como existem matrizes diagonais que não são inversíveis (por exemplo, a matriz nula), também existem matrizes diagonalizáveis não inversíveis (um exemplo é a matriz nula).
\item (1,3) \textbf{Verdadeira}. Se $u$ e $v$ são ortogonais, então $\langle u, v \rangle = 0 = \langle v, u \rangle$. Consequentemente:
\begin{align*}
    \norm{u+v}
& = \sqrt{ \langle u+v, u+v \rangle }
  = \sqrt{ \langle u+v, u \rangle + \langle u+v, v \rangle }
  = \sqrt{ \langle u, u \rangle + \langle v, u \rangle
  + \langle u, v \rangle + \langle v, v \rangle } \\
& = \sqrt{ \langle u, u \rangle + 0 + 0 + \langle v, v \rangle }
  = \sqrt{ \norm{u}^2 + \norm{v}^2 }
  = \sqrt{3^2 + 1^2} = \sqrt{10}.
\end{align*}
\end{enumerate}


\Exercise[title={2,5}] Considere o espaço $\R^4$ com o produto interno usual e seja $W$ o subespaço vetorial gerado pela base $B = \{ v_1, v_2, v_3\}$, em que
$v_1 = (6, 0, 0, 0)$,
$v_2 = (3, 0, 4, 0)$ e
$v_3 = (0, 2, 2, 1)$. Obtenha:
\begin{enumerate}
\item (1,5) Uma base ortonormal para $W$.
\item (1,0) Uma base para o complemento ortogonal $W^\perp$ do subespaço $W$.
\end{enumerate}
\Answer
\begin{enumerate}
\item Observe que $v_2$ não é ortogonal aos vetores $v_1$ e $v_3$ pois $\langle v_1, v_2 \rangle = 18$ e $\langle v_1, v_2 \rangle = 8$. Para obter uma base ortogonal para o mesmo espaço, pode-se utilizar o processo de ortogonalização de Gram–Schmidt:
\begin{itemize}
\item $w_1
= v_1
= (6, 0, 0, 0)$
\item $w_2
= v_2 - \frac{\langle v_2, w_1 \rangle}{\norm{w_1}^2}w_1
= (3, 0, 4, 0) - \frac{18}{36}(6, 0, 0, 0)
= (0, 0, 4, 0)$
\item $w_2
= v_3 - \frac{\langle v_3, w_1 \rangle}{\norm{w_1}^2}w_1
      - \frac{\langle v_3, w_2 \rangle}{\norm{w_2}^2}w_2
= (0, 2, 2, 1) - \frac{0}{36}(6, 0, 0, 0)
               - \frac{8}{16}(0, 0, 4, 0)
= (0,2,0,1)$
\end{itemize}
O conjunto $C = \{ w_1, w_2, w_3 \} = \{
(6, 0, 0, 0),  (0, 0, 4, 0), (0, 2, 0, 1)
\}$ é uma base de $W$, mas como seus vetores não têm norma um, é preciso normaliza-los para obter a base ortonormal $D = \left\{ \frac{w_1}{\norm{w_1}}, \frac{w_2}{\norm{w_2}}, \frac{w_3}{\norm{w_3}} \right\} = \left\{
(1, 0, 0, 0),  (0, 0, 1, 0), (0, \frac{2}{\sqrt{5}}, 0, \frac{1}{\sqrt{5}})
\right\}$.


\item Todo vetor $u = (x,y,z,w) \in W^\perp$ é ortogonal aos elementos de $B$. Mas
\begin{align*}
\begin{cases}
\langle u, v_1\rangle = 0\\
\langle u, v_2\rangle = 0\\
\langle u, v_3\rangle = 0
\end{cases}
\Leftrightarrow
\begin{cases}
\langle (x,y,z,w), (6, 0, 0, 0)\rangle = 0\\
\langle (x,y,z,w), (3, 0, 4, 0)\rangle = 0\\
\langle (x,y,z,w), (0, 2, 2, 1)\rangle = 0
\end{cases}
\Leftrightarrow
\begin{cases}
6x = 0\\
3x+4z = 0\\
2y+2z+w = 0
\end{cases}
\Leftrightarrow
\begin{cases}
x = 0 \\
w = -2y\\
z = 0
\end{cases}
\end{align*}

Assim, $v = (0,y,0,-2y) = y(0,1,0,-2)$ e $B_1 = \{ (0,1,0,-2) \}$ é uma base de $W^\perp$.
\end{enumerate}


\Exercise[title={2,5}] Seja
$B = \{ (2,2,0), (-1, 3, 0), (0, 0, 5) \}$
uma base de $\R^3$ e considere o produto interno definido por $\langle (a, b, c), (x,y,z) \rangle = 3ax + by + cz$.
\begin{enumerate}
\item (1,0) Verifique que, em relação ao produto interno acima, a base $B$ é ortogonal.
\item (1,0) Use o produto interno acima para escrever o vetor $v = (-10, 6, -5)$ como combinação linear dos vetores de $B$.
\item (0,5) Obtenha uma base ortonormal de $\R^3$ a partir de $B$ (utilize o produto interno dado).
\end{enumerate}
\Answer
\begin{enumerate}
\item Sejam $v_1 = (2,2,0)$, $v_2 = (-1, 3, 0)$ e $v_3 =(0, 0, 5)$. Para concluir que a base $B = \{ v_1, v_2, v_3\}$ é ortogonal, basta observar que $\langle v_i, v_j \rangle = 0$ sempre que $i \neq j$:
\begin{itemize}
\item $\langle v_1, v_2 \rangle = \langle (2,2,0), (-1, 3, 0) \rangle = 3 \cdot 2 \cdot (-1) + 2 \cdot 3 + 0 = 0$.
\item $\langle v_2, v_3 \rangle = \langle (-1, 3, 0), (0, 0, 5) \rangle = 3 \cdot 0 + 0 + 0 = 0$.
\item $\langle v_1, v_3 \rangle = \langle (2,2,0), (0, 0, 5) \rangle = 3 \cdot 0 + 0 + 0 = 0$.
\end{itemize}

\item Para qualquer vetor $w \in \R^3$ tem-se $\displaystyle w = \frac{ \langle w, v_1\rangle }{ \norm{v_1}^2 } v_1
   +\frac{ \langle w, v_2\rangle }{ \norm{v_2}^2 } v_2
   +\frac{ \langle w, v_3\rangle }{ \norm{v_3}^2 } v_3
$
Mas
\begin{itemize}
\item $\langle (-10, 6, -5), (2,2,0)\rangle = -60 + 12 + 0 = -48$
\item $\langle (-10, 6, -5), (-1, 3, 0)\rangle = 30 + 18 + 0 = 48$
\item $\langle (-10, 6, -5), (0, 0, 5)\rangle = 0 + 0 - 25 = -25$
\item $\norm{ (2,2,0) }^2 = 12 + 4 + 0 = 16$
\item $\norm{ (-1,3,0) }^2 = 3 + 9 + 0 = 12$
\item $\norm{ (0,0,5) }^2 = 0 + 0 + 25 = 25$
\end{itemize}

Então
\[
(-10, 6, -5)
= \frac{ -48 }{ 16 } v_1
 +\frac{ 48 }{ 12 } v_2
 +\frac{ -25 }{ 25 } v_3
= -3 (2,2,0)
 + 4 (-1,3,0)
 - 1 (0,0,5).
 \]
\item Como a base é ortogonal, basta normalizar cada um dos vetores (isto é, dividi-los pelas respectivas normas) e o resultado será uma base ortonormal:
\begin{align*}
C
&= \left\{
\frac{v_1}{\norm{v_1}},
\frac{v_2}{\norm{v_2}},
\frac{v_3}{\norm{v_3}}
\right\}
= \left\{
\frac{(2,2,0)}{\sqrt{16}},
\frac{(-1,3,0)}{\sqrt{12}},
\frac{(0,0,5)}{\sqrt{25}}
\right\}\\
&= \left\{
\left(\frac{1}{2},\frac{1}{2},0\right),
\left(\frac{-1}{\sqrt{12}},\frac{3}{\sqrt{12}},0\right)
\left(0,0,1\right)
\right\}.
\end{align*}

\end{enumerate}
\end{ExerciseList}

\begin{center}
BOA PROVA!
\end{center}

\newpage
\restoregeometry
\section*{Respostas}
\shipoutAnswer
\end{document}
