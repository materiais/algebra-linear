\documentclass[12pt,a4paper]{article}
\usepackage{cmap} % Makes the PDF copiable. See http://tex.stackexchange.com/a/64198/25761
\usepackage[T1]{fontenc}
\usepackage[brazil]{babel}
\usepackage[utf8]{inputenc}
\usepackage{amsmath}
\usepackage{amsfonts}
\usepackage{amssymb}
\usepackage{amsthm}
\usepackage{textcomp} % \degree
\usepackage{gensymb} % \degree
\usepackage[usenames,svgnames,dvipsnames]{xcolor}
\usepackage{hyperref}
\usepackage{multicol}
\usepackage{graphicx}
\usepackage[margin=2cm]{geometry}
\usepackage{systeme}

\hypersetup{
    colorlinks = true,
    allcolors = {blue}
}

% TODO: Consider using exsheets
% http://linorg.usp.br/CTAN/macros/latex/contrib/exsheets/exsheets_en.pdf
%
% http://ctan.org/tex-archive/macros/latex/contrib/exercise/
% Options: answerdelayed,lastexercise,noanswer
\usepackage[answerdelayed,lastexercise]{exercise}

\addto\captionsbrazil{%
\def\listexercisename{Lista de exerc\'icios}%
\def\ExerciseName{Exerc\'icio}%
\def\AnswerName{Solu\c{c}\~ao do exerc\'icio}%
\def\ExerciseListName{Ex.}%
\def\AnswerListName{Solu\c{c}\~ao}%
\def\ExePartName{Parte}%
\def\ArticleOf{de\ }%
}

\renewcommand{\ExerciseHeaderTitle}{(\ExerciseTitle)\ }
\renewcommand{\ExerciseListHeader}{%\ExerciseHeaderDifficulty%
\textbf{%\ExerciseListName\
\ExerciseHeaderNB.\ %
%\ --- \
\ExerciseHeaderTitle}%
%\ExerciseHeaderOrigin
\ignorespaces}
\renewcommand{\AnswerListHeader}{\textbf{\ExerciseHeaderNB.\ (\AnswerListName)\ }}

\newcommand*\im[1]{\operatorname{Im}\left(#1\right)}
\newcommand*\ger[1]{\operatorname{ger}\left\{#1\right\}}
\newcommand*\R{\mathbb{R}}

% Loop Space / CC BY-SA-3.0 / https://tex.stackexchange.com/a/2238/25761
\newenvironment{amatrix}[1]{%
  \left[\begin{array}{@{}*{#1}{c}|c@{}}
}{%
  \end{array}\right]
}

% Loop Space / CC BY-SA-3.0 / https://tex.stackexchange.com/a/3164/25761
%--------grstep
% For denoting a Gauss' reduction step.
% Use as: \grstep{\rho_1+\rho_3} or \grstep[2\rho_5 \\ 3\rho_6]{\rho_1+\rho_3}
\newcommand{\grstep}[2][\relax]{%
   \ensuremath{\mathrel{
       {\mathop{\longrightarrow}\limits^{#2\mathstrut}_{
                                     \begin{subarray}{l} #1 \end{subarray}}}}}}
\newcommand{\swap}{\leftrightarrow}

\renewcommand{\theenumi}{\alph{enumi}}
\renewcommand\labelenumi{(\theenumi) }

\newcommand*\tipo{Prova III}
\newcommand*\turma{CIV122-02U}
\newcommand*\disciplina{ALI0001}
\newcommand*\eu{Helder G. G. de Lima}
\newcommand*\data{25/05/2016}

\author{\eu}
\title{\tipo - \disciplina}
\date{\data}

\begin{document}
\thispagestyle{empty}
\newgeometry{margin=2cm,bottom=0.5cm}
\begin{center}
\includegraphics[width=9.0cm]{marca} \\
\textbf{\tipo\ (\disciplina / \turma)} \\
Prof. \eu\footnote{
Este é um material de acesso livre distribuído sob os termos da licença \href{https://creativecommons.org/licenses/by-sa/4.0/deed.pt_BR}{Creative Commons Atribuição-CompartilhaIgual 4.0 Internacional}}
\end{center}

\noindent Nome do(a) aluno(a): \underline{\hspace{9,7cm}} Data: \underline{\data}

%\section*{Instruções}
\begin{center}\fbox{
\begin{minipage}{14cm}

{\footnotesize
\begin{itemize}
\renewcommand{\theenumi}{\Roman{enumi}}
\item Identifique-se em todas as folhas.
\item Mantenha o celular e os demais equipamentos eletrônicos desligados durante a prova.
\item Anule \textsc{\textbf{uma}} das 5 questões (apenas 4 serão corrigidas): \framebox(30,10){}
\end{itemize}
}

\end{minipage}
}
\end{center}

%\section*{Questões}
\begin{ExerciseList}

\Exercise[title={2,5}] Considere as bases $B = \left\{ \begin{bmatrix}
-1 & 0\\ 0 & 0
\end{bmatrix}, \begin{bmatrix}
 2 & 4\\ 0 & 0
\end{bmatrix}, \begin{bmatrix}
-1 & 2\\ 0 & 2
\end{bmatrix}\right\}$ e $C = \left\{ \begin{bmatrix}
1 & 0\\ 0 & 0
\end{bmatrix}, \begin{bmatrix}
1 & 1\\ 0 & 0
\end{bmatrix}, \begin{bmatrix}
1 & 1\\ 0 & 1
\end{bmatrix}\right\}$ do espaço vetorial das matrizes triangulares superiores $2 \times 2$, isto é, $V = \left\{ \begin{bmatrix}
a & b\\ 0 & c
\end{bmatrix} \mid a,b,c \in \R \right\}$.
\begin{enumerate}
\item (2,0) Determine as matrizes de mudança de base de $B$ para $C$ e de $C$ para $B$.
\item (0,5) Se $v \in V$ é o vetor cujas coordenadas na base $B$ são dadas por $[v]_B =
\begin{bmatrix}
3 \\
2 \\
1
\end{bmatrix}$, utilize o item anterior para encontrar as coordenadas de $v$ em relação à base $C$, ou seja, $[v]_C$.
\end{enumerate}
\Answer
\begin{enumerate}
\item
Para encontrar a matriz de mudança da base $B$ para a base $C$, é preciso escrever cada vetor de $B$ como combinação linear dos vetores de $C$. Assim, como
\[
a
\begin{bmatrix}
1 & 0\\ 0 & 0
\end{bmatrix}
+ b
\begin{bmatrix}
1 & 1\\ 0 & 0
\end{bmatrix}
+ c
\begin{bmatrix}
1 & 1\\ 0 & 1
\end{bmatrix}
=
\begin{bmatrix}
a+b+c & b+c\\ 0 & c
\end{bmatrix},\forall a,b,c \in \R,
\]
tem-se
\begin{align*}
\begin{bmatrix}
-1 & 0\\ 0 & 0
\end{bmatrix}
& =
\begin{bmatrix}
a+b+c & b+c\\ 0 & c
\end{bmatrix}
\Leftrightarrow
\systeme{
a+b+c=-1,
b+c=0,
c=0
}
\Leftrightarrow
\begin{cases}
a=-1\\
b=0\\
c=0
\end{cases} \\
%--------------------------
\begin{bmatrix}
2 & 4\\ 0 & 0
\end{bmatrix}
& =
\begin{bmatrix}
a+b+c & b+c\\ 0 & c
\end{bmatrix}
\Leftrightarrow
\systeme{
a+b+c=2,
b+c=4,
c=0
}
\Leftrightarrow
\begin{cases}
a=-2\\
b=4\\
c=0
\end{cases} \\
%--------------------------
\begin{bmatrix}
-1 & 2\\ 0 & 2
\end{bmatrix}
& =
\begin{bmatrix}
a+b+c & b+c\\ 0 & c
\end{bmatrix}
\Leftrightarrow
\systeme{
a+b+c=-1,
b+c=2,
c=2
}
\Leftrightarrow
\begin{cases}
a=-3\\
b=0\\
c=2
\end{cases}
\end{align*}
e portanto $[I]^B_C =
\begin{bmatrix}
-1 & -2 & -3 \\
 0 &  4 &  0 \\
 0 &  0 &  2
\end{bmatrix}$ é a matriz de transição da base $B$ para a base $C$.
\item Como, $[I]^C_B = \left([I]^B_C\right)^{-1}$, basta realizar um escalonamento para obter a matriz de mudança de base de $C$ para $B$ (outra alternativa seria fazer como no item anterior, escrevendo cada vetor de $C$ como combinação linear dos vetores de $B$, e resolver os sistemas lineares correspondentes). Tem-se:
\begin{align*}
\begin{bmatrix}
-1 & -2 & -3 & 1 & 0 & 0\\
 0 &  4 &  0 & 0 & 1 & 0 \\
 0 &  0 &  2 & 0 & 0 & 1
\end{bmatrix}
& \grstep[L_3/2]{ -L_1 }
\begin{bmatrix}
1 & 2 & 3 & -1 & 0 & 0\\
0 & 4 & 0 &  0 & 1 & 0 \\
0 & 0 & 1 &  0 & 0 & 1/2
\end{bmatrix}
\grstep[L_2/4]{L_1-3L_3}
\begin{bmatrix}
1 & 2 & 0 & -1 & 0 & -3/2 \\
0 & 1 & 0 &  0 & 1/4 & 0 \\
0 & 0 & 1 &  0 & 0 & 1/2
\end{bmatrix} \\
& \grstep{ L_1-2L_2 }
\begin{bmatrix}
1 & 0 & 0 & -1 & -1/2 & -3/2 \\
0 & 1 & 0 &  0 & 1/4 & 0 \\
0 & 0 & 1 &  0 & 0 & 1/2
\end{bmatrix}
\end{align*}
Portanto, $[I]^C_B =
\begin{bmatrix}
-1 & -1/2 & -3/2 \\
 0 &  1/4 &  0 \\
 0 &  0 &  1/2
\end{bmatrix}$.

\item As coordenadas de $v$ em relação à base $C$ são o resultado da multiplicação da matriz de mudança da base $B$ para a base $C$ pelo vetor de coordenadas de $v$ na base $B$, isto é,
\[
[v]_C
=
[I]^B_C [v]_B
=
\begin{bmatrix}
-1 & -2 & -3 \\
 0 &  4 &  0 \\
 0 &  0 &  2
\end{bmatrix}
\cdot
\begin{bmatrix}
3 \\
2 \\
1
\end{bmatrix}
=
\begin{bmatrix}
-10 \\
  8 \\
  2
\end{bmatrix}.
\]
\end{enumerate}

\Exercise[title={2,5}] Prove (se for verdadeira) ou dê um contraexemplo (se for falsa):
\begin{enumerate}
\item (1,3) Se $T: U \to V$ é uma transformação linear e $U = V$ então $T$ é um isomorfismo.
\item (1,2) Se $T: \R^3 \to M_{2 \times 2}$ é uma transformação linear injetora então $T$ também é sobrejetora.
\end{enumerate}
\Answer
\begin{enumerate}
\item \textbf{Falsa}. Não é suficiente (nem necessário) que o domínio de uma transformação linear seja igual ao seu contradomínio para que a transformação seja um isomorfismo. Por exemplo, se $T: \R^2 \to \R^2$ é a transformação linear nula, definida por $T(x,y) = (0,0)$, então o domínio e o contradomínio são iguais, mas $N(T) = \R^2 \neq \{(0,0)\}$, o que significa que $T$ não é injetora. Logo $T$ não é um isomorfismo, apesar de seu domínio e seu contradomínio coincidirem.

\item \textbf{Falsa}. Se $T: \R^3 \to M_{2 \times 2}$ é injetora, então $\dim{N(T)} = 0$. Consequentemente, pelo teorema do núcleo e da imagem, tem-se
\[
\dim{\im{T}} = \dim{\R^3} - \dim{N(T)} = 3 - 0 = 3 < 4 = \dim{M_{2 \times 2}}.
\]
Então $\im{T} \neq M_{2 \times 2}$ e $T$ não é sobrejetora.
\end{enumerate}


\Exercise[title={2,5}] Obtenha a fórmula explícita da transformação linear $T: \R^3 \to \R^3$, tal que
$T(3,2,-1) = (7,0,0)$,
$T(1,1,0) = (0,0,0)$,
$T(-1,0,0) = (0,3,0)$ e verifique se $(-3,-1,1) \in N(T)$.

\Answer Dado $(x,y,z) \in \R^3$, sejam $a,b,c$ tais que
\[
(x,y,z) = a(3,2,-1) + b(1,1,0) + c(-1,0,0) = (3a+b-c,2a+b,-a)
\]
então
\[
\systeme{
3a+b-c = x,
2a+b = y,
-a = z
}
\Leftrightarrow
\begin{cases}
a=-z\\
b=y+2z\\
c=-x+y-z
\end{cases}
\]
Logo,
\begin{align*}
T(x,y,z)
& = T( (-z)(3,2,-1) + (y+2z)(1,1,0) + (-x+y-z)(-1,0,0) )\\
& = (-z)T(3,2,-1) + (y+2z)T(1,1,0) + (-x+y-z)T(-1,0,0)\\
& = (-z)(7,0,0) + (y+2z)(0,0,0) + (-x+y-z)(0,3,0)\\
& = (-7z,-3x+3y-3z,0).
\end{align*}
Em particular, tem-se
\[
T(-3,-1,1)
= (-7 \cdot 1,-3 \cdot (-3)+3 \cdot(-1)-3 \cdot 1,0)
= (-7,3,0)
\neq (0,0,0).
\] Portanto $(-3,-1,1) \not \in N(T)$.


\Exercise[title={2,5}] Considere a função $L: \R^3 \to P_2$ definida por $L(a,b,c) = (2c+b)x + (3a)x^2$.
\begin{enumerate}
\item (1,0) Mostre que $L$ é uma transformação linear.
\item (1,5) Calcule as dimensões do núcleo e da imagem de $L$, isto é, $\dim(N(L))$ e $\dim(\im{L})$.
\end{enumerate}
\Answer
\begin{enumerate}
\item Dados $u = (a_1,b_1,c_1)$ e $v = (a_2,b_2,c_2)$, e $k \in \R$, tem-se:
\begin{align*}
L(u+v)
& = L(a_1+a_2, b_1+b_2, c_1+c_2)
  = (2(c_1+c_2)+b_1+b_2)x + 3(a_1+a_2)x^2\\
& = (2c_1+2c_2+b_1+b_2)x + (3a_1+3a_2)x^2\\
& = \left[(2c_1+b_1)x + 3a_1x^2\right] + \left[(2c_2+b_2)x + 3a_2x^2\right]\\
& = L(a_1, b_1, c_1) + L(a_2, b_2, c_2)
  = L(u) + L(v)\\
L(ku)
& = L(ka_1, kb_1, kc_1)
  = (2(kc_1)+kb_1)x + 3(ka_1)x^2\\
& = k(2c_1+b_1)x + k(3a_1)x^2
  = k\left[(2c_1+b_1)x + 3a_1x^2\right]
  = kL(a_1, b_1, c_1)
  = kL(u)
\end{align*}
Portanto, $L$ é uma transformação linear.

\item \textbf{Solução 1:} Um vetor $v = (a,b,c)$ pertence ao núcleo de $L$ se, e somente se, $L(a,b,c) = (2c+b)x + (3a)x^2 = 0 + 0x + 0x^2$, o que equivale às condições
\[
\begin{cases}
2c+b=0\\
3a=0
\end{cases}
\Leftrightarrow
\begin{cases}
b=-2c\\
a=0
\end{cases}
\Leftrightarrow
v = (a,b,c) = (0,-2c,c) = c(0,-2,1).
\]
Isto significa que $N(L) = \ger{(0,-2,1)}$ e como este vetor é diferente de zero, ele é L.I. e forma uma base de $N(L)$. Consequentemente, $\dim{N(L)} = 1$. Pelo teorema do núcleo e da imagem, tem-se
\[
\dim{\im{L}} = \dim{\R^3} - \dim{N(L)} = 3 - 1 = 2.
\]

\textbf{Solução 2:} Observe que $\im{L} = \ger{x,x^2}$, e como estes vetores são linearmente independentes, eles são uma base da imagem de $L$. Logo, $\dim{\im{L}} = 2$ e $\dim{N(L)} = \dim{\R^3} - \dim{\im{L}} = 3 - 2 = 1$.

\end{enumerate}

\Exercise[title={2,5}] Seja $T(a,b,c) = \begin{bmatrix}
  a & c-a\\
c-a & b-c
\end{bmatrix}$ a transformação linear de $\R^3$ no espaço $S_{2\times 2}$ das matrizes simétricas $2 \times 2$.
\begin{enumerate}
\item (1,0) Mostre que $T$ é um isomorfismo.
\item (1,5) Obtenha uma fórmula explícita para o isomorfismo inverso $T^{-1}: S_{2 \times 2} \to \R^3$.
\end{enumerate}

\Answer \begin{enumerate}
\item Uma transformação linear é um isomorfismo se for injetora e sobrejetora. Como $B = \{ (1,0,0), (0,1,0), (0,0,1)\}$ é uma base de $\R^3$ e $C = \left\{
\begin{bmatrix}
1 & 0\\
0 & 0
\end{bmatrix},
\begin{bmatrix}
0 & 1\\
1 & 0
\end{bmatrix},
\begin{bmatrix}
0 & 0\\
0 & 1
\end{bmatrix}
\right\}$ é uma base de $S_{2 \times 2}$, segue que $\dim{\R^3} = 3 = \dim{ S_{2 \times 2} }$, e $T$ será injetora se, e somente se, for sobrejetora. Então basta mostrar que que $N(T) = \{(0,0,0)\})$. Tem-se:
\[
(a,b,c) \in N(T)
\Leftrightarrow
T(a,b,c)
=
 \begin{bmatrix}
  a & c-a\\
c-a & b-c
\end{bmatrix}
=
\begin{bmatrix}
0 & 0\\
0 & 0
\end{bmatrix}
\Leftrightarrow
\systeme{
a=0,
c-a=0,
b-c=0
}
\Leftrightarrow
\begin{cases}
a=0\\
b=0\\
c=0
\end{cases}
\]
Então $T$ é, de fato, um isomorfismo.
\item \textbf{Solução 1:} Pela definição de função inversa, tem-se
\begin{align*}
(a,b,c)
= T^{-1}\left(
\begin{bmatrix}
x & y\\
y & z
\end{bmatrix}\right)
& \Leftrightarrow
T(a,b,c)
=
\begin{bmatrix}
x & y\\
y & z
\end{bmatrix}
\Leftrightarrow
\begin{bmatrix}
  a & c-a\\
c-a & b-c
\end{bmatrix}
=
\begin{bmatrix}
x & y\\
y & z
\end{bmatrix}\\
& \Leftrightarrow
\systeme{
a=x,
c-a=y,
b-c=z
}
\Leftrightarrow
\begin{cases}
a=x\\
b=x+y+z\\
c=x+y
\end{cases}
\end{align*}
Portanto, $T^{-1}\left(
\begin{bmatrix}
x & y\\
y & z
\end{bmatrix}\right) = (x,x+y+z,x+y)$.

\textbf{Solução 2:} Obter a matriz de $T$ em relação às bases $B$ e $C$, calcular a matriz inversa, que será a matriz de $T^{-1}$, e então utilizá-la para obter a fórmula desejada. Aplicando $T$ aos vetores da base $B$, resulta que $
[T]^B_C = \begin{bmatrix}
 1 & 0 & 0\\
-1 & 0 & 1\\
 0 & 1 & -1
\end{bmatrix}$ e então a matriz da transformação inversa $[T^{-1}]^C_B = ([T]^B_C)^{-1}$ pode ser obtida por escalonamento:
\begin{align*}
\begin{bmatrix}
 1 & 0 &  0 & 1 & 0 & 0\\
-1 & 0 &  1 & 0 & 1 & 0\\
 0 & 1 & -1 & 0 & 0 & 1
\end{bmatrix}
& \grstep{ L_2 \swap L_3 }
\begin{bmatrix}
 1 & 0 &  0 & 1 & 0 & 0\\
 0 & 1 & -1 & 0 & 0 & 1\\
-1 & 0 &  1 & 0 & 1 & 0
\end{bmatrix}
\grstep{ L_3+L_1 }
\begin{bmatrix}
1 & 0 &  0 & 1 & 0 & 0\\
0 & 1 & -1 & 0 & 0 & 1\\
0 & 0 &  1 & 1 & 1 & 0
\end{bmatrix} \\
& \grstep{ L_2+L_1 }
\begin{bmatrix}
1 & 0 & 0 & 1 & 0 & 0\\
0 & 1 & 0 & 1 & 1 & 1\\
0 & 0 & 1 & 1 & 1 & 0
\end{bmatrix}
\Rightarrow
[T^{-1}]^C_B = \begin{bmatrix}
1 & 0 & 0\\
1 & 1 & 1\\
1 & 1 & 0
\end{bmatrix}.
\end{align*}
Assim, sendo $u=
\begin{bmatrix}
x & y\\
y & z
\end{bmatrix}$ e $[T^{-1}(u)]_B = [T^{-1}]^C_B \cdot [u]_C = \begin{bmatrix}
1 & 0 & 0\\
1 & 1 & 1\\
1 & 1 & 0
\end{bmatrix}
\cdot
\begin{bmatrix}
x\\
y\\
z
\end{bmatrix}
=
\begin{bmatrix}
x\\
x+y+z\\
x+y
\end{bmatrix}$, conclui-se que $T^{-1}\left(
\begin{bmatrix}
x & y\\
y & z
\end{bmatrix}\right) = (x,x+y+z,x+y)$.
\end{enumerate}
\end{ExerciseList}


\begin{center}
BOA PROVA!
\end{center}

\newpage
\restoregeometry
\section*{Respostas}
\shipoutAnswer
\end{document}
