\documentclass[12pt,a4paper]{article}
\usepackage{cmap} % Makes the PDF copiable. See http://tex.stackexchange.com/a/64198/25761
\usepackage[T1]{fontenc}
\usepackage[brazil]{babel}
\usepackage[utf8]{inputenc}
\usepackage{amsmath}
\usepackage{amsfonts}
\usepackage{amssymb}
\usepackage{amsthm}
\usepackage{textcomp} % \degree
\usepackage{gensymb} % \degree
\usepackage[usenames,svgnames,dvipsnames]{xcolor}
\usepackage{hyperref}
\usepackage{multicol}
\usepackage{graphicx}
\usepackage[margin=2cm]{geometry}
\usepackage{systeme}

\hypersetup{
    colorlinks = true,
    allcolors = {blue}
}

% TODO: Consider using exsheets
% http://linorg.usp.br/CTAN/macros/latex/contrib/exsheets/exsheets_en.pdf
%
% http://ctan.org/tex-archive/macros/latex/contrib/exercise/
% Options: answerdelayed,lastexercise,noanswer
\usepackage[answerdelayed,lastexercise]{exercise}

\addto\captionsbrazil{%
\def\listexercisename{Lista de exerc\'icios}%
\def\ExerciseName{Exerc\'icio}%
\def\AnswerName{Solu\c{c}\~ao do exerc\'icio}%
\def\ExerciseListName{Ex.}%
\def\AnswerListName{Solu\c{c}\~ao}%
\def\ExePartName{Parte}%
\def\ArticleOf{de\ }%
}

\renewcommand{\ExerciseHeaderTitle}{(\ExerciseTitle)\ }
\renewcommand{\ExerciseListHeader}{%\ExerciseHeaderDifficulty%
\textbf{%\ExerciseListName\
\ExerciseHeaderNB.\ %
%\ --- \
\ExerciseHeaderTitle}%
%\ExerciseHeaderOrigin
\ignorespaces}
\renewcommand{\AnswerListHeader}{\textbf{\ExerciseHeaderNB.\ (\AnswerListName)\ }}

\newtheorem*{note}{Observação}
\newcommand*\R{\mathbb{R}}

% Loop Space / CC BY-SA-3.0 / https://tex.stackexchange.com/a/2238/25761
\newenvironment{amatrix}[1]{%
  \left[\begin{array}{@{}*{#1}{c}|c@{}}
}{%
  \end{array}\right]
}

% Loop Space / CC BY-SA-3.0 / https://tex.stackexchange.com/a/3164/25761
%--------grstep
% For denoting a Gauss' reduction step.
% Use as: \grstep{\rho_1+\rho_3} or \grstep[2\rho_5 \\ 3\rho_6]{\rho_1+\rho_3}
\newcommand{\grstep}[2][\relax]{%
   \ensuremath{\mathrel{
       {\mathop{\longrightarrow}\limits^{#2\mathstrut}_{
                                     \begin{subarray}{l} #1 \end{subarray}}}}}}
\newcommand{\swap}{\leftrightarrow}

\renewcommand{\theenumi}{\alph{enumi}}
\renewcommand\labelenumi{(\theenumi) }

\newcommand*\tipo{Prova I}
\newcommand*\turma{MAT102-03U/LEF102-02U}
\newcommand*\disciplina{ALI0001/ALG2001}
\newcommand*\eu{Helder G. G. de Lima}
\newcommand*\data{22/03/2016}

\author{\eu}
\title{\tipo - \disciplina}
\date{\data}

\begin{document}
\thispagestyle{empty}
\newgeometry{margin=2cm,bottom=0.5cm}
\begin{center}
\includegraphics[width=9.0cm]{marca} \\
\textbf{\tipo\ (\disciplina / \turma)} \\
Prof. \eu\footnote{
Este é um material de acesso livre distribuído sob os termos da licença \href{https://creativecommons.org/licenses/by-sa/4.0/deed.pt_BR}{Creative Commons Atribuição-CompartilhaIgual 4.0 Internacional}}
\end{center}

\noindent Nome do(a) aluno(a): \underline{\hspace{9,7cm}} Data: \underline{\data}

%\section*{Instruções}
\begin{center}\fbox{
\begin{minipage}{14cm}

{\footnotesize
\begin{itemize}
\renewcommand{\theenumi}{\Roman{enumi}}
\item Identifique-se em todas as folhas.
\item Mantenha o celular e os demais equipamentos eletrônicos desligados durante a prova.
\item Anule \textsc{\textbf{uma}} das 6 questões (apenas 5 serão corrigidas): \framebox(30,10){}
\end{itemize}
}

\end{minipage}
}
\end{center}

%\section*{Questões}
\begin{ExerciseList}
\Exercise[title={1,8}] Dar exemplos de matrizes quadradas $A \neq B$ tais que:
\begin{multicols}{2}
\begin{enumerate}
\item $A$, $B$ e $AB$ são matrizes simétricas.
\item $A$, $B$ e $AB$ são matrizes antissimétricas.
\item $A$ e $B$ são simétricas mas $AB$ não.
\item $A$ e $B$ são antissimétricas mas $AB$ não.
\end{enumerate}
\end{multicols}
\Answer
\begin{enumerate}
\item Se $A = \begin{bmatrix}
1 & 0 \\
0 & 2
\end{bmatrix}$ e $B = \begin{bmatrix}
3 & 0 \\
0 & 4
\end{bmatrix}$ então $AB
= \begin{bmatrix}
1 & 0 \\
0 & 2
\end{bmatrix}
\cdot
\begin{bmatrix}
3 & 0 \\
0 & 4
\end{bmatrix}
=
\begin{bmatrix}
3 & 0 \\
0 & 8
\end{bmatrix}$. Assim, o produto é uma matriz simétrica, pois $(AB)^T
=\begin{bmatrix}
3 & 0 \\
0 & 8
\end{bmatrix}
= AB$.

\textbf{Outra alternativa}: se $A = I$ e $B = 0$, então $AB = 0 = 0^T = (AB)^T$.

\item Se $A = \begin{bmatrix}
0 & 1 \\
-1 & 0
\end{bmatrix}$ e $B = \begin{bmatrix}
0 & 0 \\
0 & 0
\end{bmatrix}$ então $AB
= \begin{bmatrix}
0 & 0 \\
0 & 0
\end{bmatrix}
= -(AB)^T$. Assim, o produto é uma matriz antissimétrica, pois $(AB)^T = -AB$.

\textbf{Outra alternativa}: se $A = 0$ e $B = \begin{bmatrix}
0 & 1 \\
-1 & 0
\end{bmatrix}$ então $(AB)^T = 0^T = 0 = -0 = -AB$.

\item Se $A = \begin{bmatrix}
0 & 1 \\
1 & 0
\end{bmatrix}$ e $B = \begin{bmatrix}
1 & 2 \\
2 & 3
\end{bmatrix}$ então $AB
= \begin{bmatrix}
0 & 1 \\
1 & 0
\end{bmatrix}
\cdot
\begin{bmatrix}
1 & 2 \\
2 & 3
\end{bmatrix}
=
\begin{bmatrix}
2 & 3 \\
1 & 2
\end{bmatrix}$. Neste caso, o produto não é uma matriz simétrica, já que $(AB)^T
=\begin{bmatrix}
2 & 1 \\
3 & 2
\end{bmatrix}
\neq AB$.
\item Se $A = \begin{bmatrix}
0 & 1 \\
-1 & 0
\end{bmatrix}$ e $B = \begin{bmatrix}
0 & 2 \\
-2 & 0
\end{bmatrix}$ então $AB
= \begin{bmatrix}
-2 & 0 \\
0 & -2
\end{bmatrix}
\neq \begin{bmatrix}
2 & 0 \\
0 & 2
\end{bmatrix}
= -(AB)^T$. Assim, o produto não é uma matriz antissimétrica, pois $(AB)^T \neq -AB$.
\end{enumerate}


\Exercise[title={1,8}] Denote por $0$ a matriz nula $3 \times 3$ e por $I$ a matriz identidade $3 \times 3$.
\begin{enumerate}
\item Dê um exemplo de uma matriz $A \neq 0$, tal que $A^3 = 0$.
\item Verifique que $(I-A)$ é a inversa de $(I + A + A^2)$, sendo $A$ a matriz obtida no item anterior
\end{enumerate}

\Answer
\begin{enumerate}
\item Se $A = \begin{bmatrix}
0 &0 &4\\
0 &0 &0\\
0 &0 &0
\end{bmatrix}$ então
\[
A^3 =\left(
\begin{bmatrix}
0 &0 &4\\
0 &0 &0\\
0 &0 &0
\end{bmatrix}
\cdot
\begin{bmatrix}
0 &0 &4\\
0 &0 &0\\
0 &0 &0
\end{bmatrix}\right)
\cdot
\begin{bmatrix}
0 &0 &4\\
0 &0 &0\\
0 &0 &0
\end{bmatrix}
=
\begin{bmatrix}
0 &0 &0\\
0 &0 &0\\
0 &0 &0
\end{bmatrix}
\cdot
\begin{bmatrix}
0 &0 &4\\
0 &0 &0\\
0 &0 &0
\end{bmatrix}
=
\begin{bmatrix}
0 &0 &0\\
0 &0 &0\\
0 &0 &0
\end{bmatrix}.
\]

\textbf{(outras soluções)} há infinitas matrizes não nulas que elevadas ao cubo resultam em $0$:
\[
\begin{bmatrix}
0 & 0 & 0\\
0 & 1 & 1\\
0 &-1 &-1
\end{bmatrix},
\begin{bmatrix}
0 & 0 & 0\\
1 & 0 & 1\\
0 & 0 & 0
\end{bmatrix},
\begin{bmatrix}
0 & 1 & 0\\
6 & 0 & -2\\
0 & 3 & 0
\end{bmatrix},
\ldots
\]

\item \textbf{(solução 1)} Qualquer matriz $A$ escolhida para o item anterior satisfaz $A^3=0$, então:
\[
(I - A)(I + A + A^2)
= I(I + A + A^2)
 -A(I + A + A^2)
= I + A + A^2
 -A - A^2 - A^3
= I.
\]
Logo, $(I - A)$ é a inversa de $(I + A + A^2)$, ou seja, $(I + A + A^2)^{-1} = I - A$.

\textbf{(solução 2)} Outra opção é calcular explicitamente a inversa de
\[
I + A + A^2 =
\begin{bmatrix}
1 &0 &4\\
0 &1 &0\\
0 &0 &1\\
\end{bmatrix},
\]
por Gauss-Jordan (se fez uma boa escolha para $A$, não serão necessários muitos passos):
\[
[ I + A + A^2 | I] =
\begin{bmatrix}
1 &0 &4& 1& 0& 0\\
0 &1 & 0& 0& 1& 0\\
0 &0 & 1& 0& 0& 1
\end{bmatrix}
\grstep{ L_1 - 4L_3 }
\begin{bmatrix}
1 &0 & 0& 1& 0& -4\\
0 &1 & 0& 0& 1& 0\\
0 &0 & 1& 0& 0& 1
\end{bmatrix}
=[I| (I + A + A^2)^{-1}].
\]
Logo, $(I + A + A^2)^{-1} = \begin{bmatrix}
1 &0 &-4\\
0 &1 &0\\
0 &0 &1\\
\end{bmatrix}
= I - A$.
\end{enumerate}


\Exercise[title={1,8}] Sejam $T = \begin{bmatrix}
-5&-12&-18\\
 2&  5&  6\\
 1&  2&  4
\end{bmatrix}$, $I$ a matriz identidade $3 \times 3$, $X = \begin{bmatrix}
x\\
y\\
z
\end{bmatrix}$ e $0 = \begin{bmatrix}
0\\
0\\
0
\end{bmatrix}$. Expresse as equações a seguir como sistemas lineares homogêneos e determine suas soluções:
\begin{multicols}{2}
\begin{enumerate}
\item $(T - I)X = 0$
\item $(T - 2I)X = 0$
\end{enumerate}
\end{multicols}
\Answer Observe que $T - I = \begin{bmatrix}
-6&-12&-18\\
 2&  4&  6\\
 1&  2&  3
\end{bmatrix}$ e $T - 2I = \begin{bmatrix}
-7&-12&-18\\
 2&  3&  6\\
 1&  2&  2
\end{bmatrix}$.

Para resolver os sistemas lineares homogêneos associados a estas matrizes, basta usar a eliminação de Gauss-Jordan:
\begin{enumerate}

\item
\begin{align*}
& \begin{amatrix}{3}
-6&-12&-18& 0\\
 2&  4&  6& 0\\
 1&  2&  3& 0
\end{amatrix}
\grstep{ -\frac{1}{6}L_1 }
\begin{amatrix}{3}
 1&  2&  3& 0\\
 2&  4&  6& 0\\
 1&  2&  3& 0
\end{amatrix}
\grstep{ L_2 - 2 L_1 }
\begin{amatrix}{3}
 1&  2&  3& 0\\
 0&  0&  0& 0\\
 1&  2&  3& 0
\end{amatrix}
\grstep{ L_3 - L_1 }
\begin{amatrix}{3}
 1&  2&  3& 0\\
 0&  0&  0& 0\\
 0&  0&  0& 0
\end{amatrix}
\end{align*}
Assim, o sistema homogêneo é equivalente à equação $x + 2y + 3z = 0$, que é satisfeita por qualquer $(x,y,z)$ em que $x = -2y - 3z$, isto é, $(-2y - 3z, y, z)$, com $y,z$ arbitrários.
\item
\begin{align*}
& \begin{amatrix}{3}
-7&-12&-18& 0\\
 2&  3&  6& 0\\
 1&  2&  2& 0
\end{amatrix}
\grstep{ -\frac{1}{7}L_1 }
\begin{amatrix}{3}
 1&12/7&18/7& 0\\
 2&  3&  6& 0\\
 1&  2&  2& 0
\end{amatrix}
\grstep{ L_2 - 2 L_1 }
\begin{amatrix}{3}
 1&12/7&18/7& 0\\
 0& -3/7 & 6/7 & 0\\
 1&  2&  2& 0
\end{amatrix}\\
\grstep{ L_3 - L_1 }
&
\begin{amatrix}{3}
 1&12/7&18/7& 0\\
 0& -3/7 & 6/7 & 0\\
 0&  2/7& -4/7& 0
\end{amatrix}
\grstep{ -\frac{7}{3} L_2 }
\begin{amatrix}{3}
 1&12/7&18/7& 0\\
 0& 1 & -2 & 0\\
 0&  2/7& -4/7& 0
\end{amatrix}
\grstep{ L_3 - \frac{2}{7} L_2 }
\begin{amatrix}{3}
 1&12/7&18/7& 0\\
 0& 1 & -2 & 0\\
 0& 0 & 0& 0
\end{amatrix}\\
\grstep{ L_1 - \frac{12}{7} L_2 }
&
\begin{amatrix}{3}
 1& 0 & 6 & 0\\
 0& 1 & -2 & 0\\
 0& 0 & 0& 0
\end{amatrix}
\end{align*}
Assim, o sistema homogêneo é equivalente a
\[
\systeme{
x + 6 z = 0,
y - 2 z = 0
}
\]
que é satisfeito por qualquer $(x, y, z)$ em que $x = -6z$ e $y = 2z$, ou seja, qualquer $(-6z, 2z, z)$, com $z$ arbitrário.
\end{enumerate}



\Exercise[title={1,8}] Se $A = \begin{bmatrix}
1 & 2\\3&4\\0&5
\end{bmatrix}$, determine se existe $X = \begin{bmatrix}
x&z&u\\y&w&v
\end{bmatrix}$ tal que $A \cdot X = I \in M_{3 \times 3} (\R)$ (em outras palavras, verifique se alguma matriz $X$ pode ser \emph{``inversa à direita''} de $A$).

\Answer Sejam $A = \begin{bmatrix}
1 & 2\\
3 & 4\\
0 & 5
\end{bmatrix}$ e $X = \begin{bmatrix}
x & z & u\\
y & w & v
\end{bmatrix}$ tais que $AX = I$, isto é,
\[
\begin{bmatrix}
1 & 2\\
3 & 4\\
0 & 5
\end{bmatrix}
\cdot
\begin{bmatrix}
x & z & u\\
y & w & v
\end{bmatrix}
=\begin{bmatrix}
1 & 0 & 0 \\
0 & 1 & 0 \\
0 & 0 & 1
\end{bmatrix}.
\]
Então as colunas de $X$ são soluções dos sistemas
\[
\begin{cases}
x+2y = 1\\
3x+4y = 0\\
5y = 0,
\end{cases}
\begin{cases}
z+2w = 1\\
3z+4w = 0\\
5w = 0
\end{cases}
\text{ e }
\begin{cases}
u+2v = 1\\
3u+4v = 0\\
5v = 0.
\end{cases}
\]
Como todos têm a mesma matriz de coeficientes, pode-se fazer um único escalonamento:
\begin{align*}
\begin{bmatrix}
1 & 2 & 1 & 0 & 0 \\
3 & 4 & 0 & 1 & 0 \\
0 & 5 & 0 & 0 & 1
\end{bmatrix}
& \grstep{ L_2 - 3 L_1 }
\begin{bmatrix}
1 & 2 & 1 & 0 & 0 \\
0 & -2 & -3 & 1 & 0 \\
0 & 5 & 0 & 0 & 1
\end{bmatrix}
\grstep{ -\frac{1}{2} L_2 }
\begin{bmatrix}
1 & 2 & 1 & 0 & 0 \\
0 & 1 & 3/2 & -1/2 & 0 \\
0 & 5 & 0 & 0 & 1
\end{bmatrix} \\
& \grstep{ L_3 - 5 L_2 }
\begin{bmatrix}
1 & 2 & 1 & 0 & 0 \\
0 & 1 & 3/2 & -1/2 & 0 \\
0 & 0 & -15/2 & 5/2 & 1
\end{bmatrix}
\grstep{ L_1 - 2 L_2 }
\begin{bmatrix}
1 & 0 & -2 & 1 & 0 \\
0 & 1 & 3/2 & -1/2 & 0 \\
0 & 0 & -15/2 & 5/2 & 1
\end{bmatrix}
\end{align*}
A última linha indica que $x,y,z,w,u,v$ precisariam satisfazer $0x + 0y = -15/2$, $0z + 0w = 5/2$ e $0u + 0v = 1$, o que é impossível. Logo não existe $X \in M_{2 \times 3} (\R)$ tal que $AX = I$.


\Exercise[title={1,8}] Uma matriz quadrada $Q$ que satisfaz $Q^2 = Q$ é chamada de \emph{matriz idempotente}.
Mostre que se $A$ for uma matriz $n \times n$ idempotente, então:
\begin{enumerate}
\item $I-A$ também é idempotente
\item $(2A-I)^{-1} = 2A - I$
\end{enumerate}

\Answer
\begin{enumerate}

\item Se $A$ é idempotente então
\[
(I - A)^2
= (I - A)(I - A)
= I^2 - IA - AI + A^2
= I - A - A + A
= I - A,
\]
ou seja, $I-A$ também é idempotente.

\item Se $A$ é idempotente então:
\[
(2A - I) (2A - I)
= 4A^2 - (2A)I - I(2A) + I^2
= 4A - 2A - 2A + I
= I,
\]
Isto significa que $(2A-I)^{-1} = 2A - I$
\end{enumerate}


\Exercise[title={1,8}] Para que valores de $m$ o sistema $\begin{cases}
 -x    -4z =-6 \\
 2x    +9z =13 \\
     y -3z =-1 \\
-5x-3y-11z =m-30 \\
\end{cases}$ \textbf{não} tem solução?
\Answer Aplicando-se o método de Gauss-Jordan à matriz aumentada, resulta:
\begin{align*}
&
\begin{amatrix}{3}
-1 &  0 & -4  & -6 \\
 2 &  0 &  9  & 13 \\
 0 &  1 & -3  & -1 \\
-5 & -3 & -11 & m-30
\end{amatrix}
\grstep{ -L_1 }
\begin{amatrix}{3}
 1 &  0 &  4  &  6 \\
 2 &  0 &  9  & 13 \\
 0 &  1 & -3  & -1 \\
-5 & -3 & -11 & m-30
\end{amatrix}
\grstep{ L_2 - 2 L_1 }
\begin{amatrix}{3}
 1 &  0 &   4 &  6 \\
 0 &  0 &   1 &  1 \\
 0 &  1 &  -3 & -1 \\
-5 & -3 & -11 & m-30
\end{amatrix}\\
\grstep{ L_4 +5 L_1 }
&
\begin{amatrix}{3}
1 &  0 &  4 &  6 \\
0 &  0 &  1 &  1 \\
0 &  1 & -3 & -1 \\
0 & -3 &  9 &  m
\end{amatrix}
\grstep{ L_2 \swap L_3 }
\begin{amatrix}{3}
1 &  0 &  4 &  6 \\
0 &  1 & -3 & -1 \\
0 &  0 &  1 &  1 \\
0 & -3 &  9 &  m
\end{amatrix}
\grstep{ L_4 +3 L_2 }
\begin{amatrix}{3}
1 & 0 &  4 &  6 \\
0 & 1 & -3 & -1 \\
0 & 0 &  1 &  1 \\
0 & 0 &  0 &  m-3
\end{amatrix}\\
\grstep{ L_2 +3 L_3 }
&
\begin{amatrix}{3}
1 & 0 & 4 &  6 \\
0 & 1 & 0 &  2 \\
0 & 0 & 1 &  1 \\
0 & 0 & 0 &  m-3
\end{amatrix}
\grstep{ L_1 -4 L_3 }
\begin{amatrix}{3}
1 & 0 & 0 &  2 \\
0 & 1 & 0 &  2 \\
0 & 0 & 1 &  1 \\
0 & 0 & 0 &  m-3
\end{amatrix}.
\end{align*}
e é equivalente por linhas a
\[
\begin{amatrix}{3}
1 & 0 & 0 & 2 \\
0 & 1 & 0 & 2 \\
0 & 0 & 1 & 1 \\
0 & 0 & 0 & m-3
\end{amatrix}.
\]
Então para que não existam soluções, é preciso que $m-3 \neq 0$, isto é, que $m \neq 3$.
\end{ExerciseList}

\medskip
\begin{note}
As questões anteriores passam a valer 2,0 pontos caso:
\begin{enumerate}
\item \textbf{(0,1) Organize} respostas legíveis para \textbf{tudo} o que foi perguntado
\item \textbf{(0,1) Explique} a resolução por escrito, mostrando onde usou a teoria estudada
\end{enumerate}
\end{note}

\begin{center}
BOA PROVA!
\end{center}

\newpage
\restoregeometry
\section*{Respostas}
\shipoutAnswer
\end{document}
