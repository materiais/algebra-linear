\documentclass[12pt,a4paper]{article}
\usepackage{cmap} % Makes the PDF copiable. See http://tex.stackexchange.com/a/64198/25761
\usepackage[T1]{fontenc}
\usepackage[brazil]{babel}
\usepackage[utf8]{inputenc}
\usepackage{amsmath}
\usepackage{amsfonts}
\usepackage{amssymb}
\usepackage{amsthm}
\usepackage{textcomp} % \degree
\usepackage{gensymb} % \degree
\usepackage[usenames,svgnames,dvipsnames]{xcolor}
\usepackage{hyperref}
\usepackage{multicol}
\usepackage{graphicx}
\usepackage[margin=2cm]{geometry}
\usepackage{systeme}

\hypersetup{
    colorlinks = true,
    allcolors = {blue}
}

% TODO: Consider using exsheets
% http://linorg.usp.br/CTAN/macros/latex/contrib/exsheets/exsheets_en.pdf
%
% http://ctan.org/tex-archive/macros/latex/contrib/exercise/
% Options: answerdelayed,lastexercise,noanswer
\usepackage[answerdelayed,lastexercise]{exercise}

\addto\captionsbrazil{%
\def\listexercisename{Lista de exerc\'icios}%
\def\ExerciseName{Exerc\'icio}%
\def\AnswerName{Solu\c{c}\~ao do exerc\'icio}%
\def\ExerciseListName{Ex.}%
\def\AnswerListName{Solu\c{c}\~ao}%
\def\ExePartName{Parte}%
\def\ArticleOf{de\ }%
}

\renewcommand{\ExerciseHeaderTitle}{(\ExerciseTitle)\ }
\renewcommand{\ExerciseListHeader}{%\ExerciseHeaderDifficulty%
\textbf{%\ExerciseListName\
\ExerciseHeaderNB.\ %
%\ --- \
\ExerciseHeaderTitle}%
%\ExerciseHeaderOrigin
\ignorespaces}
\renewcommand{\AnswerListHeader}{\textbf{\ExerciseHeaderNB.\ (\AnswerListName)\ }}

\newtheorem*{note}{Observação}
\newcommand*\R{\mathbb{R}}

% Loop Space / CC BY-SA-3.0 / https://tex.stackexchange.com/a/2238/25761
\newenvironment{amatrix}[1]{%
  \left[\begin{array}{@{}*{#1}{c}|c@{}}
}{%
  \end{array}\right]
}

% Loop Space / CC BY-SA-3.0 / https://tex.stackexchange.com/a/3164/25761
%--------grstep
% For denoting a Gauss' reduction step.
% Use as: \grstep{\rho_1+\rho_3} or \grstep[2\rho_5 \\ 3\rho_6]{\rho_1+\rho_3}
\newcommand{\grstep}[2][\relax]{%
   \ensuremath{\mathrel{
       {\mathop{\longrightarrow}\limits^{#2\mathstrut}_{
                                     \begin{subarray}{l} #1 \end{subarray}}}}}}

\renewcommand{\theenumi}{\alph{enumi}}
\renewcommand\labelenumi{(\theenumi) }

\newcommand*\tipo{Prova I}
\newcommand*\turma{PRO112-02A}
\newcommand*\disciplina{ALI0001}
\newcommand*\eu{Helder G. G. de Lima}
\newcommand*\data{23/03/2016}

\author{\eu}
\title{\tipo - \disciplina}
\date{\data}

\begin{document}
\thispagestyle{empty}
\newgeometry{margin=2cm,bottom=0.5cm}
\begin{center}
\includegraphics[width=9.0cm]{marca} \\
\textbf{\tipo\ (\disciplina / \turma)} \\
Prof. \eu\footnote{
Este é um material de acesso livre distribuído sob os termos da licença \href{https://creativecommons.org/licenses/by-sa/4.0/deed.pt_BR}{Creative Commons BY-SA 4.0}}
\end{center}

\noindent Nome do(a) aluno(a): \underline{\hspace{9,7cm}} Data: \underline{\data}

%\section*{Instruções}
\begin{center}\fbox{
\begin{minipage}{14cm}

{\footnotesize
\begin{itemize}
\renewcommand{\theenumi}{\Roman{enumi}}
\item Identifique-se em todas as folhas.
\item Mantenha o celular e os demais equipamentos eletrônicos desligados durante a prova.
\item Anule \textsc{\textbf{uma}} das 6 questões (apenas 5 serão corrigidas): \framebox(30,10){}
\end{itemize}
}

\end{minipage}
}
\end{center}

\section*{Questões}
\begin{ExerciseList}


\Exercise[title={1,8}] Determine as soluções do sistema linear
$
\systeme[xyzw]{
3x+9y+3w=36,
-x-3y+z-w=-1,
2x+6y+6w=44
}
$
\Answer A eliminação de Gauss-Jordan para a matriz aumentada associada ao sistema consiste dos seguintes passos:
\begin{align*}
&
\begin{amatrix}{4}
3 & 9 & 0 & 3 & 36 \\
-1 & -3 & 1 & -1& -1\\
2 & 6 & 0 & 6 & 44
\end{amatrix}
\grstep{ \frac{1}{3}L_1 }
\begin{amatrix}{4}
1 & 3 & 0 & 1 & 12 \\
-1 & -3 & 1 & -1& -1\\
2 & 6 & 0 & 6 & 44
\end{amatrix}
\grstep{ L_2 + L_1 }
\begin{amatrix}{4}
1 & 3 & 0 & 1 & 12 \\
0 & 0 & 1 & 0 & 11\\
2 & 6 & 0 & 6 & 44
\end{amatrix}\\
&
\grstep{ L_3 - 2 L_1 }
\begin{amatrix}{4}
1 & 3 & 0 & 1 & 12 \\
0 & 0 & 1 & 0 & 11\\
0 & 0 & 0 & 4 & 20
\end{amatrix}
\grstep{ \frac{1}{4} L_3 }
\begin{amatrix}{4}
1 & 3 & 0 & 1 & 12 \\
0 & 0 & 1 & 0 & 11\\
0 & 0 & 0 & 1 & 5
\end{amatrix}
\grstep{ L_1-L_3 }
\begin{amatrix}{4}
1 & 3 & 0 & 0 & 7 \\
0 & 0 & 1 & 0 & 11\\
0 & 0 & 0 & 1 & 5
\end{amatrix}
\end{align*}
Disto conclui-se que o sistema original é equivalente a
\[
\systeme[xyzw]{
x+3y=7,
z=11,
w=5
}
\]
cujas soluções são da forma $(x,y,z,w) = (7-3y,y,11,5)$, com $y$ arbitrário.


\Exercise[title={1,8}] Considere a matriz quadrada $A = \begin{bmatrix}
-1 & 5 & 5\\
3 & -17 & -18\\
-3 & 18 & 19
\end{bmatrix}$, $X=\begin{bmatrix}x\\y\\z\end{bmatrix}$ e $B=\begin{bmatrix}-1\\1\\-1\end{bmatrix}$.
Calcule $A^{-1}$ e utilize-a para determinar os valores de $X$ para que $AX = B$.
\Answer O cálculo da inversa é feito pelo método de Gauss-Jordan:
\begin{align*}
[ A | I] =
& \begin{bmatrix}
-1 &   5 &   5 & 1 & 0 & 0 \\
 3 & -17 & -18 & 0 & 1 & 0 \\
-3 &  18 &  19 & 0 & 0 & 1
\end{bmatrix}
\grstep{ -L_1 }
\begin{bmatrix}
 1 &  -5 &  -5 & -1 & 0 & 0 \\
 3 & -17 & -18 &  0 & 1 & 0 \\
-3 &  18 &  19 &  0 & 0 & 1
\end{bmatrix} \\
\grstep{ L_2 - 3L_1 }
& \begin{bmatrix}
 1 &  -5 & -5 & -1 & 0 & 0 \\
 0 &  -2 & -3 &  3 & 1 & 0 \\
-3 &  18 & 19 &  0 & 0 & 1
\end{bmatrix}
\grstep{ L_3 + 3L_1 }
\begin{bmatrix}
1 & -5 & -5 & -1 & 0 & 0 \\
0 & -2 & -3 &  3 & 1 & 0 \\
0 &  3 &  4 & -3 & 0 & 1
\end{bmatrix}\\
\grstep{ -\frac{1}{2} L_2 }
& \begin{bmatrix}
1 & -5 & -5 & -1 & 0 & 0 \\
0 &  1 & 3/2 & -3/2 & -1/2 & 0 \\
0 &  3 &  4 & -3 & 0 & 1
\end{bmatrix}
\grstep{ L_3 - 3L_2 }
\begin{bmatrix}
1 & -5 & -5 & -1 & 0 & 0 \\
0 &  1 & 3/2 & -3/2 & -1/2 & 0 \\
0 &  0 & -1/2 & 3/2 & 3/2 & 1
\end{bmatrix}\\
\grstep{ -2 L_3}
& \begin{bmatrix}
1 & -5 & -5 & -1 & 0 & 0 \\
0 &  1 & 3/2 & -3/2 & -1/2 & 0 \\
0 &  0 &  1 & -3 & -3 & -2
\end{bmatrix}
\grstep{ L_2 - \frac{3}{2} L_3 }
\begin{bmatrix}
1 & -5 & -5 & -1 & 0 & 0 \\
0 &  1 &  0 &  3 & 4 & 3 \\
0 &  0 &  1 & -3 & -3 & -2
\end{bmatrix}\\
\grstep{ L_1 + 5 L_3 }
& \begin{bmatrix}
1 & -5 & 0 & -16 & -15 & -10 \\
0 & 1 & 0 &  3 & 4 & 3 \\
0 & 0 & 1 & -3 & -3 & -2
\end{bmatrix}
\grstep{ L_1 + 5 L_2 }
\begin{bmatrix}
1 & 0 & 0 & -1 &  5 &  5 \\
0 & 1 & 0 &  3 &  4 &  3 \\
0 & 0 & 1 & -3 & -3 & -2
\end{bmatrix}
=[I|A^{-1}].
\end{align*}
Assim, $A^{-1}=
\begin{bmatrix}
-1 &  5 & 5 \\
 3 &  4 & 3 \\
-3 & -3 & -2
\end{bmatrix}$. Sendo $AX = B$, e $A$ uma matriz inversível, segue do item anterior que
\[
X = A^{-1}B =
\begin{bmatrix}
-1 &  5 & 5 \\
 3 &  4 & 3 \\
-3 & -3 & -2
\end{bmatrix}
\cdot
\begin{bmatrix}-1\\1\\-1\end{bmatrix}
=\begin{bmatrix}1\\-2\\2\end{bmatrix}.
\]


\Exercise[title={1,8}] Seja $W =
\begin{bmatrix}
1 & -1 & 0 \\
0 & -1 & -1 \\
0 & 0 & 1
\end{bmatrix}$. Verifique que $W \cdot W^T$ é uma matriz simétrica e inversível.
\Answer A simetria de $W \cdot W^T$ é uma consequência imediata das propriedades da transposição e da multiplicação de matrizes:
\[
(W \cdot W^T)^T = (W^T)^T \cdot W = W \cdot W^T.
\]
Quanto à inversibilidade de $W \cdot W^T$, há duas opções:
\begin{enumerate}
\item Fazer o produto $W \cdot W^T$ e depois calcular a sua inversa explicitamente, por Gauss-Jordan.
\item Verificar que a forma escalonada reduzida de $W$ (que já tem alguns zeros em posições convenientes) é igual à identidade (pois isso é equivalente a $W$ ser inversível), e lembrar que o produto de matrizes inversíveis é inversível, e que a inversa da transposta é a transposta da inversa.
\end{enumerate}
Os detalhes de cada abordagem são como segue:
\begin{enumerate}
\item $W \cdot W^T
= \begin{bmatrix}
1 & -1 & 0 \\
0 & -1 & -1 \\
0 & 0 & 1
\end{bmatrix}
\cdot
\begin{bmatrix}
 1 & 0 & 0 \\
-1 & -1 & 0 \\
 0 & -1 & 1
\end{bmatrix}
= \begin{bmatrix}
2 & 1 & 0 \\
1 &  2 & -1 \\
0 & -1 &  1
\end{bmatrix}$ então
\begin{align*}
[ W \cdot W^T | I ] =
& \begin{bmatrix}
2 &  1 &  0 & 1 & 0 & 0 \\
1 &  2 & -1 & 0 & 1 & 0 \\
0 & -1 &  1 & 0 & 0 & 1
\end{bmatrix}
\grstep{ \frac{1}{2} L_1 }
\begin{bmatrix}
1 &  1/2 &  0 & 1/2 & 0 & 0 \\
1 &  2 & -1 & 0 & 1 & 0 \\
0 & -1 &  1 & 0 & 0 & 1
\end{bmatrix}\\
\grstep{ L_2 - L_1 }
&
\begin{bmatrix}
1 &  1/2 &  0 &  1/2 & 0 & 0 \\
0 &  3/2 & -1 & -1/2 & 1 & 0 \\
0 & -1 &  1 & 0 & 0 & 1
\end{bmatrix}
\grstep{ \frac{2}{3}L_2 }
\begin{bmatrix}
1 &  1/2 &  0 &  1/2 & 0 & 0 \\
0 &  1 & -2/3 & -1/3 & 2/3 & 0 \\
0 & -1 &  1 & 0 & 0 & 1
\end{bmatrix}\\
\grstep{ L_3 + L_2 }
&
\begin{bmatrix}
1 & 1/2 &  0 &  1/2 & 0 & 0 \\
0 & 1 & -2/3 & -1/3 & 2/3 & 0 \\
0 & 0 &  1/3 & -1/3 & 2/3 & 1
\end{bmatrix}
\grstep{ 3L_3 }
\begin{bmatrix}
1 & 1/2 &  0 &  1/2 & 0 & 0 \\
0 & 1 & -2/3 & -1/3 & 2/3 & 0 \\
0 & 0 &  1 & -1 & 2 & 3
\end{bmatrix}\\
\grstep{ L_2+\frac{2}{3}L_3 }
&
\begin{bmatrix}
1 & 1/2 & 0 & 1/2 & 0 & 0 \\
0 &   1 & 0 &  -1 & 2 & 2 \\
0 &   0 & 1 &  -1 & 2 & 3
\end{bmatrix}
\grstep{ L_1 - \frac{1}{2}L_2 }
\begin{bmatrix}
1 & 0 & 0 &  1 & -1 & -1 \\
0 & 1 & 0 & -1 & 2 & 2 \\
0 & 0 & 1 & -1 & 2 & 3
\end{bmatrix}
\end{align*}
Portanto existe $(W \cdot W^T)^{-1} = \begin{bmatrix}
1 & -1 & -1 \\
-1 & 2 & 2 \\
-1 & 2 & 3
\end{bmatrix}$ e $W \cdot W^T$ é, de fato, inversível.
\item A forma escalonada reduzida por linhas da matriz $W$ é obtida assim:
\begin{align*}
W=\begin{bmatrix}
1 & -1 & 0 \\
0 & -1 & -1 \\
0 & 0 & 1
\end{bmatrix}
\grstep{ -L_2 }
\begin{bmatrix}
1 & -1 & 0 \\
0 & 1 & 1 \\
0 & 0 & 1
\end{bmatrix}
\grstep{ L_2 - L_3 }
\begin{bmatrix}
1 & -1 & 0 \\
0 & 1 & 0 \\
0 & 0 & 1
\end{bmatrix}
\grstep{ L_1 + L_2 }
\begin{bmatrix}
1 & 0 & 0 \\
0 & 1 & 0 \\
0 & 0 & 1
\end{bmatrix}
\end{align*}
Como foi obtida a matriz identidade, conclui-se que a matriz $W$ é inversível. De fato, a inversa poderia ser calculada por Gauss-Jordan, e o resultado seria $W^{-1} = \begin{bmatrix}
1 & -1 & -1 \\
0 & -1 & -1 \\
0 &  0 &  1
\end{bmatrix}$
Como $W$ é inversível, resulta que $W^T$ também é inversível, e o produto destas matrizes inversíveis será inversível. A inversa pode ser obtida por
{\footnotesize
\[
  (W \cdot W^T)^{-1}
= (W^T)^{-1} \cdot W^{-1}
= (W^{-1})^T \cdot W^{-1}
= \begin{bmatrix}
 1 &  0 & 0 \\
-1 & -1 & 0 \\
-1 & -1 & 1
\end{bmatrix}
\cdot
\begin{bmatrix}
1 & -1 & -1 \\
0 & -1 & -1 \\
0 &  0 &  1
\end{bmatrix}
=
\begin{bmatrix}
1 & -1 & -1 \\
-1 & 2 & 2 \\
-1 & 2 & 3
\end{bmatrix}.
\]
}

\end{enumerate}


\Exercise[title={1,8}] Mostre que se $A$, $B$ e $A+B$ são inversíveis, então $A(A^{-1}+B^{-1})B(A+B)^{-1} = I$.

\Answer Para mostrar que a igualdade é sempre válida, independentemente de quais sejam as matrizes inversíveis $A$, $B$ e $A+B$, basta usar as propriedades das operações com matrizes, como segue:
\begin{align*}
A(A^{-1}+B^{-1})B(A+B)^{-1}
& = (AA^{-1} + AB^{-1})B(A+B)^{-1}
  = (I + AB^{-1})B(A+B)^{-1}\\
& = (IB+AI)(A+B)^{-1}
  = (B+A)(A+B)^{-1}
  = I
\end{align*}
\textbf{Nota:} cuidado para não escrever que $(A+B)^{-1} = A^{-1} + B^{-1}$, pois em geral isto não é verdade, tal como ocorre para números reais: $(2 + 3)^{-1} = \frac{1}{5}$ enquanto que  $2^{-1} + 3^{-1} = \frac{5}{6}$.


\Exercise[title={1,8}] Decida quais das afirmações são verdadeiras para todas as matrizes inversíveis $A$ e $B$ (prove-as) e quais são falsas para algumas matrizes (neste caso, dê um exemplo de que é falsa).

\begin{multicols}{2}
\begin{enumerate}
\item $3B^2$ também é inversível
\item $A-B$ também é inversível
\item Se $A$ é simétrica, $A^{-1}$ também é
\item Se $A$ é antissimétrica, $A^{-1}$ também é
\end{enumerate}
\end{multicols}

\Answer
\begin{enumerate}
\item \textbf{Verdadeiro}. Se $A$ é inversível então
\[
(3B^2)(\frac{1}{3}(B^{-1})^2)
= 3 \frac{1}{3} (B^2(B^{-1})^2)
= (BB^{-1})^2
= I^2 = I,
\]
o que significa que $3B^2$ é inversível, e sua inversa é $\frac{1}{3}(B^{-1})^2$.
\item \textbf{Falso}. Se $A$ é qualquer matriz inversível (por exemplo $A=I$), e $B = A$ então $A-B = A-A = 0$, que não é uma matriz inversível.
\item \textbf{Verdadeiro}. Se $A$ é simétrica, então $A^T = A$ e resulta que
\[
(A^{-1})^T = (A^T)^{-1} = A^{-1}
\]
Logo, $A^{-1}$ também é simétrica.
\item \textbf{Verdadeiro}. Se $A$ é antissimétrica, então $A^T = -A$ e resulta que
\[
(A^{-1})^T = (A^T)^{-1} = (-A)^{-1} = -(A^{-1})
\]
Logo, $A^{-1}$ também é antissimétrica.
\end{enumerate}


\Exercise[title={1,8}] Considere o conjunto $M_{3 \times 3} (\R)$ das matrizes $3 \times 3$ com entradas em $\R$.
\begin{enumerate}
\item Dê um exemplo de uma matriz $A \in M_{3 \times 3} (\R)$ tal que $A = A^{-1}$.
\item Encontre mais um exemplo $B \neq A$ que satisfaça $B = B^{-1}$.
\item Calcule $C^2$, sendo $C = \frac{1}{2}(B + I)$ e $B = B^{-1}$ como acima.
\end{enumerate}

\Answer
\begin{enumerate}
\item
Se $A = \begin{bmatrix}
1 & 0 & 0\\
0 & 1 & 0\\
0 & 0 & 1
\end{bmatrix}$ então $A^2 = I$, e consequentemente $A = A^{-1}$.
\item Se $B = \begin{bmatrix}
0 & 1 & 0 \\
1 & 0 & 0 \\
0 & 0 & 1
\end{bmatrix}$ então $B^2 = I$ e $B = B^{-1}$.
\item $C^2 = \left( \frac{1}{2}(B+I) \right)^2
= \frac{1}{4}(B^2 + BI + IB + I^2)
= \frac{1}{4}(2B + 2I)
= \frac{1}{2}(B + I) = C$
\end{enumerate}
\end{ExerciseList}

\medskip
\begin{note}
As questões anteriores passam a valer 2,0 pontos caso:
\begin{enumerate}
\item \textbf{(0,1) Organize} respostas legíveis para \textbf{tudo} o que foi perguntado
\item \textbf{(0,1) Explique} a resolução por escrito, mostrando onde usou a teoria estudada
\end{enumerate}
\end{note}

\begin{center}
BOA PROVA!
\end{center}

\newpage
\restoregeometry
\section*{Respostas}
\shipoutAnswer
\end{document}
