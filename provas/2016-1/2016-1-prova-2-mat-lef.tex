\documentclass[12pt,a4paper]{article}
\usepackage{cmap} % Makes the PDF copiable. See http://tex.stackexchange.com/a/64198/25761
\usepackage[T1]{fontenc}
\usepackage[brazil]{babel}
\usepackage[utf8]{inputenc}
\usepackage{amsmath}
\usepackage{amsfonts}
\usepackage{amssymb}
\usepackage{amsthm}
\usepackage{textcomp} % \degree
\usepackage{gensymb} % \degree
\usepackage[usenames,svgnames,dvipsnames]{xcolor}
\usepackage{hyperref}
\usepackage{multicol}
\usepackage{graphicx}
\usepackage[margin=2cm]{geometry}
\usepackage{systeme}

\hypersetup{
    colorlinks = true,
    allcolors = {blue}
}

% TODO: Consider using exsheets
% http://linorg.usp.br/CTAN/macros/latex/contrib/exsheets/exsheets_en.pdf
%
% http://ctan.org/tex-archive/macros/latex/contrib/exercise/
% Options: answerdelayed,lastexercise,noanswer
\usepackage[answerdelayed,lastexercise]{exercise}

\addto\captionsbrazil{%
\def\listexercisename{Lista de exerc\'icios}%
\def\ExerciseName{Exerc\'icio}%
\def\AnswerName{Solu\c{c}\~ao do exerc\'icio}%
\def\ExerciseListName{Ex.}%
\def\AnswerListName{Solu\c{c}\~ao}%
\def\ExePartName{Parte}%
\def\ArticleOf{de\ }%
}

\renewcommand{\ExerciseHeaderTitle}{(\ExerciseTitle)\ }
\renewcommand{\ExerciseListHeader}{%\ExerciseHeaderDifficulty%
\textbf{%\ExerciseListName\
\ExerciseHeaderNB.\ %
%\ --- \
\ExerciseHeaderTitle}%
%\ExerciseHeaderOrigin
\ignorespaces}
\renewcommand{\AnswerListHeader}{\textbf{\ExerciseHeaderNB.\ (\AnswerListName)\ }}

\newtheorem*{note}{Observação}
\newcommand{\fixme}{{\color{red}(...)}}
\newcommand*\ger[1]{\operatorname{ger}\left\{#1\right\}}
\newcommand*\R{\mathbb{R}}

% Loop Space / CC BY-SA-3.0 / https://tex.stackexchange.com/a/2238/25761
\newenvironment{amatrix}[1]{%
  \left[\begin{array}{@{}*{#1}{c}|c@{}}
}{%
  \end{array}\right]
}

% Loop Space / CC BY-SA-3.0 / https://tex.stackexchange.com/a/3164/25761
%--------grstep
% For denoting a Gauss' reduction step.
% Use as: \grstep{\rho_1+\rho_3} or \grstep[2\rho_5 \\ 3\rho_6]{\rho_1+\rho_3}
\newcommand{\grstep}[2][\relax]{%
   \ensuremath{\mathrel{
       {\mathop{\longrightarrow}\limits^{#2\mathstrut}_{
                                     \begin{subarray}{l} #1 \end{subarray}}}}}}

\renewcommand{\theenumi}{\alph{enumi}}
\renewcommand\labelenumi{(\theenumi) }

\newcommand*\tipo{Prova II}
\newcommand*\turma{MAT102-03U/LEF102-02U}
\newcommand*\disciplina{ALI0001/ALG2001}
\newcommand*\eu{Helder G. G. de Lima}
\newcommand*\data{28/04/2016}

\author{\eu}
\title{\tipo - \disciplina}
\date{\data}

\begin{document}
\thispagestyle{empty}
\newgeometry{margin=2cm,bottom=0.5cm}
\begin{center}
\includegraphics[width=9.0cm]{marca} \\
\textbf{\tipo\ (\disciplina / \turma)} \\
Prof. \eu\footnote{
Este é um material de acesso livre distribuído sob os termos da licença \href{https://creativecommons.org/licenses/by-sa/4.0/deed.pt_BR}{Creative Commons Atribuição-CompartilhaIgual 4.0 Internacional}}
\end{center}

\noindent Nome do(a) aluno(a): \underline{\hspace{9,7cm}} Data: \underline{\data}

%\section*{Instruções}
\begin{center}\fbox{
\begin{minipage}{14cm}

{\footnotesize
\begin{itemize}
\renewcommand{\theenumi}{\Roman{enumi}}
\item Identifique-se em todas as folhas.
\item Mantenha o celular e os demais equipamentos eletrônicos desligados durante a prova.
\item Anule \textsc{\textbf{uma}} das 5 questões (apenas 4 serão corrigidas): \framebox(30,10){}
\end{itemize}
}

\end{minipage}
}
\end{center}

%\section*{Questões}
\begin{ExerciseList}
\Exercise[title={2,5}]
Analise as afirmações a seguir e decida quais são verdadeiras.
\begin{itemize}
\item Se for \textbf{verdadeira}, explique-a com um argumento lógico convincente, e os devidos cálculos.
\item Se for \textbf{falsa}, forneça um exemplo simples que mostre por que é falsa.
\end{itemize}
\begin{enumerate}
\item (0,6) A matriz $\begin{bmatrix}1 & 3 \\5 & 7 \end{bmatrix}$ não pertence a nenhum subespaço vetorial de $M_{2 \times 2}$.
\item (0,6) Sempre que os vetores $u$, $v$ e $w$ são L.I. os vetores $2u$, $3v$ e $5w$ também são L.I..
\item (0,6) A dimensão do espaço linha de uma matriz é igual ao número de linhas da matriz.
\item (0,7) Todos os vetores de qualquer base de $\R^2$ são sempre diferentes de zero.
\end{enumerate}
\Answer \fixme

\Exercise[title={2,5}] Seja $A = \begin{bmatrix}
 1 & -1 &  0 &  0 \\
-1 &  2 & -1 &  0 \\
 0 & -1 &  2 & -1 \\
 0 &  0 & -1 &  1
\end{bmatrix} \in M_{4\times 4}(\R)$.
\begin{enumerate}
\item (1,0) Calcule $\det(A^2)$.
\item (1,0) Encontre uma base e a dimensão do núcleo (ou espaço nulo) de $A$.
\item (0,5) Explique se as colunas de $A$ são ou não uma base do espaço coluna de $A$.
\end{enumerate}

\Answer  \fixme

\Exercise[title={2,5}]
Explique por que $W$ é ou não é subespaço do espaço vetorial $V$ nos casos a seguir:
\begin{enumerate}
\item (1,2) $W = \{ X \in M_{2 \times 2}(\R) \mid \det(X) = 0 \}$, sendo $V = M_{2 \times 2}(\R)$.
\item (1,3) $W = \{ q(x) = ax^2+bx+c \in P_2 \mid q( 0 ) = q(2) = 0 \}$, sendo $V = P_2$.
\end{enumerate}
\Answer \fixme

\Exercise[title={2,5}] Sejam $U = \left\{ \begin{bmatrix}
a & b\\
c & d
\end{bmatrix} \mid a - 2c = b - 2d = 0 \right\}$ e $W = \left\{ A \in M_{2 \times 2} \mid A^T = A \right\}$
\begin{enumerate}
\item (1,2) Encontre um conjunto de vetores que gera o espaço $U \cap W$.
\item (1,3) Obtenha uma base e a dimensão de $U \cap W$.
\end{enumerate}
\Answer  \fixme

\Exercise[title={2,5}] Sejam $C = \{ u, v, x, y \} \subset \R^4$, $u = (1, 2, 3, 0)$, $v = (-1, 1, 1, 0)$, $x = (0, 1, 1, -1)$ e $y = (1, 3, 5, 2)$.
\begin{enumerate}
\item (1,3) Mostre que $C$ é L.D. e escreva um dos vetores como combinação linear dos demais.
\item (1,2) Encontre um subconjunto de $C$ que seja uma base do espaço $W = \ger{u,v,x,y}$.
\end{enumerate}
\Answer  \fixme

\end{ExerciseList}

\begin{center}
BOA PROVA!
\end{center}

%\newpage
%\restoregeometry
%\section*{Respostas}
%\shipoutAnswer
\end{document}
