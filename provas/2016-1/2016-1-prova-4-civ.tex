\documentclass[12pt,a4paper]{article}
\usepackage{cmap} % Makes the PDF copiable. See http://tex.stackexchange.com/a/64198/25761
\usepackage[T1]{fontenc}
\usepackage[brazil]{babel}
\usepackage[utf8]{inputenc}
\usepackage{amsmath}
\usepackage{amsfonts}
\usepackage{amssymb}
\usepackage{amsthm}
\usepackage{textcomp} % \degree
\usepackage{gensymb} % \degree
\usepackage[usenames,svgnames,dvipsnames]{xcolor}
\usepackage{hyperref}
\usepackage{multicol}
\usepackage{graphicx}
\usepackage[margin=2cm]{geometry}
\usepackage{systeme}

\hypersetup{
    colorlinks = true,
    allcolors = {blue}
}

% TODO: Consider using exsheets
% http://linorg.usp.br/CTAN/macros/latex/contrib/exsheets/exsheets_en.pdf
%
% http://ctan.org/tex-archive/macros/latex/contrib/exercise/
% Options: answerdelayed,lastexercise,noanswer
\usepackage[answerdelayed,lastexercise]{exercise}

\addto\captionsbrazil{%
\def\listexercisename{Lista de exerc\'icios}%
\def\ExerciseName{Exerc\'icio}%
\def\AnswerName{Solu\c{c}\~ao do exerc\'icio}%
\def\ExerciseListName{Ex.}%
\def\AnswerListName{Solu\c{c}\~ao}%
\def\ExePartName{Parte}%
\def\ArticleOf{de\ }%
}

\renewcommand{\ExerciseHeaderTitle}{(\ExerciseTitle)\ }
\renewcommand{\ExerciseListHeader}{%\ExerciseHeaderDifficulty%
\textbf{%\ExerciseListName\
\ExerciseHeaderNB.\ %
%\ --- \
\ExerciseHeaderTitle}%
%\ExerciseHeaderOrigin
\ignorespaces}
\renewcommand{\AnswerListHeader}{\textbf{\ExerciseHeaderNB.\ (\AnswerListName)\ }}

\newcommand{\norm}[1]{\left|\left|{#1}\right|\right|}
\newcommand*\ger[1]{\operatorname{ger}\left\{#1\right\}}
\newcommand*\proj{\operatorname{proj}}
\newcommand*\R{\mathbb{R}}

% Loop Space / CC BY-SA-3.0 / https://tex.stackexchange.com/a/2238/25761
\newenvironment{amatrix}[1]{%
  \left[\begin{array}{@{}*{#1}{c}|c@{}}
}{%
  \end{array}\right]
}

% Loop Space / CC BY-SA-3.0 / https://tex.stackexchange.com/a/3164/25761
%--------grstep
% For denoting a Gauss' reduction step.
% Use as: \grstep{\rho_1+\rho_3} or \grstep[2\rho_5 \\ 3\rho_6]{\rho_1+\rho_3}
\newcommand{\grstep}[2][\relax]{%
   \ensuremath{\mathrel{
       {\mathop{\longrightarrow}\limits^{#2\mathstrut}_{
                                     \begin{subarray}{l} #1 \end{subarray}}}}}}

\renewcommand{\theenumi}{\alph{enumi}}
\renewcommand\labelenumi{(\theenumi) }

\newcommand*\tipo{Prova IV}
\newcommand*\turma{CIV122-02U}
\newcommand*\disciplina{ALI0001}
\newcommand*\eu{Helder G. G. de Lima}
\newcommand*\data{27/06/2016}

\author{\eu}
\title{\tipo - \disciplina}
\date{\data}

\begin{document}
\thispagestyle{empty}
\newgeometry{margin=2cm,bottom=0.5cm}
\begin{center}
\includegraphics[width=9.0cm]{marca} \\
\textbf{\tipo\ (\disciplina / \turma)} \\
Prof. \eu\footnote{
Este é um material de acesso livre distribuído sob os termos da licença \href{https://creativecommons.org/licenses/by-sa/4.0/deed.pt_BR}{Creative Commons Atribuição-CompartilhaIgual 4.0 Internacional}}
\end{center}

\noindent Nome do(a) aluno(a): \underline{\hspace{9,7cm}} Data: \underline{\data}

%\section*{Instruções}
\begin{center}\fbox{
\begin{minipage}{14cm}

{\footnotesize
\begin{itemize}
\renewcommand{\theenumi}{\Roman{enumi}}
\item Identifique-se em todas as folhas.
\item Mantenha o celular e os demais equipamentos eletrônicos desligados durante a prova.
\item Anule \textsc{\textbf{uma}} das 5 questões (apenas 4 serão corrigidas): \framebox(30,10){}
\end{itemize}
}

\end{minipage}
}
\end{center}

%\section*{Questões}
\begin{ExerciseList}
\Exercise[title={2,5}] Seja $T: P_3 \to P_3$ o operador derivação, $T(a+bx+cx^2+dx^3) = b+2cx+3dx^2$.
\begin{enumerate}
\item (1,3) Obtenha os autovalores e autovetores de $T$.
\item (1,2) A matriz do operador $T$ em relação à base canônica é diagonalizável? Explique.
\end{enumerate}
\Answer
\begin{enumerate}
\item  Os autovalores de $T$ podem ser obtidos calculando-se as raízes do polinômio característico da matriz de $T$ em relação à qualquer base de $P_3$. Se $A$ é a matriz de $T$ em relação à base canônica $C = \{1, x, x^2, x^3 \}$, então $A = \begin{bmatrix}
0 & 1 & 0 & 0 \\
0 & 0 & 2 & 0 \\
0 & 0 & 0 & 3 \\
0 & 0 & 0 & 0
\end{bmatrix}.$

Consequentemente, o seu polinômio característico é
\[
p_A(x) = \det{(xI - A)} = \begin{vmatrix}
x & -1 & 0 & 0 \\
0 & x & -2 & 0 \\
0 & 0 & x & -3 \\
0 & 0 & 0 & x
\end{vmatrix}
=x^4.
\]
Como este polinômio $c = 0$ é a única raiz deste polinômio, o operador $T$ só possui um autovalor. Os autovetores de $T$ associados ao autovalor $c = 0$ são aqueles polinômios $q(x) = a+bx+cx^2+dx^3$ tais que $T(q) = 0 \cdot q$, isto é, tais que $b+2cx+3dx^2 = 0 + 0x + 0x^2 + 0x^3$. Isto implica que $b=c=d = 0$ e portanto os autovetores são os polinômios do tipo $q(x) = a$, ou seja, os polinômios de grau zero (funções constantes).

\item Para que $A$ fosse diagonalizável, seria preciso que a multiplicidade geométrica de $c = 0$ fosse igual a sua multiplicidade algébrica. Mas isso não ocorre, pois a multiplicidade algébrica do autovalor encontrado é 4, enquanto que sua multiplicidade geométrica é 1 (o autoespaço associado tem dimensão 1). Logo, $A$ não é diagonalizável.
\end{enumerate}



\Exercise[title={2,5}] Considere a matriz $A =
\begin{bmatrix}
-3 & 4 & -8\\
-2 & 3 & -4\\
 0 & 0 &  1
\end{bmatrix}$ e obtenha:
\begin{enumerate}
\item (0,5) O polinômio característico de $A$.
\item (1,0) Uma matriz $P$ que diagonaliza $A$.
\item (0,5) Uma matriz diagonal semelhante à matriz $A$.
\item (0,5) A matriz $A^{123}$, utilizando os dados anteriores.
\end{enumerate}
\Answer
\begin{enumerate}
\item $p_A(x) = \det{(xI - A)} =
\begin{vmatrix}
x+3 &  -4 & 8 \\
  2 & x-3 & 4 \\
  0 &   0 & x-1
\end{vmatrix}
=(x-1)
\begin{vmatrix}
x+3 &  -4\\
  2 & x-3
\end{vmatrix}
=(x-1)[(x+3)(x-3) + 8]
=(x-1)[x^2 - 1]
= (x-1)^2(x+1)
= x^3-x^2-x+1$.
\item As colunas de $P$ são formadas pelas coordenadas dos autovetores de $A$. Como os autovalores de $A$ são $c_1 = c_2 = 1$ e $c_3 = -1$, os autovetores correspondentes são as soluções dos sistemas lineares homogêneos $(I-A)X = 0$ e $(-I-A)X = 0$, respectivamente, ou seja:
\begin{align*}
\begin{bmatrix}
4 & -4 & 8\\
2 & -2 & 4\\
0 &  0 & 0
\end{bmatrix}
\begin{bmatrix}
x\\
y\\
z
\end{bmatrix}
=
\begin{bmatrix}
0\\
0\\
0
\end{bmatrix}
\Leftrightarrow
\begin{cases}
4x-4y+8z&=0\\
2x-2y+4z&=0\\
0&=0
\end{cases}
\Leftrightarrow
\begin{bmatrix}
x\\
y\\
z
\end{bmatrix}
=
\begin{bmatrix}
y-2z\\
y\\
z
\end{bmatrix}
=
y
\begin{bmatrix}
1\\
1\\
0
\end{bmatrix}
+z
\begin{bmatrix}
-2\\
 0\\
 1
\end{bmatrix}.
\end{align*}
e
\begin{align*}
\begin{bmatrix}
2 & -4 &  8\\
2 & -4 &  4\\
0 &  0 & -2
\end{bmatrix}
\begin{bmatrix}
x\\
y\\
z
\end{bmatrix}
=
\begin{bmatrix}
0\\
0\\
0
\end{bmatrix}
\Leftrightarrow
\systeme{
2x-4y+8z=0,
2x-4y+4z=0,
     -2z=0
}
\Leftrightarrow
\begin{bmatrix}
x\\
y\\
z
\end{bmatrix}
=
\begin{bmatrix}
2y\\
y\\
0
\end{bmatrix}
=
y
\begin{bmatrix}
2\\
1\\
0
\end{bmatrix}.
\end{align*}
Assim, $P =
\begin{bmatrix}
1 & -2 & 2\\
1 &  0 & 1\\
0 &  1 & 0
\end{bmatrix}$ é uma matriz que diagonaliza $A$.

\textbf{Observação}: Poderiam ser obtidas outras soluções colocando $y$ em função de $x$ no primeiro sistema, ou ainda, fazendo uma escolha diferente para a ordem dos autovetores (e permutando as colunas de $P$ adequadamente), ou até mesmo escolhendo múltiplos (não nulos) destes autovetores.
\item Como a matriz $P$ do item anterior diagonaliza a matriz $A$, pode-se obter a matriz $D$ por meio da relação $D = P^{-1} A P$. A matriz $P^{-1}$ é obtida escalonando a matriz $[P | I]$, assim:
\begin{align*}
\begin{bmatrix}
1 & -2 & 2 & 1 & 0 & 0\\
1 &  0 & 1 & 0 & 1 & 0\\
0 &  1 & 0 & 0 & 0 & 1
\end{bmatrix}
& \rightarrow
\begin{bmatrix}
1 &  0 & 1 & 0 & 1 & 0\\
0 &  1 & 0 & 0 & 0 & 1\\
1 & -2 & 2 & 1 & 0 & 0
\end{bmatrix}
\rightarrow
\begin{bmatrix}
1 &  0 & 1 & 0 &  1 & 0\\
0 &  1 & 0 & 0 &  0 & 1\\
0 & -2 & 1 & 1 & -1 & 0
\end{bmatrix}\\
&\rightarrow
\begin{bmatrix}
1 & 0 & 1 & 0 &  1 & 0\\
0 & 1 & 0 & 0 &  0 & 1\\
0 & 0 & 1 & 1 & -1 & 2
\end{bmatrix}
\rightarrow
\begin{bmatrix}
1 & 0 & 0 & -1 &  2 & -2\\
0 & 1 & 0 &  0 &  0 &  1\\
0 & 0 & 1 &  1 & -1 &  2
\end{bmatrix}
\end{align*}
Portanto $P^{-1} =
\begin{bmatrix}
-1 &  2 & -2\\
 0 &  0 &  1\\
 1 & -1 &  2
\end{bmatrix}$
e $A$ é semelhante à matriz $D$ dada por
\[
D
=
\begin{bmatrix}
-1 &  2 & -2\\
 0 &  0 &  1\\
 1 & -1 &  2
\end{bmatrix}
\begin{bmatrix}
-3 & 4 & -8\\
-2 & 3 & -4\\
 0 & 0 &  1
\end{bmatrix}
\begin{bmatrix}
1 & -2 & 2\\
1 &  0 & 1\\
0 &  1 & 0
\end{bmatrix}
=
\begin{bmatrix}
1 & 0 & 0\\
0 & 1 & 0\\
0 & 0 & -1
\end{bmatrix}
\]
Note que a diagonal de $D$ é formada justamente pelos autovalores de $A$.
\item Como $D = P^{-1} A P$, conclui-se que $A = P D P^{-1}$ e que $A^{123}= P D^{123} P^{-1}$. Mas
\[
D^{123}
=
\begin{bmatrix}
1^{123} & 0 & 0\\
0 & 1^{123} & 0\\
0 & 0 & (-1)^{123}
\end{bmatrix}
=
\begin{bmatrix}
1 & 0 & 0\\
0 & 1 & 0\\
0 & 0 & -1
\end{bmatrix}
= D,
\]
então $A^{123} = P D^{123} P^{-1} = P D P^{-1} = A =
\begin{bmatrix}
-3 & 4 & -8\\
-2 & 3 & -4\\
 0 & 0 &  1
\end{bmatrix}$.
\end{enumerate}


\Exercise[title={2,5}] Identifique e justifique as afirmações a seguir que forem verdadeiras. Para as demais, forneça um simples contraexemplo:
\begin{enumerate}
\item (1,2) Toda matriz $n \times n$ possui $n$ autovalores distintos.
\item (1,3) Nos espaços com produto interno, sempre que um vetor $u$ é ortogonal aos vetores $v$ e $w$, pode-se concluir que $u$ também é ortogonal a qualquer combinação linear $\alpha v + \beta w$.
\end{enumerate}
\Answer
\begin{enumerate}
\item \textbf{Falsa}. Por exemplo, a matriz $A = \begin{bmatrix}
3 & 0 \\ 0 & 3
\end{bmatrix}$ só tem um autovalor (igual a $3$).
\item \textbf{Verdadeira}. Se $u$ é ortogonal a $v$ e a $w$, então $\langle u, v \rangle = 0 = \langle u, w \rangle$. Consequentemente, pelas propriedades do produto interno,
\[
  \langle u, \alpha v + \beta w \rangle
= \langle u, \alpha v + \beta w \rangle
= \langle u, \alpha v \rangle + \langle \beta w \rangle
= \alpha \langle u, v \rangle + \beta \langle u,w \rangle
= \alpha 0 + \beta 0
= 0.
\]
ou seja, $u$ é ortogonal a qualquer combinação linear $\alpha u + \beta w$.
\end{enumerate}


\Exercise[title={2,5}] Nos itens a seguir, considere o espaço vetorial $\R^2$ com o produto interno definido por $\langle (a, b), (c, d) \rangle = 17ac + 3ad + 3bc + 9bd$. Utilizando esta definição:
\begin{enumerate}
\item (0,5) Calcule a norma do vetor $u=(-1,3)$.
\item (1,0) Obtenha o complemento ortogonal de $W = \ger{ (-1,3) }$.
\item (1,0) Calcule as projeções de $v = (-5,5)$ sobre $W$ e sobre $W^\perp$.
\end{enumerate}
\Answer
\begin{enumerate}
\item Usando o produto interno dado, tem-se
\[
\norm{(a,b)}
= \sqrt{ \langle (a, b), (a, b) \rangle }
= \sqrt{ 17 a^2 + 6ab + 9b^2 }.
\]
Em particular,
\[
\norm{(-1,3)}
= \sqrt{ 17 (-1)^2 + 6 \cdot (-1) \cdot 3 + 9 \cdot 3^2 }
= \sqrt{ 17 - 18 + 81 }
= \sqrt{80}.
\]
\item
Um vetor $v = (x,y)$ pertence a $W^\perp$ se, e somente se, $\langle v, w\rangle = 0$ para todo $w \in W$. Em particular, deve ocorrer que
\[
\langle (x, y), (-1, 3) \rangle = -17x + 9x -3y + 27y = -8x+24y = 0,
\]
isto é, $x=24y/8 = 3y$ e $v = (3y,y) = y(3,1)$. Portanto, $W^\perp = \ger{ (3,1) }$.
\item As projeções de $v = (-5,5)$ sobre $W$ e sobre $W^\perp$ são, respectivamente:
\[
\proj_W v
= \frac{\langle (-5,5), (-1,3) \rangle}{\norm{(-1,3)}^2}(-1,3)
= \frac{160}{80}(-1,3)
= (-2,6)
\]
e
\[
\proj_{W^\perp} v
= v - \proj_W v
= (-5,5) - (-2,6)
= (-3,-1).
\]
\end{enumerate}

\textbf{Observação}: o vetor $\proj_{W^\perp} v$ também poderia ser calculado usando a base do item anterior:
\[
\proj_{W^\perp} v
= \frac{\langle (-5,5), (3,1) \rangle}{\norm{(3,1)}^2}(3,1)
= \frac{-180}{180}(3,1)
= (-3,-1).
\]


\Exercise[title={2,5}] Considere o espaço $\R^4$ com o produto interno usual e seja $W$ o subespaço vetorial gerado pela base $B = \{ v_1, v_2, v_3\}$, em que $v_1 = (0,3,0,0)$, $v_2 = (1,0,0,1)$ e $v_3 = (0,-4,3,0)$. Obtenha:
\begin{enumerate}
\item (1,0) Uma base para o complemento ortogonal $W^\perp$ do subespaço $W$.
\item (1,5) Uma base ortonormal para $W$.
\end{enumerate}
\Answer

\begin{enumerate}
\item Todo vetor $u = (x,y,z,w) \in W^\perp$ é ortogonal aos elementos de $B$. Mas
\begin{align*}
\begin{cases}
\langle u, v_1\rangle = 0\\
\langle u, v_2\rangle = 0\\
\langle u, v_3\rangle = 0
\end{cases}
\Leftrightarrow
\begin{cases}
\langle (x,y,z,w), (0,3,0,0)\rangle = 0\\
\langle (x,y,z,w), (1,0,0,1)\rangle = 0\\
\langle (x,y,z,w), (0,-4,3,0)\rangle = 0
\end{cases}
\Leftrightarrow
\begin{cases}
3y = 0\\
x + w = 0\\
- 4y + 3z = 0
\end{cases}
\Leftrightarrow
\begin{cases}
y = 0 \\
x = - w\\
z = 0
\end{cases}
\end{align*}

Assim, $v = (-w,0,0,w) = w(-1,0,0,1)$ e $B_1 = \{ (-1,0,0,1) \}$ é uma base de $W^\perp$.

\item Observe que $v_1$ não é ortogonal a $v_3$ pois $\langle v_1, v_3 \rangle = -12 \neq 0$. Para obter uma base ortogonal para o mesmo espaço, pode-se utilizar o processo de ortogonalização de Gram–Schmidt:
\begin{itemize}
\item $w_1
= v_1
= (0,3,0,0)$
\item $w_2
= v_2 - \frac{\langle v_2, w_1 \rangle}{\norm{w_1}^2}w_1
= (1,0,0,1) - \frac{0}{9}(0,3,0,0)
= (1,0,0,1)$
\item $w_2
= v_3 - \frac{\langle v_3, w_1 \rangle}{\norm{w_1}^2}w_1
      - \frac{\langle v_3, w_2 \rangle}{\norm{w_2}^2}w_2
= (0,-4,3,0) - \frac{-12}{9}(0,3,0,0)
             - \frac{0}{2}(1,0,0,1)
= (0,0,3,0)$
\end{itemize}
O conjunto $C = \{ w_1, w_2, w_3 \} = \{ (0,3,0,0), (1,0,0,1), (0,0,3,0) \}$ é uma base de $W$, mas como seus vetores não têm norma um, é preciso normaliza-los para obter a base ortonormal $D = \left\{ \frac{w_1}{\norm{w_1}}, \frac{w_2}{\norm{w_2}}, \frac{w_3}{\norm{w_3}} \right\} = \left\{ (0,1,0,0), (\frac{\sqrt{2}}{2},0,0,\frac{\sqrt{2}}{2}), (0,0,1,0) \right\}$.
\end{enumerate}
\end{ExerciseList}

\begin{center}
BOA PROVA!
\end{center}

\newpage
\restoregeometry
\section*{Respostas}
\shipoutAnswer
\end{document}
