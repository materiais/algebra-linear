\documentclass[12pt,a4paper]{article}
\usepackage{cmap} % Makes the PDF copiable. See http://tex.stackexchange.com/a/64198/25761
\usepackage[T1]{fontenc}
\usepackage[brazil]{babel}
\usepackage[utf8]{inputenc}
\usepackage{amsmath}
\usepackage{amsfonts}
\usepackage{amssymb}
\usepackage{amsthm}
\usepackage{textcomp} % \degree
\usepackage{gensymb} % \degree
\usepackage[usenames,svgnames,dvipsnames]{xcolor}
\usepackage{hyperref}
\usepackage{multicol}
\usepackage{graphicx}
\usepackage[margin=2cm]{geometry}
\usepackage{systeme}

\hypersetup{
    colorlinks = true,
    allcolors = {blue}
}

% TODO: Consider using exsheets
% http://linorg.usp.br/CTAN/macros/latex/contrib/exsheets/exsheets_en.pdf
%
% http://ctan.org/tex-archive/macros/latex/contrib/exercise/
% Options: answerdelayed,lastexercise,noanswer
\usepackage[answerdelayed,lastexercise]{exercise}

\addto\captionsbrazil{%
\def\listexercisename{Lista de exerc\'icios}%
\def\ExerciseName{Exerc\'icio}%
\def\AnswerName{Solu\c{c}\~ao do exerc\'icio}%
\def\ExerciseListName{Ex.}%
\def\AnswerListName{Solu\c{c}\~ao}%
\def\ExePartName{Parte}%
\def\ArticleOf{de\ }%
}

\renewcommand{\ExerciseHeaderTitle}{(\ExerciseTitle)\ }
\renewcommand{\ExerciseListHeader}{%\ExerciseHeaderDifficulty%
\textbf{%\ExerciseListName\
\ExerciseHeaderNB.\ %
%\ --- \
\ExerciseHeaderTitle}%
%\ExerciseHeaderOrigin
\ignorespaces}
\renewcommand{\AnswerListHeader}{\textbf{\ExerciseHeaderNB.\ (\AnswerListName)\ }}

\newtheorem*{note}{Observação}
\newcommand*\sen{\operatorname{sen}}
\newcommand*\R{\mathbb{R}}

% Loop Space / CC BY-SA-3.0 / https://tex.stackexchange.com/a/2238/25761
\newenvironment{amatrix}[1]{%
  \left[\begin{array}{@{}*{#1}{c}|c@{}}
}{%
  \end{array}\right]
}

% Loop Space / CC BY-SA-3.0 / https://tex.stackexchange.com/a/3164/25761
%--------grstep
% For denoting a Gauss' reduction step.
% Use as: \grstep{\rho_1+\rho_3} or \grstep[2\rho_5 \\ 3\rho_6]{\rho_1+\rho_3}
\newcommand{\grstep}[2][\relax]{%
   \ensuremath{\mathrel{
       {\mathop{\longrightarrow}\limits^{#2\mathstrut}_{
                                     \begin{subarray}{l} #1 \end{subarray}}}}}}
\newcommand{\swap}{\leftrightarrow}

\renewcommand{\theenumi}{\alph{enumi}}
\renewcommand\labelenumi{(\theenumi) }

\newcommand*\tipo{Prova I}
\newcommand*\turma{CIV122-02U}
\newcommand*\disciplina{ALI0001}
\newcommand*\eu{Helder G. G. de Lima}
\newcommand*\data{30/08/2016}

\author{\eu}
\title{\tipo - \disciplina}
\date{\data}

\begin{document}
\thispagestyle{empty}
\newgeometry{margin=2cm,bottom=0.5cm}
\begin{center}
\includegraphics[width=9.0cm]{marca} \\
\textbf{\tipo\ (\disciplina / \turma)} \\
Prof. \eu\footnote{
Este é um material de acesso livre distribuído sob os termos da licença \href{https://creativecommons.org/licenses/by-sa/4.0/deed.pt_BR}{Creative Commons BY-SA 4.0}}
\end{center}

\noindent Nome do(a) aluno(a): \underline{\hspace{9,7cm}} Data: \underline{\data}

%\section*{Instruções}
\begin{center}\fbox{
\begin{minipage}{14cm}

{\footnotesize
\begin{itemize}
\renewcommand{\theenumi}{\Roman{enumi}}
\item Identifique-se em todas as folhas.
\item Mantenha o celular e os demais equipamentos eletrônicos desligados durante a prova.
\item Anule \textsc{\textbf{uma}} das 6 questões (apenas 5 serão corrigidas): \framebox(30,10){}
\end{itemize}
}

\end{minipage}
}
\end{center}

%\section*{Questões}
\begin{ExerciseList}
\Exercise[title={1,8}]
Para $A$ e $B$ inversíveis, prove o que for verdadeiro e dê contraexemplos para as demais:

\begin{multicols}{2}
\begin{enumerate}
\item $AB$ é inversível
\item $A+B$ é inversível
\item $A^T$ é inversível
\item $A$ não pode ser antissimétrica
\end{enumerate}
\end{multicols}
\Answer
\begin{enumerate}
\item \textbf{Verdadeiro}. Se $A$ e $B$ são quaisquer matrizes inversíveis então
\[(AB)(B^{-1}A^{-1}) = A A^{-1} = I,\]
o que significa que $AB$ é inversível, e sua inversa é $B^{-1}A^{-1}$.
\item \textbf{Falso}. Se $A$ é qualquer matriz inversível (por exemplo $A=I$), e $B = -A$ então $B$ também é inversível e $A + B = A+(-A) = 0$, que não é uma matriz inversível.
\item \textbf{Verdadeiro}. Se $M$ é uma matriz inversível, então
\[
M^T \cdot (M^{-1})^T
= (M^{-1} \cdot M)^T
= I^T
= I.
\]
Logo, $M^T$ também é inversível, e tem inversa $(A^{-1})^T$.
\item \textbf{Falso}. A matriz $A = \begin{bmatrix}
0 & 3\\
-3 & 0
\end{bmatrix}$ é antissimétrica e inversível, e sua inversa também é antissimétrica: $A^{-1} = \begin{bmatrix}
0 & -1/3 \\1/3 & 0
\end{bmatrix}$.
\end{enumerate}



\Exercise[title={1,8}] Seja $D=\begin{bmatrix}
3 & 0\\
0 & 2\\
\end{bmatrix}$ e $Q=\begin{bmatrix}
3 & 1\\
5 & 2
\end{bmatrix}$. Calcule e verifique se são simétricas as matrizes:
\begin{multicols}{2}
\begin{enumerate}
\item $A = Q D Q^{-1}$
\item $B = Q D^2 - A^2 Q$
\end{enumerate}
\end{multicols}
\Answer Primeiramente, deve ser encontrada a inversa de $Q$, e isso pode ser feito usando a eliminação de Gauss-Jordan:
\begin{align*}
\begin{bmatrix}
3 & 1 & 1 & 0 \\
5 & 2 & 0 & 1
\end{bmatrix}
& \grstep{ \frac{1}{3}L_1 }
\begin{bmatrix}
1 & 1/3 & 1/3 & 0 \\
5 & 2 & 0 & 1
\end{bmatrix}
\grstep{ L_2 - 5 L_1 }
\begin{bmatrix}
1 & 1/3 & 1/3 & 0 \\
0 & 1/3 & -5/3 & 1
\end{bmatrix} \\
&
\grstep{ 3L_2 }
\begin{bmatrix}
1 & 1/3 & 1/3 & 0 \\
0 & 1 & -5 & 3
\end{bmatrix}
\grstep{ L_1-\frac{1}{3} L_2 }
\begin{bmatrix}
1 & 0 & 2 & -1 \\
0 & 1 & -5 & 3
\end{bmatrix}
\Rightarrow
Q^{-1}=
\begin{bmatrix}
2 & -1 \\
-5 & 3
\end{bmatrix}
\end{align*}
No entanto, no caso $2 \times 2$ é mais rápido usar a fórmula: $\displaystyle \begin{bmatrix}
a & b\\ c & d
\end{bmatrix}^{-1} = \frac{1}{ad-bc} \begin{bmatrix}
d & -b\\ -c & a
\end{bmatrix}$.

\begin{enumerate}
\item $A
= QDQ^{-1}
=
\begin{bmatrix}
3 & 1\\
5 & 2
\end{bmatrix}
\begin{bmatrix}
3 & 0\\
0 & 2
\end{bmatrix}
\begin{bmatrix}
2 & -1 \\
-5 & 3
\end{bmatrix}
=
\begin{bmatrix}
9 & 2\\
15 & 4
\end{bmatrix}
\begin{bmatrix}
2 & -1 \\
-5 & 3
\end{bmatrix}
=
\begin{bmatrix}
8 & -3 \\
10 & -3
\end{bmatrix}
$, que não é uma matriz simétrica.
\item Usando a definição de $A$, e as propriedades das operações com matrizes, tem-se:
\begin{align*}
Q D^2 - A^2 Q
& = Q D^2 - (QDQ^{-1})^2 Q
  = Q D^2 - (QDQ^{-1})(QDQ^{-1}) Q \\
& = Q D^2 - QD(Q^{-1}Q)D(Q^{-1}Q)
  = Q D^2 - QDIDI
  = Q D^2 - QD^2
  = 0
\end{align*}
O mesmo resultado pode ser obtido usando os valores numéricos de $A$, $Q$ e $D$, e fazendo um pouco mais de cálculo:
\begin{align*}
Q D^2 - A^2 Q
& =
\begin{bmatrix}
3 & 1 \\
5 & 2
\end{bmatrix}
\begin{bmatrix}
3 & 0 \\
0 & 2
\end{bmatrix}^2
-
\begin{bmatrix}
 8 & -3 \\
10 & -3
\end{bmatrix}^2
\begin{bmatrix}
3 & 1 \\
5 & 2
\end{bmatrix}
 =
\begin{bmatrix}
3 & 1 \\
5 & 2
\end{bmatrix}
\begin{bmatrix}
9 & 0 \\
0 & 4
\end{bmatrix}
-
\begin{bmatrix}
34 & -15 \\
50 & -21
\end{bmatrix}
\begin{bmatrix}
3 & 1 \\
5 & 2
\end{bmatrix} \\
& =
\begin{bmatrix}
27 & 4 \\
45 & 8
\end{bmatrix}
-
\begin{bmatrix}
27 & 4 \\
45 & 8
\end{bmatrix}
=
\begin{bmatrix}
0 & 0 \\
0 & 0
\end{bmatrix}
\end{align*}
Portanto, $Q D^2 - A^2 Q = 0$, que é uma matriz simétrica.
\end{enumerate}

\Exercise[title={1,8}] Determine quais das seguintes matrizes $A = (a_{ij}) \in M_{3 \times 3} (\R)$ são simétricas ($A^T = A$), e quais são antissimétricas ($A^T = -A$). As entradas $a_{ij}$ de cada matriz são dadas por:
\begin{multicols}{3}
\begin{enumerate}
\item $a_{ij} = i + 6 (i - j)^3 - j$
\item $a_{ij} = i^3 - 3ij + 1 + j^3$
\item $a_{ij} = i(j+1)$
\end{enumerate}
\end{multicols}
\Answer
\begin{enumerate}
\item Se $a_{ij} = i + 6 (i - j)^3 - j$, então
 $A = \begin{bmatrix}
0 & -7 & -50 \\
7 & 0 & -7\\
50 & 7 & 0
\end{bmatrix} = -A^T$, ou seja, $A$ é antissimétrica.
\item Se $a_{ij} = i^3 - 3ij + 1 + j^3$, então
$A = \begin{bmatrix}
0 & 4 & 20\\
4 & 5 & 18\\
20 & 18 & 28
\end{bmatrix} = A^T$, ou seja, $A$ é simétrica.
\item Se $a_{ij} = i(j+1)$, então
$A = \begin{bmatrix}
2 & 3 & 4 \\
4 & 6 & 8 \\
6 & 9 & 12
\end{bmatrix}$, ou seja, $A$ não é simétrica nem antissimétrica.
\end{enumerate}


\Exercise[title={1,8}] Obtenha os conjuntos de soluções dos seguintes sistemas lineares homogêneos, sendo $0$ a matriz nula $4 \times 1$, $I$ a matriz identidade $4 \times 4$ e $P = \begin{bmatrix}
 3 &  4 &  2 & 1\\
 0 &  1 &  0 & 0\\
 0 &  0 &  1 & 0\\
-2 & -4 & -2 & 0
\end{bmatrix}$.
\begin{multicols}{2}
\begin{enumerate}
\item $(P - I)X = 0$
\item $(P - 2I)X = 0$
\end{enumerate}
\end{multicols}
\Answer Observe que $P - I = \begin{bmatrix}
 2 &  4 &  2 & 1\\
 0 &  0 &  0 & 0\\
 0 &  0 &  0 & 0\\
-2 & -4 & -2 & -1
\end{bmatrix}$ e $P - 2I = \begin{bmatrix}
 1 &  4 &  2 & 1\\
 0 & -1 &  0 & 0\\
 0 &  0 & -1 & 0\\
-2 & -4 & -2 & -1
\end{bmatrix}$.

Para resolver os sistemas lineares homogêneos associados a estas matrizes, basta usar a eliminação de Gauss-Jordan:
\begin{enumerate}

\item
\begin{align*}
& \begin{amatrix}{4}
 2 &  4 &  2 & 1 & 0\\
 0 &  0 &  0 & 0 & 0\\
 0 &  0 &  0 & 0 & 0\\
-2 & -4 & -2 & -1 & 0
\end{amatrix}
& \grstep{ \frac{1}{2}L_1 }
\begin{amatrix}{4}
 1 &  2 &  1 & 1/2 & 0\\
 0 &  0 &  0 & 0 & 0\\
 0 &  0 &  0 & 0 & 0\\
-2 & -4 & -2 & -1 & 0
\end{amatrix}
\grstep{ L_2 \swap L_4 }
\begin{amatrix}{4}
 1 &  2 &  1 & 1/2 & 0\\
-2 & -4 & -2 & -1 & 0\\
 0 &  0 &  0 & 0 & 0\\
 0 &  0 &  0 & 0 & 0
\end{amatrix}\\
& \grstep{ L_2 +2 L_1 }
\begin{amatrix}{4}
 1 &  2 &  1 & 1/2 & 0\\
 0 &  0 &  0 & 0 & 0\\
 0 &  0 &  0 & 0 & 0\\
 0 &  0 &  0 & 0 & 0
\end{amatrix}
\end{align*}

Assim, o sistema homogêneo é equivalente a $x_1 + 2 x_2 + x_3 + \frac{1}{2}x_4 = 0$, que é satisfeito por qualquer $(x_1, x_2, x_3, x_4)$ em que $x_1 = -2 x_2 -x_3 - \frac{1}{2}x_4$, isto é, $(-2 x_2 -x_3 - \frac{1}{2}x_4, x_2, x_3, x_4)$, com $x_2,x_3,x_4$ arbitrários.
\item
\begin{align*}
& \begin{amatrix}{4}
 1 &  4 &  2 & 1 & 0\\
 0 & -1 &  0 & 0 & 0\\
 0 &  0 & -1 & 0 & 0\\
-2 & -4 & -2 & -2 & 0
\end{amatrix}
\grstep{ L_4 + 2 L_1 }
\begin{amatrix}{4}
 1 &  4 &  2 & 1 & 0\\
 0 & -1 &  0 & 0 & 0\\
 0 &  0 & -1 & 0 & 0\\
 0 &  4 &  2 & 0 & 0
\end{amatrix}
\grstep{ -L_2 }
\begin{amatrix}{4}
 1 & 4 &  2 & 1 & 0\\
 0 & 1 &  0 & 0 & 0\\
 0 & 0 & -1 & 0 & 0\\
 0 & 4 &  2 & 0 & 0
\end{amatrix}\\
\grstep{ L_4 - 4 L_2 }
& \begin{amatrix}{4}
 1 & 4 &  2 & 1 & 0\\
 0 & 1 &  0 & 0 & 0\\
 0 & 0 & -1 & 0 & 0\\
 0 & 0 &  2 & 0 & 0
\end{amatrix}
\grstep{ -L_3 }
\begin{amatrix}{4}
 1 & 4 & 2 & 1 & 0\\
 0 & 1 & 0 & 0 & 0\\
 0 & 0 & 1 & 0 & 0\\
 0 & 0 & 2 & 0 & 0
\end{amatrix}
\grstep{ L_4-2L_3 }
\begin{amatrix}{4}
 1 & 4 & 2 & 1 & 0\\
 0 & 1 & 0 & 0 & 0\\
 0 & 0 & 1 & 0 & 0\\
 0 & 0 & 0 & 0 & 0
\end{amatrix} \\
\grstep{ L_1-2L_3 }
&\begin{amatrix}{4}
 1 & 4 & 0 & 1 & 0\\
 0 & 1 & 0 & 0 & 0\\
 0 & 0 & 1 & 0 & 0\\
 0 & 0 & 0 & 0 & 0
\end{amatrix}
\grstep{ L_1-4L_2 }
\begin{amatrix}{4}
 1 & 0 & 0 & 1 & 0\\
 0 & 1 & 0 & 0 & 0\\
 0 & 0 & 1 & 0 & 0\\
 0 & 0 & 0 & 0 & 0
\end{amatrix}
\end{align*}

Assim, o sistema homogêneo é equivalente a
\[
\systeme{
x_1 + x_4 = 0,
x_2 = 0,
x_3 = 0
}
\]
que é satisfeito por qualquer $(x_1, x_2, x_3, x_4)$ em que $x_1 = -x_4$ e os demais $x_i$'s são nulos, isto é, $(-x_4, 0, 0, x_4)$, com $x_4$ arbitrário.
\end{enumerate}

\Exercise[title={1,8}] Determine as soluções do sistema linear $\begin{cases}
4x + 3y - z -2 w = 8\\
5x + 4y - z -3 w = 11.
\end{cases}
$
\Answer A eliminação de Gauss-Jordan consiste dos seguintes passos:
\begin{align*}
& \begin{amatrix}{4}
4 & 3 & -1 & -2 &  8\\
5 & 4 & -1 & -3 & 11\\
\end{amatrix}
\grstep{ \frac{1}{4}L_1 }
\begin{amatrix}{4}
1 & 3/4 & -1/4 & -1/2 &  2\\
5 & 4 & -1 & -3 & 11\\
\end{amatrix}
\grstep{ L_2 -5 L_1 }
\begin{amatrix}{4}
1 & 3/4 & -1/4 & -1/2 &  2\\
0 & 1/4 & 1/4 & -1/2 & 1\\
\end{amatrix}\\
&\grstep{ 4L_2 }
\begin{amatrix}{4}
1 & 3/4 & -1/4 & -1/2 &  2\\
0 & 1 & 1 & -2 & 4\\
\end{amatrix}
\grstep{ L_1-\frac{3}{4}L_2 }
\begin{amatrix}{4}
1 & 0 & -1 & 1 & -1\\
0 & 1 & 1 & -2 & 4\\
\end{amatrix}
\end{align*}
Disto conclui-se que o sistema original é equivalente a
\[
\begin{cases}
1x + 0y - z  + w = -1\\
0x + 1y + z -2 w = 4.
\end{cases}
\]
Assim, as soluções do sistema são da forma $(x,y,z,w)$, com $x=-1+z-w$ e $y = 4-z+2w$.

\Exercise[title={1,8}] Uma matriz quadrada $Q$ com entradas em $\R$ será denominada \emph{ortogonal} se $Q^{-1} = Q^T$.
\begin{enumerate}
\item Liste todas as matrizes $Q$ ortogonais de tamanho $1 \times 1$.
\item Verifique que a matriz $R =\begin{bmatrix}
\cos(\theta)& \sen(\theta) \\
-\sen(\theta) & \cos(\theta)
\end{bmatrix}$ é ortogonal, qualquer que seja $\theta \in \R$.
\item Explique porque o produto de matrizes ortogonais também é uma matriz ortogonal.
\end{enumerate}

\Answer
\begin{enumerate}
\item Se $A=\begin{bmatrix}a_{11}\end{bmatrix}$, então $A^{-1} = \begin{bmatrix}1/a_{11}\end{bmatrix}$ e $A^T = A$. Assim,
\[
A \text{ é ortogonal}
\Leftrightarrow A^{-1} = A^T
\Leftrightarrow a_{11} = 1/a_{11}
\Leftrightarrow a_{11}^2 = 1
\Leftrightarrow a_{11} = \pm 1.
\]
Ou seja, as únicas matrizes ortogonais de tamanho $1 \times 1$ são $A=\begin{bmatrix}1\end{bmatrix} = I$ e $A=\begin{bmatrix}-1\end{bmatrix} = -I$.
\item Se $R =\begin{bmatrix}
\cos(\theta)& \sen(\theta) \\
-\sen(\theta) & \cos(\theta)
\end{bmatrix}$
então $R^T =\begin{bmatrix}
\cos(\theta)& -\sen(\theta) \\
\sen(\theta) & \cos(\theta)
\end{bmatrix}$.
Consequentemente,
\[
R \cdot R^T
\begin{bmatrix}
\cos^2(\theta) + \sen^2(\theta) & -\cos(\theta)\sen(\theta) + \sen(\theta)\cos(\theta) \\
- \sen(\theta)\cos(\theta) + \cos(\theta)\sen(\theta) & \cos^2(\theta) + \sen^2(\theta)
\end{bmatrix}
=
\begin{bmatrix}
1 & 0 \\
0 & 1
\end{bmatrix}
= I
\]
Portanto, $R^{-1} = R^T$, ou seja, $R$ é ortogonal.
\item Sejam $A$ e $B$ matrizes $n \times n$ ortogonais. Então $A^{-1} = A^T$ e $B^{-1} = B^T$. Logo,
\[
(AB)^{-1} = B^{-1} A^{-1} = B^T A^T = (AB)^T,
\]
ou seja, $AB$ é ortogonal.

\end{enumerate}
\end{ExerciseList}

\medskip
\begin{note}
As questões anteriores passam a valer 2,0 pontos caso:
\begin{enumerate}
\item \textbf{(0,1) Organize} respostas legíveis para \textbf{tudo} o que foi perguntado
\item \textbf{(0,1) Explique} a resolução por escrito, mostrando onde usou a teoria estudada
\end{enumerate}
\end{note}

\begin{center}
BOA PROVA!
\end{center}

\newpage
\restoregeometry
\section*{Respostas}
\shipoutAnswer
\end{document}
