\documentclass[12pt,a4paper]{article}
\usepackage{cmap} % Makes the PDF copiable. See http://tex.stackexchange.com/a/64198/25761
\usepackage[T1]{fontenc}
\usepackage[brazil]{babel}
\usepackage[utf8]{inputenc}
\usepackage{amsmath}
\usepackage{amsfonts}
\usepackage{amssymb}
\usepackage{amsthm}
\usepackage{textcomp} % \degree
\usepackage{gensymb} % \degree
\usepackage[usenames,svgnames,dvipsnames]{xcolor}
\usepackage{hyperref}
\usepackage{multicol}
\usepackage{graphicx}
\usepackage[margin=2cm]{geometry}
\usepackage{systeme}

\hypersetup{
    colorlinks = true,
    allcolors = {blue}
}

% TODO: Consider using exsheets
% http://linorg.usp.br/CTAN/macros/latex/contrib/exsheets/exsheets_en.pdf
%
% http://ctan.org/tex-archive/macros/latex/contrib/exercise/
% Options: answerdelayed,lastexercise,noanswer
\usepackage[answerdelayed,lastexercise]{exercise}

\addto\captionsbrazil{%
\def\listexercisename{Lista de exerc\'icios}%
\def\ExerciseName{Exerc\'icio}%
\def\AnswerName{Solu\c{c}\~ao do exerc\'icio}%
\def\ExerciseListName{Ex.}%
\def\AnswerListName{Solu\c{c}\~ao}%
\def\ExePartName{Parte}%
\def\ArticleOf{de\ }%
}

\renewcommand{\ExerciseHeaderTitle}{(\ExerciseTitle)\ }
\renewcommand{\ExerciseListHeader}{%\ExerciseHeaderDifficulty%
\textbf{%\ExerciseListName\
\ExerciseHeaderNB.\ %
%\ --- \
\ExerciseHeaderTitle}%
%\ExerciseHeaderOrigin
\ignorespaces}
\renewcommand{\AnswerListHeader}{\textbf{\ExerciseHeaderNB.\ (\AnswerListName)\ }}

\newcommand{\fixme}{{\color{red}(...)}}
\newcommand{\norm}[1]{\left|\left|{#1}\right|\right|}
\newcommand*\R{\mathbb{R}}

% Loop Space / CC BY-SA-3.0 / https://tex.stackexchange.com/a/2238/25761
\newenvironment{amatrix}[1]{%
  \left[\begin{array}{@{}*{#1}{c}|c@{}}
}{%
  \end{array}\right]
}

% Loop Space / CC BY-SA-3.0 / https://tex.stackexchange.com/a/3164/25761
%--------grstep
% For denoting a Gauss' reduction step.
% Use as: \grstep{\rho_1+\rho_3} or \grstep[2\rho_5 \\ 3\rho_6]{\rho_1+\rho_3}
\newcommand{\grstep}[2][\relax]{%
   \ensuremath{\mathrel{
       {\mathop{\longrightarrow}\limits^{#2\mathstrut}_{
                                     \begin{subarray}{l} #1 \end{subarray}}}}}}

\renewcommand{\theenumi}{\alph{enumi}}
\renewcommand\labelenumi{(\theenumi) }

\newcommand*\tipo{Prova IV}
\newcommand*\turma{MAT102-03U/LEF102-02U}
\newcommand*\disciplina{ALI0001}
\newcommand*\eu{Helder G. G. de Lima}
\newcommand*\data{28/06/2016}

\author{\eu}
\title{\tipo - \disciplina}
\date{\data}

\begin{document}
\thispagestyle{empty}
\newgeometry{margin=2cm,bottom=0.5cm}
\begin{center}
\includegraphics[width=9.0cm]{marca} \\
\textbf{\tipo\ (\disciplina / \turma)} \\
Prof. \eu\footnote{
Este é um material de acesso livre distribuído sob os termos da licença \href{https://creativecommons.org/licenses/by-sa/4.0/deed.pt_BR}{Creative Commons Atribuição-CompartilhaIgual 4.0 Internacional}}
\end{center}

\noindent Nome do(a) aluno(a): \underline{\hspace{9,7cm}} Data: \underline{\data}

%\section*{Instruções}
\begin{center}\fbox{
\begin{minipage}{14cm}

{\footnotesize
\begin{itemize}
\renewcommand{\theenumi}{\Roman{enumi}}
\item Identifique-se em todas as folhas.
\item Mantenha o celular e os demais equipamentos eletrônicos desligados durante a prova.
\item Anule \textsc{\textbf{uma}} das 5 questões (apenas 4 serão corrigidas): \framebox(30,10){}
\end{itemize}
}

\end{minipage}
}
\end{center}

%\section*{Questões}
\begin{ExerciseList}
\Exercise[title={2,5}] Seja $L: M_{2\times 2} \to M_{2\times 2}$ o operador linear definido por $L(X) = X + X^T$.
\begin{enumerate}
\item (1,5) Obtenha os autovalores e autovetores de $L$.
\item (1,0) A matriz do operador $L$ em relação à base canônica é diagonalizável? Explique.
\end{enumerate}
\Answer \fixme


\Exercise[title={2,5}] Considere a matriz $A =
\begin{bmatrix}
 3 &  0 &  4\\
 3 &  0 &  3\\
-2 &  0 & -3
\end{bmatrix}$ e obtenha:
\begin{enumerate}
\item (0,5) O polinômio característico de $A$.
\item (1,0) Uma matriz $P$ que diagonaliza $A$.
\item (0,5) Uma matriz diagonal semelhante à matriz $A$.
\item (0,5) A matriz $A^{100}$, utilizando os dados anteriores.
\end{enumerate}
\Answer \fixme


\Exercise[title={2,5}] Identifique e justifique as afirmações a seguir que forem verdadeiras. Para as demais, forneça um simples contra-exemplo:
\begin{enumerate}
\item (1,3) Nos espaços com produto interno, quaisquer vetores ortogonais $u$ e $v$, não nulos, são linearmente independentes.
\item (1,2) Sempre que $A$ e $B$ são matrizes semelhantes pode-se dizer que $A^2$ é semelhante a $B^2$.
\end{enumerate}
\Answer
\begin{enumerate}
\item \textbf{Verdadeira}. Se os vetores $u$ e $v$ fossem linearmente dependentes, um deles seria múltiplo do outro. Em outras palavras, existiria, por exemplo, algum $\alpha \in \R$ tal que $u = \alpha v$. Com isso, resultaria da ortogonalidade entre $u$ e $v$ que:
\[
0
= \langle u,v \rangle
= \langle \alpha v, v \rangle
= \alpha \langle v, v \rangle
= \alpha \norm{v}^2.
\]
Mas isso implica que $\alpha = 0$ ou $\norm{v} = 0$, o que é impossível para vetores não nulos $u$ e $v$. Logo, $u$ e $v$ só podem ser linearmente independentes.

\item \textbf{Verdadeira} Se $A$ é semelhante a $B$ então existe alguma matriz inversível $P$ tal que $A = P^{-1} B P$. Consequentemente,
\[
A^2 = (P^{-1} B P)(P^{-1} B P)
    = P^{-1} B (PP^{-1}) B P
    = P^{-1} B I B P
    = P^{-1} B^2 P,
\]
isto é, $A^2$ e $B^2$ são semelhantes.
\end{enumerate}


\Exercise[title={2,5}] Seja
$B = \{ (-4, 1, 0), (1, 2, 0), (0, 0, 6) \}$
uma base de $\R^3$ e considere o produto interno definido por $\langle (a, b, c), (x,y,z) \rangle = \frac{1}{9}(ax+2by+cz)$.
\begin{enumerate}
\item (1,0) Verifique que, em relação ao produto interno acima, a base $B$ é ortogonal.
\item (1,0) Use o produto interno acima para escrever o vetor $v = (-5, 8, 6)$ como combinação linear dos vetores de $B$.
\item (0,5) Obtenha uma base ortonormal de $\R^3$ a partir de $B$.
\end{enumerate}
\Answer \fixme


\Exercise[title={2,5}] Considere o espaço $\R^4$ com o produto interno usual e seja $W$ o subespaço vetorial gerado pela base $B = \{ v_1, v_2, v_3\}$, em que $v_1 = (0,3,0,0)$, $v_2 = (0,-4,0,3)$ e $v_3 = (1,1,1,0)$. Obtenha:
\begin{enumerate}
\item (1,0) Uma base para o complemento ortogonal $W^\perp$ do subespaço $W$.
\item (1,5) Uma base ortonormal para $W$.
\end{enumerate}
\Answer \fixme
\end{ExerciseList}

\begin{center}
BOA PROVA!
\end{center}

\newpage
\restoregeometry
\section*{Respostas}
\shipoutAnswer
\end{document}
