\documentclass[12pt,a4paper]{article}
\usepackage{cmap} % Makes the PDF copiable. See http://tex.stackexchange.com/a/64198/25761
\usepackage[T1]{fontenc}
\usepackage[brazil]{babel}
\usepackage[utf8]{inputenc}
\usepackage{amsmath}
\usepackage{amsfonts}
\usepackage{amssymb}
\usepackage{amsthm}
\usepackage{textcomp} % \degree
\usepackage{gensymb} % \degree
\usepackage[usenames,svgnames,dvipsnames]{xcolor}
\usepackage{hyperref}
\usepackage{multicol}
\usepackage{graphicx}
\usepackage[margin=2cm]{geometry}
\usepackage{systeme}

\hypersetup{
    colorlinks = true,
    allcolors = {blue}
}

% TODO: Consider using exsheets
% http://linorg.usp.br/CTAN/macros/latex/contrib/exsheets/exsheets_en.pdf
%
% http://ctan.org/tex-archive/macros/latex/contrib/exercise/
% Options: answerdelayed,lastexercise,noanswer
\usepackage[answerdelayed,lastexercise]{exercise}

\addto\captionsbrazil{%
\def\listexercisename{Lista de exerc\'icios}%
\def\ExerciseName{Exerc\'icio}%
\def\AnswerName{Solu\c{c}\~ao do exerc\'icio}%
\def\ExerciseListName{Ex.}%
\def\AnswerListName{Solu\c{c}\~ao}%
\def\ExePartName{Parte}%
\def\ArticleOf{de\ }%
}

\renewcommand{\ExerciseHeaderTitle}{(\ExerciseTitle)\ }
\renewcommand{\ExerciseListHeader}{%\ExerciseHeaderDifficulty%
\textbf{%\ExerciseListName\
\ExerciseHeaderNB.\ %
%\ --- \
\ExerciseHeaderTitle}%
%\ExerciseHeaderOrigin
\ignorespaces}
\renewcommand{\AnswerListHeader}{\textbf{\ExerciseHeaderNB.\ (\AnswerListName)\ }}

\newcommand*\R{\mathbb{R}}

% Loop Space / CC BY-SA-3.0 / https://tex.stackexchange.com/a/2238/25761
\newenvironment{amatrix}[1]{%
  \left[\begin{array}{@{}*{#1}{c}|c@{}}
}{%
  \end{array}\right]
}

% Loop Space / CC BY-SA-3.0 / https://tex.stackexchange.com/a/3164/25761
%--------grstep
% For denoting a Gauss' reduction step.
% Use as: \grstep{\rho_1+\rho_3} or \grstep[2\rho_5 \\ 3\rho_6]{\rho_1+\rho_3}
\newcommand{\grstep}[2][\relax]{%
   \ensuremath{\mathrel{
       {\mathop{\longrightarrow}\limits^{#2\mathstrut}_{
                                     \begin{subarray}{l} #1 \end{subarray}}}}}}
\newcommand{\swap}{\leftrightarrow}

\renewcommand{\theenumi}{\alph{enumi}}
\renewcommand\labelenumi{(\theenumi) }

\newcommand*\tipo{Prova I}
\newcommand*\turma{NEX162-C}
\newcommand*\disciplina{ALI0001}
\newcommand*\eu{Helder G. G. de Lima}
\newcommand*\data{30/08/2016}

\author{\eu}
\title{\tipo - \disciplina}
\date{\data}

\begin{document}
\thispagestyle{empty}
\newgeometry{margin=2cm,bottom=0.5cm}
\begin{center}
\includegraphics[width=9.0cm]{marca} \\
\textbf{\tipo\ (\disciplina / \turma)} \\
Prof. \eu\footnote{
Este é um material de acesso livre distribuído sob os termos da licença \href{https://creativecommons.org/licenses/by-sa/4.0/deed.pt_BR}{Creative Commons Atribuição-CompartilhaIgual 4.0 Internacional}}
\end{center}

\noindent Nome do(a) aluno(a): \underline{\hspace{9,7cm}} Data: \underline{\data}

%\section*{Instruções}
\begin{center}\fbox{
\begin{minipage}{14cm}

{\footnotesize
\begin{itemize}
\renewcommand{\theenumi}{\Roman{enumi}}
\item Identifique-se em todas as folhas.
\item Mantenha o celular e os demais equipamentos eletrônicos desligados durante a prova.
\item Anule \textsc{\textbf{uma}} das 6 questões (apenas 5 serão corrigidas): \framebox(30,10){}
\end{itemize}
}

\end{minipage}
}
\end{center}

\section*{Questões}
\begin{ExerciseList}

\Exercise[title={2,0}] Prove (sem dar exemplo) as afirmações verdadeiras e exemplifique o erro nas demais:
\begin{enumerate}
\item Se uma matriz não é simétrica, então ela é antissimétrica.
\item A soma de matrizes inversíveis é uma matriz inversível.
\item Se uma matriz triangular inferior é antissimétrica, então ela é a matriz nula,
\end{enumerate}
\Answer
\begin{enumerate}
\item \textbf{Falso}. Por exemplo, a matriz $A=
\begin{bmatrix}
0 & 2 \\
0 & 0
\end{bmatrix}
$ não é simétrica, uma vez que $A^T
= \begin{bmatrix}
0 & 0 \\
2 & 0
\end{bmatrix}
\neq
A$, e não é antissimétrica, pois $A^T = \begin{bmatrix}
0 & 0 \\
2 & 0
\end{bmatrix}
\neq
\begin{bmatrix}
0 & -2 \\
0 & 0
\end{bmatrix}
= -A$.

\item \textbf{Falso}. Nem mesmo para matrizes de ordem $1 \times 1$ isto é verdade. Por exemplo, se $A = \begin{bmatrix} 3 \end{bmatrix}$ e $B = \begin{bmatrix} -3 \end{bmatrix}$ então $A + B = \begin{bmatrix} 0 \end{bmatrix}$, que não é uma matriz inversível.

Um outro exemplo, com matrizes $2 \times 2$, é o caso em que $A = \begin{bmatrix} 1 & 2 \\ 0 & 3 \end{bmatrix}$, $B = \begin{bmatrix} 5 & 4 \\ 6 & 3 \end{bmatrix}$ e $A + B = \begin{bmatrix} 6 & 6 \\ 6 & 6 \end{bmatrix}$. Como $\det{A+B} = 0$, esta soma de matrizes inversíveis não é inversível.

\item \textbf{Verdadeiro}. Se $A$ é triangular inferior, as entradas acima da diagonal de $A$ são iguais a zero, o que implica que as entradas abaixo da diagonal de $A^T$ também são nulas. Mas se $A$ é antissimétrica, então $A^T = -A$, e resulta que as entradas abaixo da diagonal de $-A$ (e, portanto, de $A$) são todas nulas. Assim, todas as entradas de $A$ fora da diagonal são iguais a zero. Mas a diagonal de qualquer matriz antissimétrica é igual a zero, então $A$ deve ser a matriz nula.
\end{enumerate}


\Exercise[title={2,0}] Calcule a matriz $\operatorname{adj}(M)$ e a soma $(M^2)^{-1} +(M^T)^{-1}$, sendo $M=
\begin{bmatrix}
-1 &  2 & -3\\
 3 & -1 &  0\\
 1 &  0 &  0
\end{bmatrix}$.
\Answer $\operatorname{Adj}(M)=
\begin{bmatrix}
+\begin{vmatrix}
-1 & 0\\
0 & 0
\end{vmatrix}
&
-\begin{vmatrix}
3 & 0\\
1 & 0
\end{vmatrix}
&
+\begin{vmatrix}
3 & -1\\
1 & 0
\end{vmatrix}
\\
-\begin{vmatrix}
2 & -3\\
0 & 0
\end{vmatrix}
&
+\begin{vmatrix}
-1 & -3\\
1 & 0
\end{vmatrix}
&
-\begin{vmatrix}
-1 & 2\\
 1 & 0
\end{vmatrix}
\\
+\begin{vmatrix}
 2 & -3\\
-1 &  0
\end{vmatrix}
&
-\begin{vmatrix}
-1 & -3\\
 3 &  0
\end{vmatrix}
&
+\begin{vmatrix}
-1 &  2\\
 3 & -1
\end{vmatrix}
\\
\end{bmatrix}^T
=
\begin{bmatrix}
0 & 0 & -3\\
0 & 3 & -9\\
1 & 2 &  -5
\end{bmatrix}
$
então
\[
M^{-1}
= \frac{1}{\det{M}} \operatorname{Adj}(M)
=
\frac{1}{-3}
\begin{bmatrix}
0 & 0 & -3\\
0 & 3 & -9\\
1 & 2 &  -5
\end{bmatrix}
=
\begin{bmatrix}
0 & 0 & 1\\
0 & -1 & 3\\
\frac{-1}{3} & \frac{-2}{3} & \frac{5}{3}
\end{bmatrix}
\]

Além disso, como $(M^2)^{-1} +(M^T)^{-1} = (M^{-1})^2 +(M^{-1})^T$ e
\begin{align*}
(M^{-1})^2
=
\frac{-1}{3}
\begin{bmatrix}
0 & 0 & -3\\
0 & 3 & -9\\
1 & 2 &  -5
\end{bmatrix}
\cdot
\frac{-1}{3}
\begin{bmatrix}
0 & 0 & -3\\
0 & 3 & -9\\
1 & 2 &  -5
\end{bmatrix}
=
\frac{1}{9}
\begin{bmatrix}
-3&-6&15\\
-9&-9&18\\
-5&-4&4
\end{bmatrix}
=
\begin{bmatrix}
\frac{-1}{3} & \frac{-2}{3} & \frac{5}{3}\\
-1 & -1 & 2 \\
\frac{-5}{9} & \frac{-4}{9} & \frac{4}{9}
\end{bmatrix}
\end{align*}
tem-se
\[
(M^2)^{-1} +(M^T)^{-1}
=
\begin{bmatrix}
\frac{-1}{3} & \frac{-2}{3} & \frac{5}{3}\\
-1 & -1 & 2 \\
\frac{-5}{9} & \frac{-4}{9} & \frac{4}{9}
\end{bmatrix}
+
\begin{bmatrix}
0 &  0 & \frac{-1}{3}\\
0 & -1 & \frac{-2}{3}\\
1 &  3 & \frac{5}{3}
\end{bmatrix}
=
\begin{bmatrix}
\frac{-1}{3} & \frac{-2}{3} & \frac{4}{3} \\
-1           &      -2      & \frac{4}{3} \\
\frac{4}{9}  & \frac{23}{9} & \frac{19}{9}
\end{bmatrix}.
\]


\Exercise[title={2,0}] Suponha que $M$ é uma matriz quadrada inversível e que $N = \left(3 M^{-1}\right)^T\left(\frac{1}{3}M\right)^{-1}$.
\begin{enumerate}
\item Use as propriedades das operações com matrizes para simplificar a fórmula de $N$. Na fórmula resultante, deve ocorrer no máximo uma operação de cada tipo (transposição, inversão, produto matricial, produto por escalar, etc).
\item Calcule a matriz $N$ no caso em que $M =
\begin{bmatrix}
-1 & -2 & 0 & 2\\
-2 &  2 & 0 & 1\\
 0 &  0 & 3 & 0\\
 2 &  1 & 0 & 2
\end{bmatrix}$.
\end{enumerate}

\Answer
\begin{enumerate}
\item Como $(mA)^T = mA^T$, $(mA)^{-1} = m^{-1}A^{-1}$ e $(A^T)^{-1} = (A^{-1})^T$, tem-se:
\[
D = \left(3 M^{-1}\right)^T\left(\frac{1}{3}M\right)^{-1}
  = 3 \left(M^{-1}\right)^T \left(\frac{1}{3}\right)^{-1} M^{-1}
  = 9 \left(M^T\right)^{-1} M^{-1}
  = 9 \left(M M^T \right)^{-1}.
\]

\item Pelo item anterior, $D = 9 \left(M M^T \right)^{-1}$, e como $M^T = M$ (ou seja, $M$ é simétrica), tem-se:
\[
M M^T
= M^2
= \begin{bmatrix}
-1 & -2 & 0 & 2\\
-2 &  2 & 0 & 1\\
 0 &  0 & 3 & 0\\
 2 &  1 & 0 & 2
\end{bmatrix}
\begin{bmatrix}
-1 & -2 & 0 & 2\\
-2 &  2 & 0 & 1\\
 0 &  0 & 3 & 0\\
 2 &  1 & 0 & 2
\end{bmatrix}
=
\begin{bmatrix}
9 & 0 & 0 & 0\\
0 & 9 & 0 & 0\\
0 & 0 & 9 & 0\\
0 & 0 & 0 & 9
\end{bmatrix}
= 9I.
\]
Assim, $D = 9 (9I)^{-1} = 9 \left(\frac{1}{9} I^{-1}\right) = I$.
\end{enumerate}

\Exercise[title={2,0}] Considerando que $A=
\begin{bmatrix}
 3 & -6 &  9\\
 k & 10 &  0\\
-2 &  4 & -5
\end{bmatrix}
$, $X=
\begin{bmatrix}
x\\
y\\
z
\end{bmatrix}$ e $B=
\begin{bmatrix}
-12\\
  5\\
  7
\end{bmatrix}$, quais devem ser os valores de $k \in \R$ (se houver) para que o sistema linear $AX = B$ seja possível e determinado? E para ser possível e indeterminado? E para ser impossível?
\Answer
\begin{enumerate}
\item Observe que $\det{A} = \begin{vmatrix}
 3 & -6 &  9\\
 k & 10 &  0\\
-2 &  4 & -5
\end{vmatrix} = 6k+30$. Então para $k \neq -5$, tem-se $\det{A} \neq 0$ e a matriz $A$ é inversível, o que implica que o sistema é \textbf{possível e determinado}.
\item Caso contrário, considerando $k = -5$ tem-se:
\begin{align*}
[A|B]
& =
\begin{amatrix}{3}
3 & -6 & 9 & -12 \\
-5 & 10 & 0 & 5\\
-2 & 4 & -5 & 7
\end{amatrix}
\grstep[ \frac{-1}{5}L_2 ]{ \frac{1}{3}L_1 }
\begin{amatrix}{3}
1 & -2 & 3 & -4 \\
1 & -2 & 0 & -1\\
-2 & 4 & -5 & 7
\end{amatrix}
\grstep[ L_3 + 2 L_1 ]{ L_2 - L_1 }
\begin{amatrix}{3}
1 & -2 & 3 & -4 \\
0 & 0 & -3 & 3 \\
0 & 0 & 1 & -1
\end{amatrix}\\
&
\grstep{ \frac{-1}{3} L_2 }
\begin{amatrix}{3}
1 & -2 & 3 & -4 \\
0 & 0 & 1 & -1 \\
0 & 0 & 1 & -1
\end{amatrix}
\grstep[ L_3 - L_2 ]{ L_1 - 3 L_2 }
\begin{amatrix}{3}
1 & -2 & 0 & -1 \\
0 & 0 & 1 & -1 \\
0 & 0 & 0 & 0
\end{amatrix}
\end{align*}
Assim, tanto o posto da matriz $A$ quanto da matriz ampliada $[A|B]$ são iguais a $2$, sendo que o número de variáveis é maior do que $2$, logo o sistema é \textbf{possível e indeterminado}.

\item Pela análise acima, nenhum valor de $k \in \R$ torna o sistema \textbf{impossível} pois o posto de $A$ é sempre igual ao posto da matriz ampliada $[A|B]$.
\end{enumerate}

\Exercise[title={2,0}] Obtenha todo(s) os $(t,x,y,z)$, se existirem, que verificam $
\systeme{
x+8y+2z+8t=2,
2x+16y-z-t=-7,
x+8y+z+7t=-5.
}$
\Answer Pode-se obter um sistema linear equivalente ao do enunciado aplicando o método de Gauss-Jordan à matriz ampliada, como segue:
\begin{align*}
[A|B]
& =
\begin{amatrix}{4}
8 & 1 & 8 & 2 & 2 \\
-1 & 2 & 16 & -1 & -7 \\
7 & 1 & 8 & 1 & -5
\end{amatrix}
\grstep{ \frac{1}{4}L_1 }
\begin{amatrix}{4}
1 & \frac{1}{8} & 1 & \frac{1}{4} & \frac{1}{4} \\
-1 & 2 & 16 & -1 & -7 \\
7 & 1 & 8 & 1 & -5
\end{amatrix}
\grstep[ L_3 - 7 L_1 ]{ L_2 + L_1 }
\begin{amatrix}{4}
1 & \frac{1}{8} & 1 & \frac{1}{4} & \frac{1}{4} \\
0 & \frac{17}{8} & 17 & \frac{-3}{4} & \frac{-27}{4} \\
0 & \frac{1}{8} & 1 & \frac{-3}{4} & \frac{-27}{4}
\end{amatrix} \\
\grstep{ \frac{8}{17} L_2 }
&
\begin{amatrix}{4}
1 & \frac{1}{8} & 1 & \frac{1}{4} & \frac{1}{4} \\
0 & 1 & 8 & \frac{-6}{17} & \frac{-54}{17} \\
0 & \frac{1}{8} & 1 & \frac{-3}{4} & \frac{-27}{4}
\end{amatrix}
\grstep{ L_3 -\frac{1}{8} L_1 }
\begin{amatrix}{4}
1 & \frac{1}{8} & 1 & \frac{1}{4} & \frac{1}{4} \\
0 & 1 & 8 & \frac{-6}{17} & \frac{-54}{17} \\
0 & 0 & 0 & \frac{-12}{17} & \frac{-108}{17}
\end{amatrix}
\grstep{ \frac{-17}{12} L_3 }
\begin{amatrix}{4}
1 & \frac{1}{8} & 1 & \frac{1}{4} & \frac{1}{4} \\
0 & 1 & 8 & \frac{-6}{17} & \frac{-54}{17} \\
0 & 0 & 0 & 1 & 9
\end{amatrix}\\
\grstep{ L_1 -\frac{1}{8} L_2 }
&
\begin{amatrix}{4}
1 & 0 & 0 & \frac{5}{17} & \frac{11}{17} \\
0 & 1 & 8 & \frac{-6}{17} & \frac{-54}{17} \\
0 & 0 & 0 & 1 & 9
\end{amatrix}
\grstep[ L_2 + \frac{6}{17} L_3 ]{ L_2 - \frac{5}{17} L_3 }
\begin{amatrix}{4}
1 & 0 & 0 & 0 & -2 \\
0 & 1 & 8 & 0 & 0 \\
0 & 0 & 0 & 1 & 9
\end{amatrix}
\Leftrightarrow
\systeme{
t=-2,
x+8y=0,
z=9
}
\end{align*}
Logo, as soluções são da forma $(-2, -8y, y, 9)$, em que $y \in \R$ é uma variável livre.

\Exercise[title={2,0}] Se $A = \begin{bmatrix}
1 & 2\\
2 & 3
\end{bmatrix}$, obtenha todos os $(x,y,z,w)$ tais que $X = \begin{bmatrix}
x & y\\
z & w
\end{bmatrix}$ satisfaz $AX = XA$.
\Answer Para as matrizes dadas, tem-se
\begin{align*}
AX = XA
& \Leftrightarrow
\begin{bmatrix}
1 & 2\\
2 & 3
\end{bmatrix}
\begin{bmatrix}
x & y\\
z & w
\end{bmatrix}
=
\begin{bmatrix}
x & y\\
z & w
\end{bmatrix}
\begin{bmatrix}
1 & 2\\
2 & 3
\end{bmatrix}
\Leftrightarrow
\begin{bmatrix}
x+2z & y+2w\\
2x+3z & 2y+3w
\end{bmatrix}
=
\begin{bmatrix}
x+2y & 2x+3y\\
z+2w & 2z+3w
\end{bmatrix}
\\
& \Leftrightarrow
\begin{cases}
x+2z &= x + 2y\\
y+2w &= 2x + 3y\\
2x+3z &= z + 2w\\
2y+3w &= 2z + 3w
\end{cases}
\Leftrightarrow
\systeme[xyzw]{
 2z -2y = 0,
-2x-2y+2w = 0,
2x+2z-2w = 0,
2y-2z = 0
}
\Leftrightarrow
\systeme[xyzw]{
 z -y = 0,
-x-y+w = 0,
x+z-w = 0,
y-z = 0
}
\end{align*}
Escalonando a matriz deste sistema, resulta:
\begin{align*}
A
& =
\begin{bmatrix}
0 & -1 & 1 & 0 \\
-1 & -1 & 0 & 1 \\
1 & 0 & 1 & -1 \\
0 & 1 & -1 & 0
\end{bmatrix}
\grstep{ L_3 \swap L_1}
\begin{bmatrix}
1 & 0 & 1 & -1 \\
-1 & -1 & 0 & 1 \\
0 & -1 & 1 & 0 \\
0 & 1 & -1 & 0
\end{bmatrix}
\grstep[ L_4 + L_3 ]{ L_2+L_1}
\begin{bmatrix}
1 & 0 & 1 & -1 \\
0 & -1 & 1 & 0 \\
0 & -1 & 1 & 0 \\
0 & 0 & 0 & 0
\end{bmatrix}
\grstep[ L_3 - L_2 ]{ -L_2 }
\begin{bmatrix}
1 & 0 & 1 & -1 \\
0 & 1 & -1 & 0 \\
0 & 0 & 0 & 0 \\
0 & 0 & 0 & 0
\end{bmatrix}
\end{align*}
Desconsiderando as linhas nulas desta matriz, resulta o sistema homogêneo
\[
\systeme[xyzw]{
x+z-w = 0,
y-z = 0
}
\]
cujas soluções são da forma $(-z+w,z,z,w)$.

\end{ExerciseList}

\begin{center}
BOA PROVA!
\end{center}

\newpage
\restoregeometry
\section*{Respostas}
\shipoutAnswer
\end{document}
