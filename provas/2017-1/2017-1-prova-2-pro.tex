\documentclass[12pt,a4paper]{article}
\usepackage{cmap} % Makes the PDF copiable. See http://tex.stackexchange.com/a/64198/25761
\usepackage[T1]{fontenc}
\usepackage[brazil]{babel}
\usepackage[utf8]{inputenc}
\usepackage{amsmath}
\usepackage{amsfonts}
\usepackage{amssymb}
\usepackage{amsthm}
\usepackage{textcomp} % \degree
\usepackage{gensymb} % \degree
\usepackage[usenames,svgnames,dvipsnames]{xcolor}
\usepackage{hyperref}
\usepackage{multicol}
\usepackage{graphicx}
\usepackage[margin=2cm]{geometry}
\usepackage{systeme}

\hypersetup{
    colorlinks = true,
    allcolors = {blue}
}

% TODO: Consider using exsheets
% http://linorg.usp.br/CTAN/macros/latex/contrib/exsheets/exsheets_en.pdf
%
% http://ctan.org/tex-archive/macros/latex/contrib/exercise/
% Options: answerdelayed,lastexercise,noanswer
\usepackage[answerdelayed,lastexercise]{exercise}

\addto\captionsbrazil{%
\def\listexercisename{Lista de exerc\'icios}%
\def\ExerciseName{Exerc\'icio}%
\def\AnswerName{Solu\c{c}\~ao do exerc\'icio}%
\def\ExerciseListName{Ex.}%
\def\AnswerListName{Solu\c{c}\~ao}%
\def\ExePartName{Parte}%
\def\ArticleOf{de\ }%
}

\renewcommand{\ExerciseHeaderTitle}{(\ExerciseTitle)\ }
\renewcommand{\ExerciseListHeader}{%\ExerciseHeaderDifficulty%
\textbf{%\ExerciseListName\
\ExerciseHeaderNB.\ %
%\ --- \
\ExerciseHeaderTitle}%
%\ExerciseHeaderOrigin
\ignorespaces}
\renewcommand{\AnswerListHeader}{\textbf{\ExerciseHeaderNB.\ (\AnswerListName)\ }}

\newcommand{\fixme}{{\color{red}(...)}}
\newcommand*\R{\mathbb{R}}

% Loop Space / CC BY-SA-3.0 / https://tex.stackexchange.com/a/2238/25761
\newenvironment{amatrix}[1]{%
  \left[\begin{array}{@{}*{#1}{c}|c@{}}
}{%
  \end{array}\right]
}

% Loop Space / CC BY-SA-3.0 / https://tex.stackexchange.com/a/3164/25761
%--------grstep
% For denoting a Gauss' reduction step.
% Use as: \grstep{\rho_1+\rho_3} or \grstep[2\rho_5 \\ 3\rho_6]{\rho_1+\rho_3}
\newcommand{\grstep}[2][\relax]{%
   \ensuremath{\mathrel{
       {\mathop{\longrightarrow}\limits^{#2\mathstrut}_{
                                     \begin{subarray}{l} #1 \end{subarray}}}}}}

\renewcommand{\theenumi}{\alph{enumi}}
\renewcommand\labelenumi{(\theenumi) }

\newcommand*\tipo{Prova II}
\newcommand*\turma{PRO112-02U}
\newcommand*\disciplina{ALI0001}
\newcommand*\eu{Helder G. G. de Lima}
\newcommand*\data{24/04/2017}

\author{\eu}
\title{\tipo - \disciplina}
\date{\data}

\begin{document}
\thispagestyle{empty}
\newgeometry{margin=2cm,bottom=0.5cm}
\begin{center}
\includegraphics[width=9.0cm]{marca} \\
\textbf{\tipo\ (\disciplina / \turma)} \\
Prof. \eu\footnote{
Este é um material de acesso livre distribuído sob os termos da licença \href{https://creativecommons.org/licenses/by-sa/4.0/deed.pt_BR}{Creative Commons BY-SA 4.0}}
\end{center}

\noindent Nome do(a) aluno(a): \underline{\hspace{9,7cm}} Data: \underline{\data}

%\section*{Instruções}
\begin{center}\fbox{
\begin{minipage}{14cm}

{\footnotesize
\begin{itemize}
\renewcommand{\theenumi}{\Roman{enumi}}
\item Identifique-se em todas as folhas.
\item Mantenha o celular e os demais equipamentos eletrônicos desligados durante a prova.
\item Resolva (integralmente) apenas os itens de que precisar para somar 10,0 pontos.
\end{itemize}
}

\end{minipage}
}
\end{center}

\section*{Questões}
\begin{ExerciseList}
\Exercise[title={2,5}] Seja $V = \{ (x,y) \in \R^2 \ |\ x > 0 \}$ e considere as seguintes definições:
\begin{align*}
(a,b) + (c,d)  & = (ac, b + d), \forall (a, b) \in V, \forall (c, d) \in V, \\
k \cdot (p, q) & = (p^k, kq), \forall (p, q) \in V, \forall k \in \R.
\end{align*}
Pode-se mostrar que, com essas definições de $+$ e $\cdot$, o conjunto $V$ cumpre todas as exigências para ser considerado um espaço vetorial.
\begin{enumerate}
\item Verifique, de forma rigorosa, que a adição é comutativa.
\item Identifique que elemento de $V$ faz o papel de vetor nulo deste espaço vetorial.
\item Para cada vetor $w = (x,y) \in V$, obtenha o vetor $-w$ (oposto em relação à adição)
\end{enumerate}
\Answer \fixme

\Exercise[title={2,5}] Seja $S$ o espaço vetorial formado por todas as matrizes simétricas de ordem $3 \times 3$. Dê um exemplo de um conjunto $A$ de vetores de $S$ que sejam linearmente independentes mas que não gerem $S$ e um exemplo de um conjunto $B$ de vetores que geram $S$ mas que não sejam linearmente independentes. Justifique todas as suas afirmações.
\Answer \fixme

\Exercise[title={2,5}] Sejam $W_1= \{ (a,b,c,d) \in \R^4\ |\ a = d\}$ e $W_2= \{ (u+v,u+w,v+w, 0) \in \R^4 | u,v,w \in \R \}$. Encontre uma base de $W_1 \cap W_2$, mostre que realmente é uma base, e determine sua dimensão.
\Answer \fixme

\Exercise[title={2,5}] Prove que, dada qualquer matriz $A$ de ordem $m \times n$, o conjunto $N(A)$ formado por todas as matrizes $X$ de ordem $n \times 1$ para as quais $A X = 0_{m\times 1}$, é um subespaço vetorial de $M_{n \times 1}$. Exiba uma base e a dimensão de $N(A)$ no caso particular em que $A = \begin{bmatrix}
2 & -4 & 6 & -8
\end{bmatrix}$, juntamente com os cálculos e argumentos que garantem que encontrou uma base de $N(A)$.
\Answer \fixme

\Exercise[title={2,5}] Uma função $f: \R \to \R$ é denominada \textbf{par} quando $f(-x) = f(x), \forall x \in \R$, e \textbf{ímpar} quando $f(-x) = -f(x), \forall x \in \R$. Mostre que o espaço vetorial $F$ de todas as funções $f: \R \to \R$ é uma soma direta do subespaço $Fp$ formado pelas funções pares com o subespaço $Fi$ formado pelas funções ímpares.

(Dica: dada $f \in F$, analize as funções $g(x) = \frac{f(x) + f(-x)}{2}$ e $h(x) = \frac{f(x) - f(-x)}{2}$)
\Answer \fixme
\end{ExerciseList}

\begin{center}
BOA PROVA!
\end{center}

%\newpage
%\restoregeometry
%\section*{Respostas}
%\shipoutAnswer
\end{document}
