\documentclass[12pt,a4paper]{article}
\usepackage{cmap} % Makes the PDF copiable. See http://tex.stackexchange.com/a/64198/25761
\usepackage[T1]{fontenc}
\usepackage[brazil]{babel}
\usepackage[utf8]{inputenc}
\usepackage{amsmath}
\usepackage{amsfonts}
\usepackage{amssymb}
\usepackage{amsthm}
\usepackage{textcomp} % \degree
\usepackage{gensymb} % \degree
\usepackage[usenames,svgnames,dvipsnames]{xcolor}
\usepackage{hyperref}
\usepackage{multicol}
\usepackage{graphicx}
\usepackage[margin=2cm]{geometry}
\usepackage{systeme}

\hypersetup{
    colorlinks = true,
    allcolors = {blue}
}

% TODO: Consider using exsheets
% http://linorg.usp.br/CTAN/macros/latex/contrib/exsheets/exsheets_en.pdf
%
% http://ctan.org/tex-archive/macros/latex/contrib/exercise/
% Options: answerdelayed,lastexercise,noanswer
\usepackage[answerdelayed,lastexercise]{exercise}

\addto\captionsbrazil{%
\def\listexercisename{Lista de exerc\'icios}%
\def\ExerciseName{Exerc\'icio}%
\def\AnswerName{Solu\c{c}\~ao do exerc\'icio}%
\def\ExerciseListName{Ex.}%
\def\AnswerListName{Solu\c{c}\~ao}%
\def\ExePartName{Parte}%
\def\ArticleOf{de\ }%
}

\renewcommand{\ExerciseHeaderTitle}{(\ExerciseTitle)\ }
\renewcommand{\ExerciseListHeader}{%\ExerciseHeaderDifficulty%
\textbf{%\ExerciseListName\
\ExerciseHeaderNB.\ %
%\ --- \
\ExerciseHeaderTitle}%
%\ExerciseHeaderOrigin
\ignorespaces}
\renewcommand{\AnswerListHeader}{\textbf{\ExerciseHeaderNB.\ (\AnswerListName)\ }}

\newcommand{\fixme}{{\color{red}(...)}}
\newcommand*\R{\mathbb{R}}

% Loop Space / CC BY-SA-3.0 / https://tex.stackexchange.com/a/2238/25761
\newenvironment{amatrix}[1]{%
  \left[\begin{array}{@{}*{#1}{c}|c@{}}
}{%
  \end{array}\right]
}

% Loop Space / CC BY-SA-3.0 / https://tex.stackexchange.com/a/3164/25761
%--------grstep
% For denoting a Gauss' reduction step.
% Use as: \grstep{\rho_1+\rho_3} or \grstep[2\rho_5 \\ 3\rho_6]{\rho_1+\rho_3}
\newcommand{\grstep}[2][\relax]{%
   \ensuremath{\mathrel{
       {\mathop{\longrightarrow}\limits^{#2\mathstrut}_{
                                     \begin{subarray}{l} #1 \end{subarray}}}}}}

\renewcommand{\theenumi}{\alph{enumi}}
\renewcommand\labelenumi{(\theenumi) }

\newcommand*\tipo{Prova IV}
\newcommand*\turma{NEX171-C}
\newcommand*\disciplina{ALI0001}
\newcommand*\eu{Helder G. G. de Lima}
\newcommand*\data{27/06/2017}

\author{\eu}
\title{\tipo - \disciplina}
\date{\data}

\begin{document}
\thispagestyle{empty}
\newgeometry{margin=2cm,bottom=0.5cm}
\begin{center}
\includegraphics[width=9.0cm]{marca} \\
\textbf{\tipo\ (\disciplina / \turma)} \\
Prof. \eu\footnote{
Este é um material de acesso livre distribuído sob os termos da licença \href{https://creativecommons.org/licenses/by-sa/4.0/deed.pt_BR}{Creative Commons BY-SA 4.0}}
\end{center}

\noindent Nome do(a) aluno(a): \underline{\hspace{9,7cm}} Data: \underline{\data}

%\section*{Instruções}
\begin{center}\fbox{
\begin{minipage}{14cm}

{\footnotesize
\begin{itemize}
\renewcommand{\theenumi}{\Roman{enumi}}
\item Identifique-se em todas as folhas.
\item Mantenha o celular e os demais equipamentos eletrônicos desligados durante a prova.
\end{itemize}
}

\end{minipage}
}
\end{center}

\section*{Questões}
\begin{ExerciseList}
\Exercise[title={2,5}] Diagonalize a matriz $A = \begin{bmatrix}
0 & 2 \\ 1 & 1
\end{bmatrix}$ e utilize a fatoração obtida para calcular $A^8$.
\Answer \fixme
\Exercise[title={2,5}] Seja $v$ um autovetor de $A$ associado ao autovalor $\lambda$.
\begin{enumerate}
\item Mostre que se $\lambda \neq 0$ então $u=Av$ também é um autovetor de $A$.
\item Mostre que $\lambda^2 - 3\lambda$ é um autovalor de $A^2 - 3A$.
\end{enumerate}

\Answer \fixme
\Exercise[title={2,5}] Seja $T: \R^3 \to \R^3$ um operador linear. Suponha que $v_1 = (0,3,1) \in N(T)$ e que $T$ duplique os vetores $v_2 = (-1,1,0)$ e $v_3 = (0,2,0)$.
\begin{enumerate}
\item Mostre que $T(1,4,1) = (2,2,0)$.
\item Qual é a matriz de $T$ em relação à base $\alpha = \{ (1,1,0), (2,0,0), (1,4,1) \}$?
\item Dê um exemplo de uma base $\beta$, se existir, em relação à qual a matriz de $T$ seja diagonal.
\end{enumerate}

\Answer \fixme
\Exercise[title={2,5}] Encontre todos os valores de $k \in \R$ tais que a matriz $A = \begin{bmatrix}
1 & k \\ 1 & -3
\end{bmatrix}$ é diagonalizável.
\Answer \fixme
\end{ExerciseList}

\section*{Pontos extras (produto interno)}
\begin{ExerciseList}
\Exercise[title={1,5}] Considerando o produto interno usual de $\R^3$, verifique se a base
\[
\alpha = \{ (2,4,4), (4,2,-4), (4,-4,2)\}
\]
é ortogonal. Ela é ortonormal?
\Answer \fixme
\Exercise[title={1,5}] Utilize o produto interno usual de $\R^3$ para calcular as coordenadas de $v = (7,9,0)$ em relação à base $\alpha$ (não deve ser resolvido nenhum sistema linear).
\Answer \fixme

\end{ExerciseList}

%\newpage
%\restoregeometry
%\section*{Respostas}
%\shipoutAnswer
\end{document}
