\documentclass[12pt,a4paper]{article}
\usepackage{cmap} % Makes the PDF copiable. See http://tex.stackexchange.com/a/64198/25761
\usepackage[T1]{fontenc}
\usepackage[brazil]{babel}
\usepackage[utf8]{inputenc}
\usepackage{amsmath}
\usepackage{amsfonts}
\usepackage{amssymb}
\usepackage{amsthm}
\usepackage{textcomp} % \degree
\usepackage{gensymb} % \degree
\usepackage[usenames,svgnames,dvipsnames]{xcolor}
\usepackage{hyperref}
\usepackage{multicol}
\usepackage{graphicx}
\usepackage[margin=2cm]{geometry}
\usepackage{systeme}

\hypersetup{
    colorlinks = true,
    allcolors = {blue}
}

% TODO: Consider using exsheets
% http://linorg.usp.br/CTAN/macros/latex/contrib/exsheets/exsheets_en.pdf
%
% http://ctan.org/tex-archive/macros/latex/contrib/exercise/
% Options: answerdelayed,lastexercise,noanswer
\usepackage[answerdelayed,lastexercise]{exercise}

\addto\captionsbrazil{%
\def\listexercisename{Lista de exerc\'icios}%
\def\ExerciseName{Exerc\'icio}%
\def\AnswerName{Solu\c{c}\~ao do exerc\'icio}%
\def\ExerciseListName{Ex.}%
\def\AnswerListName{Solu\c{c}\~ao}%
\def\ExePartName{Parte}%
\def\ArticleOf{de\ }%
}

\renewcommand{\ExerciseHeaderTitle}{(\ExerciseTitle)\ }
\renewcommand{\ExerciseListHeader}{%\ExerciseHeaderDifficulty%
\textbf{%\ExerciseListName\
\ExerciseHeaderNB.\ %
%\ --- \
\ExerciseHeaderTitle}%
%\ExerciseHeaderOrigin
\ignorespaces}
\renewcommand{\AnswerListHeader}{\textbf{\ExerciseHeaderNB.\ (\AnswerListName)\ }}

\newcommand{\fixme}{{\color{red}(...)}}
\newcommand*\R{\mathbb{R}}

% Loop Space / CC BY-SA-3.0 / https://tex.stackexchange.com/a/2238/25761
\newenvironment{amatrix}[1]{%
  \left[\begin{array}{@{}*{#1}{c}|c@{}}
}{%
  \end{array}\right]
}

% Loop Space / CC BY-SA-3.0 / https://tex.stackexchange.com/a/3164/25761
%--------grstep
% For denoting a Gauss' reduction step.
% Use as: \grstep{\rho_1+\rho_3} or \grstep[2\rho_5 \\ 3\rho_6]{\rho_1+\rho_3}
\newcommand{\grstep}[2][\relax]{%
   \ensuremath{\mathrel{
       {\mathop{\longrightarrow}\limits^{#2\mathstrut}_{
                                     \begin{subarray}{l} #1 \end{subarray}}}}}}

\renewcommand{\theenumi}{\alph{enumi}}
\renewcommand\labelenumi{(\theenumi) }

\newcommand*\tipo{Prova II}
\newcommand*\turma{NEX171-C}
\newcommand*\disciplina{ALI0001}
\newcommand*\eu{Helder G. G. de Lima}
\newcommand*\data{25/04/2017}

\author{\eu}
\title{\tipo - \disciplina}
\date{\data}

\begin{document}
\thispagestyle{empty}
\newgeometry{margin=2cm,bottom=0.5cm}
\begin{center}
\includegraphics[width=9.0cm]{marca} \\
\textbf{\tipo\ (\disciplina / \turma)} \\
Prof. \eu\footnote{
Este é um material de acesso livre distribuído sob os termos da licença \href{https://creativecommons.org/licenses/by-sa/4.0/deed.pt_BR}{Creative Commons BY-SA 4.0}}
\end{center}

\noindent Nome do(a) aluno(a): \underline{\hspace{9,7cm}} Data: \underline{\data}

%\section*{Instruções}
\begin{center}\fbox{
\begin{minipage}{14cm}

{\footnotesize
\begin{itemize}
\renewcommand{\theenumi}{\Roman{enumi}}
\item Identifique-se em todas as folhas.
\item Mantenha o celular e os demais equipamentos eletrônicos desligados durante a prova.
\item Resolva (integralmente) apenas os itens de que precisar para somar 10,0 pontos.
\end{itemize}
}

\end{minipage}
}
\end{center}

\section*{Questões}
\begin{ExerciseList}
\Exercise[title={2,5}] Seja $V = \{ (x,y) \in \R^2 \ |\ y > 0 \}$.
\begin{enumerate}
\item Exemplifique por que $V$ não é um subespaço de $\R^2$ (com as operações usuais).
\item Apesar do que ocorre com as operações usuais, o conjunto $V$ passa a ser um espaço vetorial se forem usadas as operações (não usuais) a seguir:
\begin{align*}
(a,b) + (c,d)  & = (a+c, bd), \forall (a, b) \in V, \forall (c, d) \in V, \\
k \cdot (p, q) & = (kp, q^k), \forall (p, q) \in V, \forall k \in \R.
\end{align*}
\begin{itemize}
\item Identifique que elemento de $V$ faz o papel de vetor nulo deste espaço vetorial.
\item Para cada vetor $w = (x,y) \in V$, obtenha o vetor $-w$ (oposto em relação à adição)
\end{itemize}
\end{enumerate}
\Answer \fixme

\Exercise[title={2,5}] Seja $P_2$ o espaço vetorial formado por todos os polinômios da forma $q(x) = ax^2 + bx + c$, com $a,b,c \in \R$.
Verifique se $W = \{ q(x) \in P_2\ |\ q(-1) = -q(1) \}$ é um subespaço de $P_2$. Em caso afirmativo, obtenha uma base de $W$. Caso contrário, obtenha o subespaço $G(W)$, gerado pelos vetores de $W$. Justifique todas as suas afirmações e conclusões.
\Answer \fixme

\Exercise[title={2,5}] Considere os seguintes subespaços de $\R^4$:
\begin{align*}
W_1 & = \{ (a-b,b-c,c-a,a+b-c) \in \R^4\ |\ a, b, c \in \R \} \\
W_2 & = \{ (r,s,t,u) \in \R^4\ |\ r = 0, s = t, u \in \R \}.
\end{align*}
Encontre uma base e a dimensão de $W_1 \cap W_2$.
\Answer \fixme

\Exercise[title={2,5}] Mostre que se $B_1 = \{ v, w \}$ é uma base de um espaço vetorial $V$ então o conjunto $B_2 = \{v + w, w - v \}$ também satisfaz todas as condições para ser base de $V$.
\Answer \fixme


\Exercise[title={2,5}] Uma matriz $X$ é simétrica se $X^T = X$ e é antissimétrica se $X^T = -X$. Mostre que o espaço vetorial $M_{n\times n}$ de todas as matrizes de ordem $n\times n$ é uma soma direta do subespaço $S$ formado pelas matrizes simétricas com o subespaço $A$ formado pelas matrizes antissimétricas. (Dica: dada $X \in M_{n \times n}$, analize as matrizes $Y = \frac{X + X^T}{2}$ e $Z = \frac{X - X^T}{2}$)
\Answer \fixme
\end{ExerciseList}

\begin{center}
BOA PROVA!
\end{center}

%\newpage
%\restoregeometry
%\section*{Respostas}
%\shipoutAnswer
\end{document}
