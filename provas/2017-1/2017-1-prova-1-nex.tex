\documentclass[12pt,a4paper]{article}
\usepackage{cmap} % Makes the PDF copiable. See http://tex.stackexchange.com/a/64198/25761
\usepackage[T1]{fontenc}
\usepackage[brazil]{babel}
\usepackage[utf8]{inputenc}
\usepackage{amsmath}
\usepackage{amsfonts}
\usepackage{amssymb}
\usepackage{amsthm}
\usepackage{textcomp} % \degree
\usepackage{gensymb} % \degree
\usepackage[usenames,svgnames,dvipsnames]{xcolor}
\usepackage{hyperref}
\usepackage{multicol}
\usepackage{graphicx}
\usepackage[margin=2cm]{geometry}
\usepackage{systeme}

\hypersetup{
    colorlinks = true,
    allcolors = {blue}
}

% TODO: Consider using exsheets
% http://linorg.usp.br/CTAN/macros/latex/contrib/exsheets/exsheets_en.pdf
%
% http://ctan.org/tex-archive/macros/latex/contrib/exercise/
% Options: answerdelayed,lastexercise,noanswer
\usepackage[answerdelayed,lastexercise]{exercise}

\addto\captionsbrazil{%
\def\listexercisename{Lista de exerc\'icios}%
\def\ExerciseName{Exerc\'icio}%
\def\AnswerName{Solu\c{c}\~ao do exerc\'icio}%
\def\ExerciseListName{Ex.}%
\def\AnswerListName{Solu\c{c}\~ao}%
\def\ExePartName{Parte}%
\def\ArticleOf{de\ }%
}

\renewcommand{\ExerciseHeaderTitle}{(\ExerciseTitle)\ }
\renewcommand{\ExerciseListHeader}{%\ExerciseHeaderDifficulty%
\textbf{%\ExerciseListName\
\ExerciseHeaderNB.\ %
%\ --- \
\ExerciseHeaderTitle}%
%\ExerciseHeaderOrigin
\ignorespaces}
\renewcommand{\AnswerListHeader}{\textbf{\ExerciseHeaderNB.\ (\AnswerListName)\ }}

\newcommand{\fixme}{{\color{red}(...)}}

% Loop Space / CC BY-SA-3.0 / https://tex.stackexchange.com/a/2238/25761
\newenvironment{amatrix}[1]{%
  \left[\begin{array}{@{}*{#1}{c}|c@{}}
}{%
  \end{array}\right]
}

% Loop Space / CC BY-SA-3.0 / https://tex.stackexchange.com/a/3164/25761
%--------grstep
% For denoting a Gauss' reduction step.
% Use as: \grstep{\rho_1+\rho_3} or \grstep[2\rho_5 \\ 3\rho_6]{\rho_1+\rho_3}
\newcommand{\grstep}[2][\relax]{%
   \ensuremath{\mathrel{
       {\mathop{\longrightarrow}\limits^{#2\mathstrut}_{
                                     \begin{subarray}{l} #1 \end{subarray}}}}}}

\renewcommand{\theenumi}{\alph{enumi}}
\renewcommand\labelenumi{(\theenumi) }

\newcommand*\tipo{Prova I}
\newcommand*\turma{NEX171-C}
\newcommand*\disciplina{ALI0001}
\newcommand*\eu{Helder G. G. de Lima}
\newcommand*\data{22/03/2017}

\author{\eu}
\title{\tipo - \disciplina}
\date{\data}

\begin{document}
\thispagestyle{empty}
\newgeometry{margin=2cm,bottom=0.5cm}
\begin{center}
\includegraphics[width=9.0cm]{marca} \\
\textbf{\tipo\ (\disciplina / \turma)} \\
Prof. \eu\footnote{
Este é um material de acesso livre distribuído sob os termos da licença \href{https://creativecommons.org/licenses/by-sa/4.0/deed.pt_BR}{Creative Commons BY-SA 4.0}}
\end{center}

\noindent Nome do(a) aluno(a): \underline{\hspace{9,7cm}} Data: \underline{\data}

%\section*{Instruções}
\begin{center}\fbox{
\begin{minipage}{14cm}

{\footnotesize
\begin{itemize}
\renewcommand{\theenumi}{\Roman{enumi}}
\item Identifique-se em todas as folhas.
\item Mantenha o celular e os demais equipamentos eletrônicos desligados durante a prova.
\item Resolva (integralmente) apenas os itens de que precisar para somar 10,0 pontos.
\end{itemize}
}

\end{minipage}
}
\end{center}

\section*{Questões}
\begin{ExerciseList}

\Exercise[title={2,5}] Prove (sem dar exemplo) as afirmações verdadeiras e exemplifique o erro nas demais:
\begin{enumerate}
\item Se $A$ é inversível e $A^{-1}$ é simétrica, então $A$ é simétrica.
\item Se uma matriz $2 \times 2$ inversível é triangular \textit{superior}, então sua inversa é triangular \textit{inferior}.
\end{enumerate}
\Answer \fixme

\Exercise[title={2,5}] Considere $M = P^T Q + I$, sendo $P = \begin{bmatrix}
1 & 2 & -1 & 0
\end{bmatrix}$, $Q = \begin{bmatrix}
-1 & k & 1 & 0
\end{bmatrix}$ e $I$ a matriz identidade de um tamanho apropriado. Para quais valores de $k$ a matriz $M$ é inversível?
\Answer $k \neq 1/2$

\Exercise[title={2,5}] Sejam $
A =
\begin{bmatrix}
0 & -1 \\
1 & 4
\end{bmatrix}
$, $B = A^{-1}$, $C = 2A$ e $I$ a identidade $2 \times 2$. Encontre a(s) matriz(es) $
X =
\begin{bmatrix}
x & y \\
z & w
\end{bmatrix}
$, se existir(em), de tal modo que $BX - I = C^{-1}$.
\Answer $M = -A - I =
\begin{bmatrix}
1/2 & -1 \\
1 & 9/2
\end{bmatrix}$.

\Exercise[title={2,5}] Um estudante observou que ao multiplicar uma matriz $M$ de ordem $2\times 2$ pela matriz coluna $C =
\begin{bmatrix}
-1\\3
\end{bmatrix}
$ o resultado era $D =
\begin{bmatrix}
1\\1
\end{bmatrix}
$. Outra estudante percebeu que, curiosamente, o produto de $M$ por $D$ era $C$. Qual foi a matriz $M$ utilizada nesses cálculos? Alguma outra matriz produziria os mesmos resultados?


\Answer \fixme

\Exercise[title={2,5}] Considere as matrizes $A$, $B$ e $C$ definidas a seguir:
\[
A = \begin{bmatrix}
0 & -1 & 0 & 0 \\
-2 & 1 & 0 & 0 \\
1 & 0 & 5 & 0 \\
0 & 2 & 0 & 1
\end{bmatrix},
B = \begin{bmatrix}
1 & 4 & 1 & 3
\end{bmatrix}
\text{ e }
C = \begin{bmatrix}
0 & -1 & 2 & 1
\end{bmatrix}.
\]
Determine o resultado das seguintes operações, quando forem possíveis. Caso contrário, explique o motivo da impossibilidade.
\begin{enumerate}
\item $\det{(A \cdot A^T)} - \det(2 \cdot B^T \cdot C)$
\item $(B \cdot A \cdot C^T)^{-1}$
\end{enumerate}
\Answer \fixme
\end{ExerciseList}

\begin{center}
BOA PROVA!
\end{center}

%\newpage
%\restoregeometry
%\section*{Respostas}
%\shipoutAnswer
\end{document}
