\documentclass[12pt,a4paper]{article}
\usepackage{cmap} % Makes the PDF copiable. See http://tex.stackexchange.com/a/64198/25761
\usepackage[T1]{fontenc}
\usepackage[brazil]{babel}
\usepackage[utf8]{inputenc}
\usepackage{amsmath}
\usepackage{amsfonts}
\usepackage{amssymb}
\usepackage{amsthm}
\usepackage{textcomp} % \degree
\usepackage{gensymb} % \degree
\usepackage[usenames,svgnames,dvipsnames]{xcolor}
\usepackage{hyperref}
\usepackage{multicol}
\usepackage{graphicx}
\usepackage[top=2cm, bottom=2cm, left=2cm, right=2cm]{geometry}
\usepackage{systeme}

\hypersetup{
    colorlinks = true,
    allcolors = {blue}
}

\newcommand{\fixme}{{\color{red}(...)}}
\newcommand\ii{\mathrm{i}} 

\newcommand*\N{\mathbb{N}}
\newcommand*\R{\mathbb{R}}
\newcommand*\C{\mathbb{C}}

\newcommand{\IconPc}{\includegraphics[width=1em]{computer.png}}
\newcommand{\IconCalc}{\includegraphics[width=1em]{calculator.png}}
\newcommand{\IconThink}{\includegraphics[width=1em]{pencil.png}}
\newcommand{\IconCheck}{\includegraphics[width=1em]{checkmark.png}}
\newcommand{\IconConcept}{\includegraphics[width=1em]{edit.png}}

\newlength{\SmileysLength}
\setlength{\SmileysLength}{\labelwidth}\addtolength{\SmileysLength}{\labelsep}

\newcommand{\calc}{\hspace*{-\SmileysLength}\makebox[0pt][r]{\IconCalc}%
   \hspace*{\SmileysLength}}
\newcommand{\software}{\hspace*{-\SmileysLength}\makebox[0pt][r]{\IconPc}%
   \hspace*{\SmileysLength}}
\newcommand{\teoria}{\hspace*{-\SmileysLength}\makebox[0pt][r]{\IconThink}%
   \hspace*{\SmileysLength}}
\newcommand{\conceito}{\hspace*{-\SmileysLength}\makebox[0pt][r]{\IconCheck}%
   \hspace*{\SmileysLength}}
\newcommand{\concept}{\hspace*{-\SmileysLength}\makebox[0pt][r]{\IconCheck}%
   \hspace*{\SmileysLength}}

% Loop Space / CC BY-SA-3.0 / https://tex.stackexchange.com/a/2238/25761
\newenvironment{amatrix}[1]{%
  \left[\begin{array}{@{}*{#1}{c}|c@{}}
}{%
  \end{array}\right]
}

% Loop Space / CC BY-SA-3.0 / https://tex.stackexchange.com/a/3164/25761
%--------grstep
% For denoting a Gauss' reduction step.
% Use as: \grstep{\rho_1+\rho_3} or \grstep[2\rho_5 \\ 3\rho_6]{\rho_1+\rho_3}
\newcommand{\grstep}[2][\relax]{%
   \ensuremath{\mathrel{
       {\mathop{\longrightarrow}\limits^{#2\mathstrut}_{
                                     \begin{subarray}{l} #1 \end{subarray}}}}}}
\newcommand{\swap}{\leftrightarrow}

\newcommand*\tipo{2ª Lista de Exercícios}
%\newcommand*\turma{...}
\newcommand*\disciplina{ALI0001}
\newcommand*\eu{Helder G. G. de Lima}
\newcommand*\data{\today}

\author{\eu}
\title{\tipo - \disciplina}
\date{}

\begin{document}

\begin{center}
\includegraphics[width=9.0cm]{marca.jpg} \\
\textbf{\tipo\ (\disciplina)} \\
Prof. \eu\footnote{
Este é um material de acesso livre distribuído sob os termos da licença \href{https://creativecommons.org/licenses/by-sa/4.0/deed.pt_BR}{Creative Commons Atribuição-CompartilhaIgual 4.0 Internacional}}
\end{center}

\section*{Legenda}
\begin{multicols}{4}
\begin{itemize}
\item[] \hspace*{\SmileysLength} \calc \hspace*{-\SmileysLength} Cálculos
\item[] \hspace*{\SmileysLength} \conceito \hspace*{-\SmileysLength} Conceitos
\item[] \hspace*{\SmileysLength} \teoria \hspace*{-\SmileysLength} Teoria
%\item[] \hspace*{\SmileysLength} \software \hspace*{-\SmileysLength} Software
\end{itemize}
\end{multicols}

\section*{Questões}

\begin{enumerate}
\item \conceito Revise todos os axiomas da definição de espaço vetorial $V$ sobre o corpo de escalares $\R$, verificando a validade de cada um deles nos seguintes conjuntos. Em cada item, as operações de adição e multiplicação por escalar que aparecem no lado esquerdo são as do espaço vetorial que está sendo definido, e as do lado direito são operações usuais de $\R$.

\begin{enumerate}
\item $V = \R^n$, formado por todas as sequências de $n$ números reais, e
\begin{align*}
(x_1, \ldots, x_n) + (y_1, \ldots, y_n)
& = (x_1 + y_1, \ldots, x_n + y_n)\\
k \cdot (x_1, \ldots, x_n)
& = (kx_1, \ldots, kx_n)
\end{align*}
\item $V = M_{m \times n} (\R)$, formado por todas as matrizes de ordem $m \times n$, com $m,n$ fixos e
\begin{align*}
(A + B)_{ij}
& = (A)_{ij} + (B)_{ij}, \text{ para } 1 \leq i \leq m \text{ e } 1 \leq j \leq n,\\
(k \cdot B)_{ij}
& = k(A_{ij}), \text{ para } 1 \leq i \leq m \text{ e } 1 \leq j \leq n
\end{align*}
\item $V = \mathcal{F}(\R) = \mathcal{F}(-\infty, +\infty)$, formado por todas as funções $f: \R \to \R$ e
\begin{align*}
(f + g)(x)
& = f(x) + g(x), \quad \forall x \in \R\\
(k \cdot f)(x)
& = k (f(x)), \quad \forall x \in \R
\end{align*}
\item $V = \C$, formado por todos os números complexos e
\begin{align*}
(a + b \ii) + (c + d \ii)
& = (a + c) + (b + d) \ii \\
k \cdot (a + b \ii)
& = (ka) + (kb) \ii
\end{align*}
\item $V = \R^\infty$, formado por todas as sequências infinitas de números reais, e
\begin{align*}
(x_1, x_2, \ldots) + (y_1, y_2, \ldots)
& = (x_1 + y_1, x_2 + y_2, \ldots)\\
k \cdot (x_1, \ldots, x_n)
& = (kx_1, kx_2, \ldots)
\end{align*}
\end{enumerate}

\item \conceito Verifique se as operações definidas a seguir fazem com que os conjuntos $V$ indicados sejam espaços vetoriais. Em particular, determine se há algum elemento $z \in V$ que faz o papel de ``vetor nulo'', ou seja, que satisfaz $z + v = v, \forall v \in V$, e para cada vetor $v \in V$, indique qual $x \in V$ é o seu ``oposto'', no sentido de que $x + v = z$.
\begin{enumerate}
\item $V = \R$, sendo
$\begin{cases}
x + y
= x + y + 1, \quad \forall x \in V, y \in V \\
k \cdot x
= k x + k - 1, \quad \forall k \in \R, \forall x \in V
\end{cases}$
\item $V = \R^2$, sendo
$\begin{cases}
(x, y) + (u,v)
= (x+u, y+v), \quad \forall (x, y) \in V, \forall (u, v) \in V \\
k \cdot (x,y)
= ( -3 k x + k y, 5 k x - 2 k y ), \quad \forall k \in \R, \forall (x, y) \in V
\end{cases}$
\item $V = \R^2$, sendo
$\begin{cases}
(x, y) + (u, v)
 = (x + u + 1, y + v - 2), \quad \forall (x, y) \in V, \forall (u, v) \in V \\
k \cdot (x,y)
 = ( k x + k - 1, k y - 2 k + 2 ), \quad \forall k \in \R, \forall (x, y) \in V
\end{cases}$
\end{enumerate}

\item \conceito Seja $V = \R^4$, o espaço vetorial das quádruplas de números reais, com as operações usuais. Mostre que os seguintes subconjuntos são subespaços vetoriais de $V$:
\begin{enumerate}
\item $U = \{ (a,b,c,d) \in \R^4 \mid a - b = c + d \}$.
\item $W = \{ (a,b,c,d) \in \R^4 \mid c = d = 0 \}$.
\end{enumerate}

\item \conceito Seja $V = M_{3 \times 3}(\R)$, o espaço vetorial das matrizes quadradas de ordem $3$, com as operações usuais. Mostre que os seguintes subconjuntos são subespaços vetoriais de $V$:
\begin{enumerate}
\item As matrizes simétricas: $S = \{ X \in M_{3 \times 3}(\R) \mid X^T = X \}$.
\item As matrizes antissimétricas: $A = \{ X \in M_{3 \times 3}(\R) \mid X^T = -X \}$.
\item As matrizes que comutam com uma matriz $B$ fixada: $C = \{ X \in V \mid BX = XB \}$.
\item $U = \{ A = (a_{ij}) \in M_{3 \times 3}(\R) \mid a_{ij} = 0 \text{ sempre que } i + j \neq 4 \}$.
\end{enumerate}

\item \conceito Seja $V = \mathcal{F}(\R)$. Mostre que os seguintes subconjuntos são subespaços vetoriais de $V$:
\begin{enumerate}
\item O conjunto $P_\infty$ formado por todas as funções polinomiais (de qualquer grau):
\[
P_\infty = \{ p(x) \in \mathcal{F}(\R) \mid p(x) = a_n x^n + \ldots + a_1 x^1 + a_0, \text{ para algum natural } n, \text{ e } a_i \in \R \}.
\]
\item Dado um natural $n$, o conjunto $P_n$ de todos polinômios de grau menor ou igual a $n$:
\[
P_n = \{ q(x) \in P_\infty \mid \operatorname{grau}(q(x)) \leq n \}.
\]
\item As funções pares, isto é, tais que $f(-x) = f(x)$, para todo $x \in \R$.
\item As funções ímpares, isto é, tais que $f(-x) = -f(x)$, para todo $x \in \R$.
\item O conjunto $C^0(\R)$ das funções contínuas
\item O conjunto $C^1(\R)$ das funções deriváveis cuja derivada é contínua
\end{enumerate}

\item \calc Em cada item, determine se $U$ é ou não um subespaço do espaço vetorial $V$ indicado:
\begin{enumerate}
\item $V = \R^2$ e $U = \{ (x, y) \in \R^2 \mid y+2x = 5 \}$
\item $V = \R^2$ e $U = \{ (x, y) \in \R^2 \mid y = -4x \}$
\item $V = M_{3 \times 1}(\R)$, $U = \{ X \in M_{3 \times 1}(\R) \mid AX = 0 \}$, sendo $A = \begin{bmatrix}
1 & 2 & 3 \\4 & 5 & 6
\end{bmatrix}$.
\item $V = M_{2 \times 2}(\R)$ e $U = \{ Z \in M_{2 \times 2}(\R) \mid Z \text{ tem zeros em sua diagonal} \}$
\item $V = M_{2 \times 2}(\R)$ e $U = \{ B \in M_{2 \times 2}(\R) \mid \det(B) \neq 0 \}$
\item $V = \mathcal{F}(\R)$ e $U = \{ f \in \mathcal{F}(\R) \mid f \text{ é um polinômio de grau exatamente igual a 3} \}$
\item $V = \mathcal{F}(\R)$ e $U = \{ f \in \mathcal{F}(\R) \mid f^\prime(-1) = f^\prime(1) = 0 \}$
\end{enumerate}

\item \calc Sejam $A = \begin{bmatrix}
11 & 15 & -1 \\
 3 &  7 & 16
\end{bmatrix}$, $B = \begin{bmatrix}
2 & 11 & -18  \\
1 & -8 &  1
\end{bmatrix}$ e $C = \begin{bmatrix}
-3 & -6 & 6\\
-2 &  4 & 2
\end{bmatrix}$ vetores de $M_{3 \times 2}(\R)$. Resolva a seguinte equação na variável $X \in M_{3 \times 2}(\R)$: $\displaystyle \frac{A-X}{5}-\frac{X+B}{2} = C$.

\item \teoria Prove que os únicos subespaços de $\R$ (com as operações usuais) são o trivial $\{ 0 \}$ e $\R$. (dica: mostre que se $U$ é um subespaço de $\R$ e $U \neq \{0\}$ então $U = \R$)

\item \teoria Seja $U$ um subespaço de um espaço vetorial $V$. Explique por que $W = \{ x \in V \mid x \not \in U \}$ não pode ser um subespaço de $V$.

\item \calc Para cada um dos subespaços vetoriais $U$ e $V$ a seguir, obtenha o subespaço $U \cap V$:
\begin{enumerate}
\item $U = \{ ax^3 + bx^2 + cx + d \in P_3(\R) \mid 3a-b+c+d = 0 \}$ e $V = \{ p(x) \in P_3(\R) \mid p^\prime(-1) = 0 \}$
\item $U = \{ g \in \mathcal{F}(\R) \mid g \text{ é par} \}$ e $V = \{ g \in \mathcal{F}(\R) \mid g \text{ é ímpar} \}$
\item $U = \{ M \in M_{n \times n}(\R) \mid M \text{ é antissimétrica} \}$ e $V = \{ M \in M_{n \times n}(\R) \mid M \text{ é simétrica} \}$
\end{enumerate}

\item \teoria Seja $V$ um espaço vetorial qualquer e $U$ um subespaço de $V$. Prove que para quaisquer vetores $x,y \in V$ vale o seguinte: se $x\in U$ e $y-x \in U$ então $y \in U$. 

\item \conceito Dê exemplos de subespaços vetoriais $V_1$ e $V_2$ de $\R^3$ que satisfaçam as seguintes condições (e explique porque satisfazem):
\begin{enumerate}
\item $V_1 \cup V_2$ não é subespaço vetorial de $\R^3$
\item $V_1 \cup V_2$ é subespaço vetorial de $\R^3$
\end{enumerate}

\item \teoria Sejam $V_1$ e $V_2$ quaisquer subespaços de um espaço vetorial $W$ arbitrário. Prove que para $V_1 \cup V_2$ ser subespaço vetorial de $W$ é preciso que $V_1 \subseteq V_2$ ou que $V_2 \subseteq V_1$.

\end{enumerate}

\newpage
\section*{Respostas}

\begin{enumerate}
\item Todos os axiomas de espaço vetorial devem ser checados individualmente. A comutatividade da adição, por exemplo, seria verificada como segue:
\begin{enumerate}
\item
$(x_1, \ldots, x_n) + (y_1, \ldots, y_n)
= (x_1 + y_1, \ldots, x_n + y_n) \\
= (y_1 + x_1, \ldots, y_n + x_n)
= (y_1, \ldots, y_n) + (x_1, \ldots, x_n)$.

O vetor nulo é a sequência com as $n$ coordenadas iguais a zero: $0 = (0, \ldots, 0)$.

\item
$(A + B)_{ij}
= (A)_{ij} + (B)_{ij}
= (B)_{ij} + (A)_{ij}
= (B + A)_{ij}$.

O vetor nulo é a matriz nula $0 = 0_{m \times n}$ em que $0_{ij} = 0, \forall i,j$. 

\item 
$(f + g)(x)
= f(x) + g(x)
= g(x) + f(x)
= (g + f)(x), \forall x \in \R$.

O vetor nulo é a função constante igual a zero: $f(x) = 0, \forall x \in \R$.

\item 
$(a + b \ii)+ (c + d \ii)
= (a + c) + (b + d) \ii
= (c + a) + (d + b) \ii
= (c + d \ii) + (a + b \ii)$.

O vetor nulo é o número complexo (e, de fato, real) $0 = 0 + 0 \ii$.

\item 
$(x_1, x_2, \ldots) + (y_1, y_2, \ldots)
= (x_1 + y_1, x_2 + y_2, \ldots) \\
= (y_1 + x_1, y_2 + x_2, \ldots)
= (y_1, y_2, \ldots) + (x_1, x_2, \ldots)$.

O vetor nulo é a sequência infinita com todas as posições iguais a zero: $0 = (0, 0, \ldots)$.

\end{enumerate}

\item 
\begin{enumerate}
\item $\R$ é um espaço vetorial com as operações indicadas. O vetor nulo é $-1$ e $-x-2$ é o oposto do vetor $x$.
\item $\R^2$ não é espaço vetorial com as operações indicadas, pois
\[1 \cdot(x,y) = (-3x+y,5x-2y) \neq (x,y), \text{ para vários valores de }x,y.\]
\item $\R^2$ é um espaço vetorial com as operações indicadas. O vetor nulo é $(-1, 2)$ e o vetor oposto de $(x,y)$ é $(-x-2, -y+4)$.
\end{enumerate}

\item 
\begin{enumerate}
\item Se $u = (a,b,c,d) \in U$ e $v = (p,q,r,s) \in U$, então $a-b = c+d$ e $p-q = r+s$. Assim, como $u + v = (a+p,b+q,c+r,d+s)$ tem-se
\[
(a+p) - (b+q)
= (a-b) + (p-q)
= (c+d) + (r+s)
= (c+r) + (d+s),
\]
ou seja, $u + v \in \R$. Do mesmo modo, se $k \in \R$, tem-se $k u = (ka, kb, kc, kd)$ e
\[
ka - kb
= k(a-b)
= k(c+d)
= kc+kd,
\]
ou seja, $ku \in \R$. Note ainda que o vetor $0 = (0,0,0,0) \in U$.
\item $W = \{ (a,b,c,d) \in \R^4 \mid c = d = 0 \}$.
Dados $(a,b,0,0) \in W$ e $(c,d,0,0) \in W$, tem-se
\begin{align*}
(a,b,0,0) + (c,d,0,0)
& = (a+c,b+d,0+0,0+0)
  = (a+c,b+d,0,0)
\in W \text{ e} \\
 k( a,  b, 0, 0)
& = (ka, kb, k0, k0)
  = (ka, kb, 0, 0)
\in W
\end{align*}
\end{enumerate}
\item 
\begin{enumerate}
\item Se $X$ e $Y$ são matrizes simétricas então $X^T = X$ e $Y^T = Y$. Consequentemente, $(X+Y)^T = X^T+Y^T = X + Y$, ou seja, $X+Y$ é simétrica. Dado $k \in \R$, $(kX)^T = k(X^T) = kX$, ou seja, $kX$ é simétrica. Note que a matriz nula também é simétrica.

\item Se $X$ e $Y$ são matrizes antissimétricas então $X^T = -X$ e $Y^T = -Y$. Consequentemente, $(X+Y)^T = X^T+Y^T = -X + (-Y) = -(X+Y)$, ou seja, $X+Y$ é antissimétrica. Dado $k \in \R$, $(kX)^T = k(X^T) = k(-X) = -(kX)$, ou seja, $kX$ é antissimétrica. Note que a matriz nula também é antissimétrica.
\item Se $X$ e $Y$ comutam com $B$, então $BX = XB$ e $BY = YB$. Consequentemente, $B(X+Y) = BX+BY = XB + YB = (X+Y)B$, ou seja, $X+Y$ comuta com $B$. Dado $k \in \R$, $B(kX) = k(BX) = k(XB) = (kX)B$, ou seja, $kX$ comuta com $B$. Note que $B0 = 0B = 0$, isto é, a matriz nula também comuta com $B$.

\item
$U = \{ A = (a_{ij}) \in M_{3 \times 3}(\R) \mid a_{ij} = 0 \text{ sempre que } i + j \neq 4 \}$.
A condição que define o conjunto $U$ indica que para toda matriz $A \in U$ é da forma $A = \begin{bmatrix}
     0 &      0 & a_{13}\\
     0 & a_{22} & 0     \\
a_{31} &      0 & 0
\end{bmatrix}$, pois $a_{11} = a_{12} = a_{21} = a_{23} = a_{32} = a_{33} = 0$ (em particular, a matriz nula $3 \times 3$ está em $U$). Além disso, ao somar $A \in U$ com $B \in U$, as entradas $[A+B]_{ij}$ em que $i+j \neq 4$ serão iguais a zero pois
\[[A+B]_{ij} = a_{ij} + b_{ij} = 0 + 0 = 0.\]
Logo, $A+B \in U$, e este é um conjunto fechado para a adição. Em relação à multiplicação por um escalar $k \in \R$, tem-se:
\[[kA]_{ij} = ka_{ij} = k0 = 0,\]
ou seja, $kA \in U$ (a multiplicação por escalar é fechada em $U$), e conclui-se que $U$ é um subespaço de $M_{3 \times 3} (\R)$.
\end{enumerate}
\item 
\begin{enumerate}
\item
Se $p$ e $q$ são polinômios e $c \in \R$, a soma $p+q$ e o produto $c \cdot p$ também são polinômios, logo $p + q \in P_\infty$ e $cp \in P_\infty$. Mais precisamente, se $p(x) = a_n x^n + \ldots + a_1 x^1 + a_0$ e $q(x) = b_m x^m + \ldots + b_1 x^1 + b_0$, com $m, n \in \N$ e $a_i,b_i \in \R$ então $p + q$ é dado por
\begin{align*}
(p + q)(x)
= p(x)+q(x)
& = (a_n x^n + \ldots + a_1 x^1 + a_0)
  + (b_m x^m + \ldots + b_1 x^1 + b_0) \\
& = (a_k + b_k) x^k + \ldots + (a_1+b_1) x^1 + (a_0+b_0) \in P_\infty,
\end{align*}
onde $k$ é o maior dos graus $m$ e $n$. De forma análoga,
\begin{align*}
(c \cdot p)(x)
= c \cdot p(x)
& = c(a_n x^n + \ldots + a_1 x^1 + a_0) \\
& = (ca_n) x^n + \ldots + (ca_1) x^1 + (ca_0) \in P_\infty.
\end{align*}

Perceba que o polinômio nulo ($p(x) = 0$) é o vetor nulo neste espaço vetorial.

\item Dado um natural $n$, o conjunto $P_n$ de todos polinômios de grau menor ou igual a $n$:
\[
P_n = \{ p(x) \in P_\infty \mid \operatorname{grau}(p(x)) \leq n \}.
\]

Pelo exercício anterior, a soma de polinômios resulta em um polinômio, e a multiplicação de um polinômio por um escalar real qualquer também é um polinômio. Levando isso em conta, para ver que $P_n$ é um subespaço vetorial de $\mathcal{F}(\R)$ basta lembrar que se $p, q \in P_n$ então:
\begin{itemize}
\item $\operatorname{grau}(p+q) < \operatorname{grau}(p) + \operatorname{grau}(q)$, se os termos de maior grau são opostos
\item $\operatorname{grau}(p+q) = \operatorname{grau}(p) + \operatorname{grau}(q)$, nos demais casos
\item $\operatorname{grau}(c \cdot p) = \operatorname{grau}(p)$ se $c \neq 0$;
\item $\operatorname{grau}(c \cdot p) = \operatorname{grau}(0) < \operatorname{grau}(p), \text{ se } c = 0$
\end{itemize}
Este último item também mostra que o polinômio nulo é um elemento de $P_n$.
\item Se $f$ e $g$ são funções pares, então para todo $x \in \R$ tem-se
\[
(f+g)(-x)
= f(-x) + g(-x)
= f(x) + g(x)
= (f + g)(x),
\]
ou seja, $f+g$ também é par. Analogamente, dado qualquer $c \in \R$,
\[
(cf)(-x)
= cf(-x)
= cf(x)
= (cf)(x),
\]
e portanto $cf$ também é par, e o conjunto dado é um subespaço de $\mathcal{F}(\R)$. A função constante igual a zero, que é o vetor nulo deste espaço, é uma função par.

\item Se $f$ e $g$ são funções ímpares, então para todo $x \in \R$ tem-se
\[
(f+g)(-x)
= f(-x) + g(-x)
= (-f(x)) + (-g(x))
= -(f(x) + g(x))
= -(f + g)(x),
\]
ou seja, $f+g$ também é ímpar. Analogamente, dado qualquer $c \in \R$,
\[
(cf)(-x)
= cf(-x)
= c(-f(x))
= -c(f(x))
= -(cf)(x),
\]
e portanto $cf$ também é ímpar e o conjunto é fechado para ambas as operações do espaço vetorial, ou seja, é um subespaço de $\mathcal{F}(\R)$. A função constante igual a zero também é uma função ímpar.

\item Como a soma de funções contínuas é contínua e a multiplicação de funções contínuas por constantes resulta em funções contínuas, conclui-se que $C^0(\R)$ é fechado para ambas as operações do espaço vetorial, e consequentemente é um subespaço de $\mathcal{F}(\R)$.

\item Sejam $f,g \in C^1(\R)$ e $c \in \R$. Então:
\begin{itemize}
\item As funções $f$ e $g$ são deriváveis e suas derivadas são contínuas. Como a soma de funções deriváveis é uma função derivável, $f+g$ também é derivável e vale $(f+g)' = f^\prime + g^\prime$. Além disso, pelo exercício anterior $f^\prime+g^\prime$ é contínua, pois é soma de funções contínuas. Logo, $f+g$ é uma função derivável cuja derivada é contínua, ou seja, $f+g \in C^1(\R)$.
\item Sendo $f$ derivável, a função $cf$ também é derivável e $(cf)^\prime = c (f^\prime)$. Mas $f^\prime$ é contínua, então $c (f^\prime)$ também é (pelo exercício anterior), ou seja, $cf \in C^1(\R)$.
\end{itemize}
Como as duas operações são fechadas, segue que $C^\prime(\R)$ é um subespaço de $\mathcal{F}(\R)$.
\end{enumerate}

\item
\begin{enumerate}
\item Como $(0,0)$ é o vetor nulo de $\R^2$ e $(0,0) \not\in U$, pois $0+2\cdot 0 \neq 5$, este conjunto $U$ não é um subespaço vetorial de $\R^2$ (geometricamente, é uma reta que não passa pela origem). Note que a adição e a multiplicação por escalar não são fechadas em $U$. Por exemplo, $(0,5)+(3,-1)=(3,4) \not\in U$ pois $3+2\cdot (-1) \neq 5$.

\item Sejam $(a,b) \in U$, $(c,d) \in U$ e $k \in \R$. Então
\begin{itemize}
\item $(a,b)+(c,d) = (a+c,b+d) \in U$ pois $b+d=(-4a)+(-4c)=-4(a+c)$.
\item $k(a,b) = (ka,kb) \in U$ pois $kb=k(-4a)=-4(ka)$.
\end{itemize}
Portanto, $U$ é um subespaço vetorial de $\R^2$ (é a reta que passa pela origem e tem a direção do vetor $(1,-4)$)

\item $U = \{ X \in M_{3 \times 1}(\R) \mid AX = 0 \}$ é um subespaço vetorial de $V = M_{3 \times 1}(\R)$ pois
\begin{itemize}
\item $A0 = 0$, ou seja, a matriz nula $3 \times 1$ pertence a $U$.
\item Se $AX_1=0$ e $AX_2=0$ então $A(X_1+X_2)=AX_1 + AX_2 = 0 + 0$.
\item Se $AX=0$ e $t \in \R$ então $A(tX)=t(AX) = t0 = 0$.
\end{itemize}
Em outras palavras, $U$ é fechado para as operações do espaço vetorial $V$. 

Note que as propriedades acima são válidas independentemente de quais sejam os números que aparecem na matriz $A$, então não é necessário expandir os produtos em termos das entradas das matrizes. De fato, $U = N(A)$, isto é, é o espaço nulo de $A$.

\item Note que $U = \left\{ \begin{bmatrix}
0 & z_{12}\\ z_{21} & 0
\end{bmatrix} \mid z_{12},z_{21} \in \R \right\}$ e que para todo $X, Y \in U$ e $m \in \R$ tem-se:
\begin{itemize}
\item $X + Y = \begin{bmatrix}
0 & x_{12}\\ x_{21} & 0
\end{bmatrix} + \begin{bmatrix}
0 & y_{12}\\ y_{21} & 0
\end{bmatrix}
=\begin{bmatrix}
0 & x_{12} + y_{12}\\ x_{21}+y_{21} & 0
\end{bmatrix} \in U$
\item $mX = m\begin{bmatrix}
0 & x_{12}\\ x_{21} & 0
\end{bmatrix}
=\begin{bmatrix}
0 & mx_{12} \\ mx_{21} & 0
\end{bmatrix} \in U$
\end{itemize}

Portanto, $U$ é um subespaço vetorial de $V$.

\item O conjunto $U$ consiste das matrizes $2 \times 2$ que são inversíveis. Como $I$ e $-I$ são matrizes inversíveis ($\det(I) = \det(-I) = 1 \neq 0$), e sua soma $I + (-I) = 0$ não é inversível (pois $\det(0) = 0$), conclui-se que $U$ não é fechado para a adição, e como tal não é um subespaço de $V$. A multiplicação por escalar também não é fechada em $U$, pois a ao multiplicar uma matriz inversível por zero o resultado é a matriz nula, que não pertence a $U$.
\item Os polinômios de grau 3 não formam um espalo vetorial pois, por exemplo, se $p(x) = 4x^3+5x$ e $q(x) = -4x^3$ então $p(x)+q(x) = 5x$ que  não tem grau 3.

\item Sejam $f,g \in U$. Então $f^\prime(-1) = f^\prime(1) = 0$ e $g^\prime(-1) = g^\prime(1) = 0$. Consequentemente:
\begin{itemize}
\item $(f+g)^\prime(1) = f^\prime(1) + g^\prime(1) = 0 + 0 = 0$
\item $(f+g)^\prime(-1) = f^\prime(-1) + g^\prime(-1) = 0 + 0 = 0$
\end{itemize}
Isto significa que a adição é fechada em $U$. De forma análoga, para todo $c \in \R$ vale:
\begin{itemize}
\item $(cf)^\prime(1) = c(f^\prime(1)) = c \cdot 0 = 0$
\item $(cf)^\prime(-1) = c(f^\prime(-1)) = c \cdot 0 = 0$
\end{itemize}
Isto significa que a multiplicação por escalar também é fechada, e assim $U$ é um subespaço vetorial de $\mathcal{F}(\R)$.
\end{enumerate}

\item Sabendo que $A,B$ e $X$ são elementos do espaço vetorial $M_{3 \times 2}$, pode-se utilizar as propriedades deste espaço (associatividade, comutatividade, etc) para ``isolar a incógnita $X$'', de forma análoga ao que seria feito se $A,B$ e $X$ fossem números reais:
\begin{align*}
\frac{A-X}{5}-\frac{X+B}{2} = C
& \Leftrightarrow
10\left(\frac{A-X}{5}-\frac{X+B}{2}\right) = 10 C
\Leftrightarrow
2(A-X)-5(X+B) = 10 C\\
& \Leftrightarrow
2A - 2X - 5X - 5B = 10 C
\Leftrightarrow
-7X = -2A + 10C + 5B\\
& \Leftrightarrow
X = \frac{1}{7}(2A - 10C - 5B)
\end{align*}
Assim, $X = \frac{1}{7}\left(2\begin{bmatrix}
11 & 15 & -1 \\
 3 &  7 & 16
\end{bmatrix} - 10\begin{bmatrix}
-3 & -6 & 6\\
-2 &  4 & 2
\end{bmatrix} - 5\begin{bmatrix}
2 & 11 & -18  \\
1 & -8 &  1
\end{bmatrix}\right)
=\begin{bmatrix}
6 & 5 & 4  \\
3 & 2 & 1
\end{bmatrix}$

\item Se $U$ é um subespaço de $\R$ e $U \neq \{0\}$ então $U$ tem algum elemento não nulo. Denote-o por $b$. Para ver que todo número real $x$ também tem que estar em $U$ basta escrever $x = (\frac{x}{b}) \cdot b$, pois sendo $U$ um subespaço vetorial de $\R$ ele é fechado para a multiplicação por escalar e portanto a multiplicação de $b \in U$ pelo escalar $\frac{x}{b}$ deve resultar em um elemento de $U$, isto é, $x \in U$. Portanto $U = \R$.

\item O conjunto $W = \{ x \in V \mid x \not \in U \}$ não é um subespaço de $V$ pois $0 \in U$ mas $0 \not \in W$.

\item
\begin{enumerate}
\item Primeiramente lembre-se que se $p(x) = ax^3 + bx^2 + cx + d$ então $p^\prime(x) = 3ax^2 + 2bx + c$ e $p^\prime(-1) = 3a - 2b + c$. Assim,
\[
U \cap V = \{ ax^3 + bx^2 + cx + d \in P_3(\R) \mid
3a -  b + c + d = 0 \text{ e }
3a - 2b + c     = 0 \}.
\]
Em outras palavras, os coeficientes de $p(x)$ são soluções do sistema linear homogêneo
\systeme{
3a -  b + c + d = 0,
3a - 2b + c     = 0
}

Escalonando a matriz associada ao sistema, obtem-se:
\begin{align*}
&
\begin{bmatrix}
3 & -1 & 1 & 1\\
3 & -2 & 1 & 0
\end{bmatrix}
\grstep{ \frac{1}{3} L_1 }
\begin{bmatrix}
1 & -1/3 & 1/3 & 1/3 \\
3 & -2 & 1 & 0
\end{bmatrix}
\grstep{ L_2 - 3L_1 }
\begin{bmatrix}
1 & -1/3 & 1/3 & 1/3 \\
0 & -1 & 0 & -1
\end{bmatrix}\\
\grstep{ -L_2 }
&
\begin{bmatrix}
1 & -1/3 & 1/3 & 1/3 \\
0 & 1 & 0 & 1
\end{bmatrix}
\grstep{ L_1 +\frac{1}{3}L_2 }
\begin{bmatrix}
1 & 0 & 1/3 & 2/3 \\
0 & 1 & 0 & 1
\end{bmatrix}
\end{align*}
ou seja, $a = \frac{1}{3} (-c-2 d)$ e $b = -d$, em que $c,d \in \R$ são variáveis livres. Assim,
\begin{align*}
U \cap V
& = \left\{ ax^3 + bx^2 + cx + d \in P_3(\R) \mid
a = \frac{-c-2 d}{3}
\text{ e }
b = -d, \text{ sendo } c,d \in \R \right\} \\
& = \left\{ \left(\frac{-c-2 d}{3}\right)x^3 + (-d)x^2 + cx + d \mid c,d \in \R \right\}.
\end{align*}

\item Se $g \in U$ então para todo $x \in \R$ ocorre $g(-x) = g(x)$ e se, além disso, $g \in V$ então para todo $x \in \R$ vale $g(-x) = -g(x)$. Neste caso, para todo $x \in \R$ tem-se $g(x) = g(-x) = -g(x)$, ou seja $2g(x) = 0$ e portanto $g(x) = 0$.

Assim, a única função que está simultaneamente em $U$ e em $V$ é a função constante igual a zero, que é o vetor nulo de $\mathcal{F}(\R)$, isto é, $U \cap V = \{ 0 \}$.

\item Se $M \in U$ então $M^T = M$ e se, além disso, $M \in V$ então $M^T = -M$. Neste caso, tem-se $M = M^T = -M$, ou seja $2M = 0$ e portanto $M = 0$.

Assim, a única matriz que é simétrica e antissimétrica é a matriz nula, ou seja, $U \cap V = \{ 0 \}$.
\end{enumerate}


\item Para quaisquer $x, y \in V$ tem-se $y = (y - x) + x$. Então se $x \in U$ e $y-x \in U$ e $U$ é um subespaço de $V$, então $U$ é fechado para a adição, e consequentemente $y = (y - x) + x \in U$.

\item Dê exemplos de subespaços vetoriais $V_1$ e $V_2$ de $\R^3$ que satisfaçam as seguintes condições (e explique porque satisfazem):
\begin{enumerate}
\item Sejam
\begin{itemize}
\item $V_1 = \{ (x,y,z) \in \R^3 \mid z=0 \}$
\item $V_2 = \{ (x,y,z) \in \R^3 \mid x=0 \}$
\end{itemize}
Então o conjunto $V_1 \cup V_2 = \{ (x,y,z) \in \R^3 \mid z=0 \text{ ou } x=0 \}$ não é um subespaço de $\R^3$ pois $(1,1,0) \in V_1 \cup V_2$ e $(0,1,1) \in V_1 \cup V_2$ mas $(1,1,0) + (0,1,1) = (1,2,1) \not \in V_1 \cup V_2$.

\item Sejam
\begin{itemize}
\item $V_1 = \R^3$
\item $V_2 = \{ (0,0,0) \}$
\end{itemize}
Então $V_1 \cup V_2 = \R^3 \cup \{ (0,0,0) \} = \R^3$, que é, naturalmente, um subespaço de $\R^3$.

\end{enumerate}


\item Suponha que $V_1$ e $V_2$ são subespaços de $W$ tais que $V_1 \cup V_2$ também é subespaço de $W$. Se $V_1$ não estiver contido em $V_2$, é porque existe algum vetor $x_1 \in V_1$ que não pertence a $V_2$, e pode-se deduzir que $V_2 \subseteq V_1$ da seguinte forma: dado um vetor $x_2 \in V_2$ arbitrário, tanto $x_1$ quanto $x_2$ são elementos de $V_1 \cup V_2$. Assumindo que esta união seja um subespaço, a soma $x_1 + x_2$ deve pertencer a $V_1 \cup V_2$. Ou seja, $x_1 + x_2$ deve ser elemento de $V_1$ ou de $V_2$:
\begin{enumerate}
\item Para que $x_1 + x_2$ fosse elemento de $V_2$ seria preciso que $x_1 = (x_1 + x_2) - x_2 \in V_2$ (pois $V_2$ é subespaço). Mas foi assumido que $x_1 \not \in V_2$, então só resta a outra possibilidade;
\item Sendo $x_1 + x_2$ um elemento de $V_1$, resulta que $x_2 = (x_1 + x_2) - x_1 \in V_1$.
\end{enumerate}
Em outras palavras, todo elemento $x_2$ de $V_2$ tem que pertencer a $V_1$, isto é, $V_2 \subseteq V_1$.
\end{enumerate}

\end{document}
