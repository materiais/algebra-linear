\documentclass[12pt,a4paper]{article}
\usepackage{cmap} % Makes the PDF copiable. See http://tex.stackexchange.com/a/64198/25761
\usepackage[T1]{fontenc}
\usepackage[brazil]{babel}
\usepackage[utf8]{inputenc}
\usepackage{amsmath}
\usepackage{amsfonts}
\usepackage{amssymb}
\usepackage{amsthm}
\usepackage{textcomp} % \degree
\usepackage{gensymb} % \degree
\usepackage[usenames,svgnames,dvipsnames]{xcolor}
\usepackage{hyperref}
\usepackage{multicol}
\usepackage{graphicx}
\usepackage[margin=2cm]{geometry}
\usepackage{systeme}

\hypersetup{
    colorlinks = true,
    allcolors = {blue}
}

\newcommand{\fixme}{{\color{red}(...)}}
\newcommand*\sen{\operatorname{sen}}

\newcommand*\R{\mathbb{R}}

\newcommand{\IconPc}{\includegraphics[width=1em]{computer.png}}
\newcommand{\IconCalc}{\includegraphics[width=1em]{calculator.png}}
\newcommand{\IconThink}{\includegraphics[width=1em]{pencil.png}}
\newcommand{\IconCheck}{\includegraphics[width=1em]{checkmark.png}}
\newcommand{\IconConcept}{\includegraphics[width=1em]{edit.png}}

\newlength{\SmileysLength}
\setlength{\SmileysLength}{\labelwidth}\addtolength{\SmileysLength}{\labelsep}

\newcommand{\calc}{\hspace*{-\SmileysLength}\makebox[0pt][r]{\IconCalc}%
   \hspace*{\SmileysLength}}
\newcommand{\software}{\hspace*{-\SmileysLength}\makebox[0pt][r]{\IconPc}%
   \hspace*{\SmileysLength}}
\newcommand{\teoria}{\hspace*{-\SmileysLength}\makebox[0pt][r]{\IconThink}%
   \hspace*{\SmileysLength}}
\newcommand{\conceito}{\hspace*{-\SmileysLength}\makebox[0pt][r]{\IconCheck}%
   \hspace*{\SmileysLength}}
\newcommand{\concept}{\hspace*{-\SmileysLength}\makebox[0pt][r]{\IconCheck}%
   \hspace*{\SmileysLength}}

% Loop Space / CC BY-SA-3.0 / https://tex.stackexchange.com/a/2238/25761
\newenvironment{amatrix}[1]{%
  \left[\begin{array}{@{}*{#1}{c}|c@{}}
}{%
  \end{array}\right]
}

% Loop Space / CC BY-SA-3.0 / https://tex.stackexchange.com/a/3164/25761
%--------grstep
% For denoting a Gauss' reduction step.
% Use as: \grstep{\rho_1+\rho_3} or \grstep[2\rho_5 \\ 3\rho_6]{\rho_1+\rho_3}
\newcommand{\grstep}[2][\relax]{%
   \ensuremath{\mathrel{
       {\mathop{\longrightarrow}\limits^{#2\mathstrut}_{
                                     \begin{subarray}{l} #1 \end{subarray}}}}}}
\newcommand{\swap}{\leftrightarrow}

\newcommand*\tipo{Lista de Exercícios - Sistemas Lineares}
\newcommand*\eu{Helder G. G. de Lima}
\newcommand*\data{\today}

\author{\eu}
\title{\tipo}
\date{\data}

\begin{document}

\begin{center}
\includegraphics[width=9.0cm]{marca} \\
\textbf{\tipo} \\
Prof. \eu\footnote{
Este é um material de acesso livre distribuído sob os termos da licença \href{https://creativecommons.org/licenses/by-sa/4.0/deed.pt_BR}{Creative Commons BY-SA 4.0}}
\end{center}

\section*{Legenda}
\begin{multicols}{4}
\begin{itemize}
\item[] \hspace*{\SmileysLength} \calc \hspace*{-\SmileysLength} Cálculos
\item[] \hspace*{\SmileysLength} \conceito \hspace*{-\SmileysLength} Conceitos
\item[] \hspace*{\SmileysLength} \teoria \hspace*{-\SmileysLength} Teoria
\item[] \hspace*{\SmileysLength} \software \hspace*{-\SmileysLength} Software
\end{itemize}
\end{multicols}

\section*{Questões}

\begin{enumerate}
\item \calc Obtenha a forma escalonada reduzida por linhas da matriz de coeficientes de cada um dos sistemas lineares a seguir, e partir dela determine as soluções dos sistemas:
\begin{multicols}{2}
\begin{enumerate}
\item $\left\{
\begin{aligned}
5s & {}-{} &  5\pi t & {}={} & -5\pi^2\\
-s & {}+{} & (\pi+3)t & {}={} & \pi(\pi+6)
\end{aligned}
\right.%, \text{ sendo } s,t \in \R
$

\item $\systeme{
4x_1+4x_2 = 16,
5x_2 -15x_4=2,
2x_1+2x_2+x_3=12,
-x_2+8x_4=3/5
}%, \text{ sendo } x_i \in \Q
$

\item $\systeme{
            3x_1-12x_2 -6x_3              +9x_5=-21,
            -x_1 +4x_2 +2x_3              -3x_5=7,
\frac{1}{2}x_1 -2x_2  -x_3+x_4-\frac{3}{2}x_5=-\frac{5}{2},
         -7x_1+28x_2+15x_3             -23x_5=53
}%, \text{ sendo } x_i \in \R
$

\item $\systeme{
      b+6c= 6,
   a+6b-5c=-3,
3a+20b-3c= 1
}%, \text{ sendo } a, b, c \in \R
$
\end{enumerate}
\end{multicols}
\item \software Considere o sistema de equações lineares
\[
\systeme*{
2x - 3y = -4,
5x +  y = 7
}
\]
e utilize um software como o GeoGebra\footnote{\url{https://www.geogebra.org/download/}} para:
\begin{enumerate}
\item Plotar o conjunto $A$ formado pelos pontos $(x, y)$ cujas coordenadas satisfazem a primeira equação e o conjunto $B$ dos que verificam a segunda equação.

\textbf{Dica}: Não é preciso um comando especial para representar equações polinomiais no GeoGebra. Basta digitá-las diretamente (mesmo se forem como \texttt{5xy\^{}2+2y\^{}3x\^{}2=1}).
\item Alterar algumas vezes os números do segundo membro, e perceber o tipo de mudança que ocorre na representação gráfica de $A$ e $B$.
\item Verificar se com alguma escolha de valores os conjuntos se intersectam. Parece ser possível que isso não aconteça dependendo dos valores escolhidos?

\textbf{Dica}: O comando \texttt{Interseção[p, q]} gera a interseção dos objetos \texttt{p} e \texttt{q}.
\end{enumerate}

\item \software Repita o exercício anterior para o seguinte sistema, em uma janela de visualização 3D:
\[
\systeme*{
 x - y + z = 1,
2x + y + z = 4,
 x + y + 5z= 7
}
\]

\item Considere os seguintes sistemas lineares nas variáveis $x,y \in \R$:
\begin{multicols}{2}
\begin{itemize}
\item[] \begin{equation}
\begin{cases}
 x + 2y &= 6 \\
 2x - c y &= 0
\end{cases}\end{equation}

\item[] \begin{equation}
\systeme*{
   x + 2y = 6,
-c x +  y = 1 - 4c
} \end{equation}
\end{itemize}
\end{multicols}
\begin{enumerate}
\item \calc Determinar para quais valores de $c$ os sistemas lineares têm uma, nenhuma ou infinitas soluções.
\item \software Obtenha as mesmas conclusões sobre $c$ experimentalmente, usando o GeoGebra. \textbf{Dica}: defina por exemplo \texttt{c=10} e use o botão direito do mouse para tornar o número visível como um ``controle deslizante'' e mova-o para ver o efeito deste parâmetro.
\end{enumerate}

\item \teoria Seja $M = \begin{bmatrix}a & b \\ c & d\end{bmatrix} \in M_{2 \times 2} (\R)$ a matriz associada a um sistema linear homogêneo. Utilize a eliminação de Gauss-Jordan para provar que se $ad-bc \neq 0$ então o sistema possui somente a solução trivial.

\end{enumerate}

\newpage
\section*{Respostas}
\begin{enumerate}
\item \begin{enumerate}
\item A matriz aumentada associada ao sistema dado é $A = \begin{amatrix}{2}
   5 & -5\pi & -5\pi^2 \\
-1 & \pi+3 & \pi(\pi + 6)
\end{amatrix}$
e sua forma escalonada reduzida é obtida por meio das seguintes operações elementares sobre as linhas:
\begin{align*}
A
&
\grstep{ \frac{1}{5} L_1 }
\begin{amatrix}{2}
   1 & -\pi & -\pi^2 \\
-1 & \pi+3 & \pi(\pi + 6)
\end{amatrix}
\grstep{ L_2 + L_1 }
\begin{amatrix}{2}
   1 & -\pi & -\pi^2 \\
   0 &    3 & 6\pi
\end{amatrix} \\
&
\grstep{ \frac{1}{3} L_2 }
\begin{amatrix}{2}
   1 & -\pi & -\pi^2 \\
   0 &    1 & 2\pi
\end{amatrix}
\grstep{ L_1 + \pi L_2 }
\begin{amatrix}{2}
   1 & 0 & \pi^2 \\
   0 & 1 & 2\pi
\end{amatrix}
\end{align*}
Esta última matriz está associada às equações
\[
\left\{
\begin{aligned}
s & = \pi^2 \\
t & = 2\pi
\end{aligned}
\right.,
\]
e, portanto, $S = \{ ( \pi^2, 2\pi ) \}$ é o conjunto das soluções do sistema proposto.

\item A redução à forma escalonada reduzida da matriz associada ao sistema é obtida através das seguintes operações elementares:
\begin{align*}
\begin{amatrix}{4}
4 &  4 & 0 &   0 & 16 \\
0 &  5 & 0 & -15 & 2 \\
2 &  2 & 1 &   0 & 12 \\
0 & -1 & 0 &   8 & 3/5
\end{amatrix}
&
\grstep{ \frac{1}{4} L_1 }
\begin{amatrix}{4}
1 &  1 & 0 &   0 & 4 \\
0 &  5 & 0 & -15 & 2 \\
2 &  2 & 1 &   0 & 12 \\
0 & -1 & 0 &   8 & 3/5
\end{amatrix}
\grstep{ L_3 - 2 L_1 }
\begin{amatrix}{4}
1 &  1 & 0 &   0 & 4 \\
0 &  5 & 0 & -15 & 2 \\
0 &  0 & 1 &   0 & 4 \\
0 & -1 & 0 &   8 & 3/5
\end{amatrix} \\
\grstep{ \frac{1}{5} L_2 }
\begin{amatrix}{4}
1 &  1 & 0 &   0 & 4 \\
0 &  1 & 0 & -3 & 2/5 \\
0 &  0 & 1 &   0 & 4 \\
0 & -1 & 0 &   8 & 3/5
\end{amatrix}
&
\grstep{ L_4 + L_2 }
\begin{amatrix}{4}
1 & 1 & 0 &  0 & 4 \\
0 & 1 & 0 & -3 & 2/5 \\
0 & 0 & 1 &  0 & 4 \\
0 & 0 & 0 &  5 & 1
\end{amatrix}
\grstep{ \frac{1}{5} L_4 }
\begin{amatrix}{4}
1 & 1 & 0 &  0 & 4 \\
0 & 1 & 0 & -3 & 2/5 \\
0 & 0 & 1 &  0 & 4 \\
0 & 0 & 0 &  1 & 1/5
\end{amatrix} \\
\grstep{ L_2 + 3L_4 }
\begin{amatrix}{4}
1 & 1 & 0 & 0 & 4 \\
0 & 1 & 0 & 0 & 1 \\
0 & 0 & 1 & 0 & 4 \\
0 & 0 & 0 & 1 & 1/5
\end{amatrix}
&
\grstep{ L_1 - L_2 }
\begin{amatrix}{4}
1 & 0 & 0 & 0 & 3 \\
0 & 1 & 0 & 0 & 1 \\
0 & 0 & 1 & 0 & 4 \\
0 & 0 & 0 & 1 & 1/5
\end{amatrix}
\end{align*}
Esta última matriz está associada às equações
\[
\left\{
\begin{aligned}
x_1 & = 3 \\
x_2 & = 1 \\
x_3 & = 4 \\
x_4 & = 1/5 \\
\end{aligned}
\right.,
\]
de modo que $S = \{ ( 3, 1, 4, 1/5 ) \}$ é o conjunto das soluções do sistema proposto.
\item A redução à forma escalonada reduzida da matriz associada ao sistema é obtida através das seguintes operações elementares:\begin{align*}
\begin{amatrix}{5}
   3 & -12 & -6 & 0 &    9 & -21 \\
   -1 &   4 &  2 & 0 &   -3 &  7 \\
1/2 &  -2 & -1 & 1 & -3/2 & -5/2 \\
   -7 &  28 & 15 & 0 & -23  & 53
\end{amatrix}
&
\grstep{ \frac{1}{3} L_1 }
\begin{amatrix}{5}
   1 & -4 & -2 &  0 &   3 & -7 \\
   -1 &  4 &  2 & 0 &  -3 &   7 \\
1/2 & -2 & -1 & 1 & -3/2 & -5/2 \\
   -7 & 28 & 15 & 0 & -23 & 53
\end{amatrix} \\
\grstep[ L_3 - \frac{1}{2} L_1 \\ L_4 + 7 L_1 ]{ L_2 + L_1 }
\begin{amatrix}{5}
1 & -4 & -2 &   0 &  3 & -7 \\
0 &  0 &  0 &   0 &  0 &  0 \\
0 &  0 &  0 &   1 & -3 & 1 \\
0 &  0 &  1 &   0 & -2 & 4
\end{amatrix}
&
\grstep{ L_2 \leftrightarrow L_4 }
\begin{amatrix}{5}
1 & -4 & -2 &   0 &  3 & -7 \\
0 &  0 &  1 &   0 & -2 & 4 \\
0 &  0 &  0 &   1 & -3 & 1 \\
0 &  0 &  0 &   0 &  0 &  0
\end{amatrix} \\
\grstep{ L_1 +2 L_2 }
\begin{amatrix}{5}
1 & -4 & 0 & 0 & -1 & 1 \\
0 &  0 & 1 & 0 & -2 & 4 \\
0 &  0 & 0 & 1 & -3 & 1 \\
0 &  0 & 0 & 0 &  0 &  0
\end{amatrix}
\end{align*}
Esta última matriz está associada às equações
\[
\left\{
\begin{aligned}
x_1 -4x_2-x_5 & =1 \\
x_3 -2x_5 & =4 \\
x_4 -3x_5 & =1 \\
0 & = 0.
\end{aligned}
\right.
\]

Logo, o conjunto das soluções do sistema proposto é
\begin{align*}
S
& = \{ (x_1, x_2, x_3, x_4, x_5 ) \in \R^5 \mid x_1 = 1 + 4x_2 + x_5, x_2 = 4 + 2x_5, x_4 = 1 + 3 x_5 \} \\
& = \{ (1 + 4x_2 + x_5, 4 + 2x_5, x_3, 1 + 3 x_5, x_5 ) \mid x_3,x_5 \in \R \}.
\end{align*}

\item

A matriz aumentada associada ao sistema dado é $[A|B] =
\begin{amatrix}{3}
   0 &  1 &  6 &  6 \\
   1 &  6 & -5 & -3 \\
   3 & 20 & -3 &  1
\end{amatrix}$
e sua forma escalonada reduzida é obtida por meio das seguintes operações elementares sobre as linhas:


\begin{align*}
[A|B]
\grstep{ L_1 \leftrightarrow L_2 }
\begin{amatrix}{3}
   1 &  6 & -5 & -3 \\
   0 &  1 &  6 &  6 \\
   3 & 20 & -3 &  1
\end{amatrix}
&
\grstep{ L_3 - 3 L_1 }
\begin{amatrix}{3}
   1 & 6 & -5 & -3 \\
   0 & 1 &  6 &  6 \\
   0 & 2 & 12 & 10
\end{amatrix}
\grstep{ L_3 - 2 L_1 }
\begin{amatrix}{3}
   1 & 6 & -5 & -3 \\
   0 & 1 &  6 &  6 \\
   0 & 0 &  0 & -2
\end{amatrix} \\
\grstep{ \frac{-1}{2} L_3 }
\begin{amatrix}{3}
   1 & 6 & -5 & -3 \\
   0 & 1 &  6 &  6 \\
   0 & 0 &  0 & 1
\end{amatrix}
&
\grstep[ L_1 + 3 L_3 ]{ L_2 - 6 L_3 }
\begin{amatrix}{3}
   1 & 6 & -5 & 0 \\
   0 & 1 &  6 & 0 \\
   0 & 0 &  0 & 1
\end{amatrix}
\grstep{ L_1 - 6 L_2 }
\begin{amatrix}{3}
   1 & 0 & -41 & 0 \\
   0 & 1 &  6 & 0 \\
   0 & 0 &  0 & 1
\end{amatrix}
\end{align*}
Como a última linha corresponde a uma equação da forma $0 = 1$, o sistema é impossível, ou seja, $S = \emptyset$.
\end{enumerate}

\item \begin{enumerate}
\item Digite \texttt{2x-3y=-4} para que o GeoGebra mostre a reta formada pelos pontos que satisfazem a primeira equação, e \texttt{5x+y=7} para representar a segunda reta.
\item Ao trocar o $-4$ por um número maior, a reta correspondente se desloca para baixo, mantendo-se paralela à reta original. Ao diminuir este valor, a reta se desloca paralelamente para cima. Na segunda equação, a troca de $7$ por um número maior resulta em um deslocamento para a direita, e a diminuição deste valor desloca a reta para a esquerda.

\item As retas, que inicialmente se intersectam em $(1,2)$, têm sempre um ponto em comum, independentemente dos valores atribuídos ao segundo membro das equações. Isso reflete o fato de que as duas equações correspondem a retas que não são paralelas entre si, e sua direção permanece inalterada mesmo quando o segundo membro é modificado.
\end{enumerate}


\item
\begin{enumerate}
\item Digite \texttt{x-y+z=1} para que o GeoGebra mostre o plano formado pelos pontos $(x,y,z)$ que satisfazem a primeira equação, e então \texttt{2x+y+z=4} e \texttt{x+y+5z=7} para representar os planos correspondentes às demais equações.
\item A trocar os valores do segundo membro de cada equação, o plano correspondente desloca-se no espaço mantendo-se paralelo à sua posição original.

\item Como os planos se intersectam inicialmente no ponto $(1,1,1)$, e sempre permanecem paralelos às suas posições iniciais, continua existindo um único ponto de interseção, quaisquer que sejam os valores do segundo membro do sistema.
\end{enumerate}

\item
\begin{enumerate}
\item
\begin{enumerate}
\item A matriz aumentada associada ao primeiro sistema pode ser levada à sua forma escalonada reduzida por linhas por meio das seguintes operações elementares:
\[
\begin{amatrix}{2}
1 & 2 & 6 \\
2 & -c & 0
\end{amatrix}
\grstep{ L_2 - 2 L_1 }
\begin{amatrix}{2}
1 & 2 & 6 \\
0 & -c-4 & -12
\end{amatrix}
\grstep{ \frac{-1}{c+4} L_2 }
\begin{amatrix}{2}
1 & 2 & 6 \\
0 & 1 & \frac{12}{c+4}
\end{amatrix}
\grstep{ L_1 - 2 L_2 }
\begin{amatrix}{2}
1 & 0 & \frac{6c}{c+4} \\
0 & 1 & \frac{12}{c+4}
\end{amatrix}
\]

Se $c = -4$ a segunda operação deixa de ser possível, e o sistema não tem solução. Por outro lado, se $c \neq -4$, todos os passos podem ser realizados e conclui-se que o sistema é possível e determinado, tendo como única solução o ponto $\left(\frac{6c}{c+4}, \frac{12}{c+4}\right)$.
\item A matriz aumentada associada ao segundo sistema pode ser levada à sua forma escalonada reduzida por linhas por meio das seguintes operações elementares:
\begin{align*}
\begin{amatrix}{2}
1 & 2 & 6 \\
-c & 1 & 1-4c
\end{amatrix}
\grstep{ L_2 + c L_1 }
\begin{amatrix}{2}
1 & 2 & 6 \\
0 & 1+2c & 1+2c
\end{amatrix}
&
\grstep{ \frac{1}{1+2c} L_2 }
\begin{amatrix}{2}
1 & 2 & 6 \\
0 & 1 & 1
\end{amatrix}
\grstep{ L_1 - 2 L_2 }
\begin{amatrix}{2}
1 & 0 & 4 \\
0 & 1 & 1
\end{amatrix}
\end{align*}

Desta vez, se $c = -1/2$ a segunda linha zera após a primeira operação elementar, e o sistema tem mais de uma solução. De fato, o escalonamento mostra que o sistema original é equivalente a um sistema formado pela primeira equação e por uma equação do tipo $0=0$, que não impõe qualquer restrição sobre os valores de $x$ e $y$. Assim, todo par da forma $(6-2y, y)$, com $y \in \R$, é solução deste sistema possível e indeterminado.

Por outro lado, nos casos em que $c \neq -1/2$, os três passos da eliminação de Gauss-Jordan podem ser realizados, e a conclusão é de que o sistema possui como única solução o ponto $(4,1)$, sendo então possível e determinado.
\end{enumerate}

\item \begin{enumerate}
\item Geometricamente, nota-se que conforme o valor de $c$ vai se aproximando de $c=-4$ a reta que corresponde à segunda equação gira em torno da origem até ficar paralela à reta da primeira equação. Quando isso ocorre, não há um ponto de interseção. Nos demais casos, as retas se intersectam em um único ponto.
\item Geometricamente, ao variar o valor de $c$, uma das retas gira em torno do ponto $(4,1)$, em que elas se intersectam, e em um caso específico (quando $c=-1/2$) as duas retas coincidem, fazendo com que todos os seus pontos sejam pontos de interseção.
\end{enumerate}
\end{enumerate}

\item Há duas possibilidades, dependendo das entradas da primeira coluna:
\begin{enumerate}
\item Se $a \neq 0$ então a redução à forma escalonada reduzida começa com as seguintes operações elementares:
\begin{align*}
\begin{amatrix}{2}
a & b & 0 \\
c & d & 0
\end{amatrix}
&
\grstep{ \frac{1}{a} L_1 }
\begin{amatrix}{2}
1 & \frac{b}{a} & 0 \\
c & d & 0
\end{amatrix}
\grstep{ L_2 - c L_1 }
\begin{amatrix}{2}
1 & \frac{b}{a} & 0 \\
0 & d-c\frac{b}{a} & 0
\end{amatrix}
=
\begin{amatrix}{2}
1 & \frac{b}{a} & 0 \\
0 & \frac{ad-bc}{a} & 0
\end{amatrix}
\end{align*}
Neste ponto, a hipótese de que $ad-bc \neq 0$ pode ser usada para concluir a eliminação:
\begin{align*}
\begin{amatrix}{2}
1 & \frac{b}{a} & 0 \\
0 & \frac{ad-bc}{a} & 0
\end{amatrix}
&
\grstep{ \frac{a}{ad-bc} L_2 }
\begin{amatrix}{2}
1 & \frac{b}{a} & 0 \\
0 & 1 & 0
\end{amatrix}
\grstep{ L_1 - \frac{b}{a} L_2 }
\begin{amatrix}{2}
1 & 0 & 0 \\
0 & 1 & 0
\end{amatrix}.
\end{align*}
Assim, o sistema $MX = 0$ só tem a solução trivial $X = 0$.

\item Se $a = 0$ então uma troca da primeira linha com a segunda faz com que o problema recaia no caso anterior, em que a primeira entrada da primeira linha não é zero. Note que neste caso $c$ não será zero, pois senão ocorreria $ad-bc = 0 \cdot d-b \cdot 0 = 0$.
\end{enumerate}
\textbf{Observação:} Note que $ad-bc$ é justamente a fórmula do determinante da matriz $\begin{bmatrix}
a & b \\
c & d
\end{bmatrix}$.
\end{enumerate}
\end{document}
