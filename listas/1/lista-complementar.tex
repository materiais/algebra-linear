\documentclass[12pt,a4paper]{article}
\usepackage{cmap} % Makes the PDF copiable. See http://tex.stackexchange.com/a/64198/25761
\usepackage[T1]{fontenc}
\usepackage[brazil]{babel}
\usepackage[utf8]{inputenc}
\usepackage{amsmath}
\usepackage{amsfonts}
\usepackage{amssymb}
\usepackage{amsthm}
\usepackage{textcomp} % \degree
\usepackage{gensymb} % \degree
\usepackage[usenames,svgnames,dvipsnames]{xcolor} % colors
\usepackage{hyperref}
\usepackage{multicol}
\usepackage{graphicx}
\usepackage[top=2cm, bottom=2cm, left=2cm, right=2cm]{geometry}
\usepackage{systeme}

\newcommand{\fixme}{{\color{red}(...)}}
\newcommand*\sen{\operatorname{sen}}
\newcommand*\adj[1]{\operatorname{adj}\left(#1\right)}
\newcommand\ii{\mathrm{i}} 

\newcommand*\R{\mathbb{R}}
\newcommand*\C{\mathbb{C}}

\newcommand{\IconPc}{\includegraphics[width=1em]{computer.png}}
\newcommand{\IconCalc}{\includegraphics[width=1em]{calculator.png}}
\newcommand{\IconThink}{\includegraphics[width=1em]{pencil.png}}
\newcommand{\IconCheck}{\includegraphics[width=1em]{checkmark.png}}
\newcommand{\IconConcept}{\includegraphics[width=1em]{edit.png}}

\newlength{\SmileysLength}
\setlength{\SmileysLength}{\labelwidth}\addtolength{\SmileysLength}{\labelsep}

\newcommand{\calc}{\hspace*{-\SmileysLength}\makebox[0pt][r]{\IconCalc}%
   \hspace*{\SmileysLength}}
\newcommand{\software}{\hspace*{-\SmileysLength}\makebox[0pt][r]{\IconPc}%
   \hspace*{\SmileysLength}}
\newcommand{\teoria}{\hspace*{-\SmileysLength}\makebox[0pt][r]{\IconThink}%
   \hspace*{\SmileysLength}}
\newcommand{\conceito}{\hspace*{-\SmileysLength}\makebox[0pt][r]{\IconCheck}%
   \hspace*{\SmileysLength}}
\newcommand{\concept}{\hspace*{-\SmileysLength}\makebox[0pt][r]{\IconCheck}%
   \hspace*{\SmileysLength}}


\author{Helder Geovane Gomes de Lima}
\title{1ª Lista de exercícios - continuação - ALI0001/ALG2001}
\date{}

\begin{document}
\thispagestyle{empty}

\begin{center}
\includegraphics[width=9.0cm]{marca.jpg}
\noindent\begin{tabular}{l c c r}
  \textbf{1ª Lista de Exercícios - Continuação} & \textbf{(ALI0001/ALG2001)}\\
\end{tabular}
\\ Prof. Helder Geovane Gomes de Lima
\end{center}

\section*{Legenda}
\begin{multicols}{3}
\begin{itemize}
\item[] \hspace*{\SmileysLength} \calc \hspace*{-\SmileysLength} Cálculos
\item[] \hspace*{\SmileysLength} \conceito \hspace*{-\SmileysLength} Conceitos
\item[] \hspace*{\SmileysLength} \teoria \hspace*{-\SmileysLength} Teoria
%\item[] \hspace*{\SmileysLength} \software \hspace*{-\SmileysLength} Software
\end{itemize}
\end{multicols}

\section*{Questões}
\begin{enumerate}
\item \conceito Dê exemplos de matrizes não nulas $A$ e $B$ de tamanho $n \times n$ (com $n \geq 2$) tais que:
\begin{enumerate}
\item $\det(A + B)    = \det(A) + \det(B)$
\item $\det(A + B) \neq \det(A) + \det(B)$
\item $\det(c A) = c \det(A)$, para algum escalar $c \neq 0$
\item $\det(c A) \neq c \det(A)$, para algum escalar $c \neq 0$
\end{enumerate}

\item \calc Verifique que as matrizes $P = \begin{bmatrix}
2 &  0 &  0 & 0\\
0 & -2 &  0 & 0\\
0 &  1 &  1 & 0\\
2 & -5 & -1 & 2
\end{bmatrix}$ e $Q = \begin{bmatrix}
 1 &  0 &  0 & -1\\
 0 &  1 & -1 &  2\\
 0 & -1 &  5 &  0\\
-1 &  2 &  0 &  7
\end{bmatrix}$ satisfazem:
\begin{enumerate}
\item $\det(PQ) = \det(QP) = \det(P) \det(Q)$
\item $\det(R^T) = \det(R)$, sendo $R = P + Q$
\item $\det(P^{-1}) = \left( \det(P) \right)^{-1}$
\end{enumerate}

\item \calc Calcule o determinante das seguintes matrizes:
\begin{enumerate}
\item $Q = 
\begin{bmatrix}
 \frac{1}{12} & \frac{2}{3} & -\frac{1}{4} \\
 \frac{1}{3}  & \frac{1}{6} & -\frac{3}{4} \\
-\frac{1}{12} & \frac{1}{6} & 0
\end{bmatrix}$

\item $R = \begin{bmatrix}
\frac{\sqrt{3}}{2} & \frac{\sqrt{2}}{4} & -\frac{\sqrt{2}}{4} \\
-\frac{1}{2} & \frac{\sqrt{6}}{4} & -\frac{\sqrt{6}}{4} \\
0 & \frac{\sqrt{2}}{2} & \frac{\sqrt{2}}{2}
\end{bmatrix}$

\item $C =
\begin{bmatrix}
\ii & 3 + \ii & -\ii & 1\\
0 & 2 & 3 & 0\\
1 & 0 & 1+\ii & 0\\
1 & 0 & 1+\ii & 2
\end{bmatrix}$, sendo $i = \sqrt{-1} \in \C$

\item $T = D D^{T}$, sendo $D = 
\begin{bmatrix}
-1 & 1 & 1 &  0 \\
 2 & 2 & 0 & -1
\end{bmatrix}$

\item $U = D^{T} D$, sendo $D$ como no item anterior

\item $A = L U$, sendo $L =
\begin{bmatrix}
2 & 0 & 0 \\
1/2 & 3 & 0 \\
-5/4 & 9/2 & 1
\end{bmatrix}$ e $U = \begin{bmatrix}
-1 & 1/2 & 0 \\
 0 & 2 & -3/2 \\
 0 & 0 & 1
\end{bmatrix}$

\item $M = P Q P^{-1}$, sendo $P =
\begin{bmatrix}
   1 &  0 &  5 &  0\\
   0 &  1 &  0 &  3\\
   0 &  0 & -2 &  2\\
   0 &  0 &  0 &  1
\end{bmatrix}$ e $Q =
\begin{bmatrix}
 1 & -1 & 20 & -7 \\
 3 &  1 & 21 & -1 \\
 0 &  0 & -3 &  1 \\
-1 &  0 & -7 &  1
\end{bmatrix}$
\end{enumerate}

\item \calc Mostre que
\begin{enumerate}
\begin{multicols}{2}
\item 
$
\begin{vmatrix}
a + c & b + d \\
x + u & y + v
\end{vmatrix}
=\\
\begin{vmatrix}
a & b \\
x & y
\end{vmatrix}
+
\begin{vmatrix}
c & d \\
x & y
\end{vmatrix}
+
\begin{vmatrix}
a & b \\
u & v
\end{vmatrix}
+
\begin{vmatrix}
c & d \\
u & v
\end{vmatrix}$

\item 
$\begin{vmatrix}
x & y & z \\
u & v & 0 \\
w & 0 & 0
\end{vmatrix}
= -wvz$

\item 
$\begin{vmatrix}
a & b & c & d \\
e & f & g & 0\\
h & i & 0 & 0\\
j & 0 & 0 & 0
\end{vmatrix}
= jigd$

\item 
$\begin{vmatrix}
a & b & c & d & e \\
f & g & h & i & 0 \\
x & y & z & 0 & 0 \\
u & v & 0 & 0 & 0 \\
w & 0 & 0 & 0 & 0
\end{vmatrix}
= wvzie$
\end{multicols}
\end{enumerate}

\item \calc Verifique as seguintes igualdades:
\begin{enumerate}
\item $\begin{vmatrix}
A & 0^T \\
0 & B
\end{vmatrix}
= \det(A) \det(B)$, para $A = \begin{bmatrix}
  7 &  2 \\
-14 & -3
\end{bmatrix}$, $B = \begin{bmatrix}
-5 & 1 & 2 \\
-10 & 3 & 6 \\
-15 & 7 & 15
\end{bmatrix}$ e $0 = \begin{bmatrix}
0 & 0 \\
0 & 0 \\
0 & 0
\end{bmatrix}$
%
%$
%\begin{vmatrix}
%  7 &  2 &   0 & 0 & 0 \\
%-14 & -3 &   0 & 0 & 0 \\
%  0 &  0 &  -5 & 1 & 2 \\
%  0 &  0 & -10 & 3 & 6 \\
%  0 &  0 & -15 & 7 & 15
%\end{vmatrix}
%=
%\begin{vmatrix}
%  7 &  2 \\
%-14 & -3
%\end{vmatrix}
%\cdot 
%\begin{vmatrix}
%-5 & 1 & 2 \\
%-10 & 3 & 6 \\
%-15 & 7 & 15
%\end{vmatrix}$

\item 
$
\begin{vmatrix}
C & D \\
D & C
\end{vmatrix}
=\det(C-D) \cdot \det(C+D)
$, sendo $C=
\begin{bmatrix}
-3 & 1 \\
-3 & 0
\end{bmatrix}$ e $D=
\begin{bmatrix}
1 & -2 \\
2 & -2
\end{bmatrix}$
%
%$
%\begin{vmatrix}
%-3 &  1 &  1 & -2 \\
%-3 &  0 &  2 & -2 \\
% 1 & -2 & -3 &  1 \\
% 2 & -2 & -3 &  0
%\end{vmatrix}
%=
%\det\left(
%\begin{bmatrix}
%-3 &  1 \\
%-3 &  0
%\end{bmatrix}
%-
%\begin{bmatrix}
%1 & -2 \\
%2 & -2
%\end{bmatrix}
%\right)
%\cdot 
%\det\left(
%\begin{bmatrix}
%-3 &  1 \\
%-3 &  0
%\end{bmatrix}
%+
%\begin{bmatrix}
%1 & -2 \\
%2 & -2
%\end{bmatrix}
%\right)
%$
\end{enumerate}

\item \calc Calcule a inversa das seguintes matrizes, utilizando a matriz adjunta correspondente:
\begin{enumerate}
\item $T = \begin{bmatrix}
1 & 0 & 3 \\
0 & 1 & 2 \\
0 & 0 & -1
\end{bmatrix}$
\item $R = \begin{bmatrix}
0 & 1 & 0 \\
\cos(\theta) & 0 & -\sen(\theta) \\
\sen(\theta) & 0 & \cos(\theta)
\end{bmatrix}$, em que $\theta$ é um número real
\end{enumerate}

\item \teoria Mostre que se $A$ é ortogonal (isto é, se $A A^T = I$), então $\det(A) = \pm1$.

\item \teoria Seja $A$ uma matriz antissimétrica $n \times n$, com $n$ ímpar. Prove que $A$ não é inversível.

\item \calc Determine para que valores de $t$ o sistema linear $(A - tI)X = 0$ possui mais de uma solução, sendo $I$ a matriz identidade, $A$ a matriz definida nos casos a seguir, $(A - tI)$ a matriz de coeficientes do sistema, e $0$ uma matriz coluna de ordem apropriada.
\begin{multicols}{3}
\begin{enumerate}
\item $A = \begin{bmatrix}
0 & 3 \\
1 & 2
\end{bmatrix}$
\item $A = \begin{bmatrix}
0 & 0 &  0 \\
1 & 0 & 20 \\
0 & 1 & -1
\end{bmatrix}$
\item $A = \begin{bmatrix}
0 & 0 & 0 &  0 \\
1 & 0 & 0 &  0 \\
0 & 1 & 0 & 3 \\
0 & 0 & 1 & -2
\end{bmatrix}$
\end{enumerate}
\end{multicols}

\item \calc Utilize a regra de Cramer para resolver os seguintes sistemas lineares:
\begin{multicols}{3}
\begin{enumerate}
\item \systeme{
-2x_1+x_2=1,
14x_1-10x_2=2
}
\item \systeme{
 2x_1+2x_2+ x_3=0,
-4x_1-5x_2-2x_3=0,
 2x_1+2x_2+3x_3=0
} 
\item \systeme{
 2x_1    +3x_3 =-3,
-4x_1+x_2-5x_3 =8,
     -x_2 +x_3 =4
} 
\end{enumerate}
\end{multicols}
\end{enumerate}


\newpage
\section*{Respostas}
\begin{enumerate}
\item
\begin{enumerate}
\item Considere $A = I \in M_{3 \times 3}(\R)$ e $B = -I \in M_{3 \times 3}(\R)$. Então $\det(A+B) = \det(0) = 0 = 1 + (-1) = \det(A) + \det(B)$.
\item Considere $A = B = I \in M_{2 \times 2}(\R)$. Então $\det(A) = \det(B) = 1$ e $\det(A+B) = \det(2I) = 4$, enquanto que $\det(A) + \det(B) = 1 + 1 = 2$.
\item Sabe-se que para $A \in M_{n \times n}(\R)$, vale $\det(cA) = c^n\det(A)$. Deste modo, $\det(cA) = c \det(A)$ se, e somente se, $c^n\det(A) = c\det(A)$. Ou seja, pode-se escolher qualquer matriz que não seja inversível, e o resultado será \[c^n\det(A) = c^n 0 = 0 = c 0 = c\det(A).\] Outra opção é escolher $c = 1$ e qualquer matriz inversível $A$.
\item Seguindo o raciocínio do item anterior, basta escolher uma matriz $A$ inversível e qualquer $c \in \R$ tal que $c^n \neq c$, ou seja, $c \neq 1$ e $c \neq 0$.
\end{enumerate}
\item 
\begin{enumerate}
\item $\det(PQ) =\det(QP) = -32 = (-8) \cdot 4 = \det(P) \cdot \det(Q)$
\item $R = P+Q = \begin{bmatrix}
3 &  0 &  0 & -1\\
0 & -1 & -1 &  2\\
0 &  0 &  6 &  0\\
1 & -3 & -1 &  9
\end{bmatrix}$ e $\det(R)^T = -60 = \det(R)$.
\item $\displaystyle \det(P^{-1}) = -1/8 = \frac{1}{-8} = \frac{1}{\det(P)} = \det(P)^{-1}$
\end{enumerate}
\item
\begin{enumerate}
\item $\det(Q) = \frac{5}{144}$
\item $\det(R) = 1$
\item $\det(C) = 14 + 14 \ii$
\item $T = DD^T = \begin{bmatrix}
3 & 0\\
0 & 9
\end{bmatrix}$ e $\det(T) = 27$
\item $U = D^TD = \begin{bmatrix}
 5 &  3 & -1 & -2\\
 3 &  5 &  1 & -2\\
-1 &  1 &  1 &  0\\
-2 & -2 &  0 &  1
\end{bmatrix}$ e $\det(T) = 0$
\item $A = LU = \begin{bmatrix}
          -2 &            1 &             0 \\
-\frac{1}{2} & \frac{25}{4} &  -\frac{9}{2} \\
 \frac{5}{4} & \frac{67}{8} & -\frac{23}{4}
\end{bmatrix}$ e $\det(T) = \det(L) \det(U) = (2 \cdot 3 \cdot 1) \cdot (-1 \cdot 2 \cdot 1)= -12$
\item $M = \begin{bmatrix}
 1 & -1 &  0 &  1\\
 0 &  1 &  0 & -1\\
-2 &  0 & -1 &  2\\
-1 &  0 &  1 & -1
\end{bmatrix}$ e $\det(M) = \det(P)\det(Q)\det(P^{-1}) = \frac{\det(P)\det(Q)}{\det(P)} = \det(Q) = -1$
\end{enumerate}

\item
\begin{enumerate}
\item \begin{align*}
\begin{vmatrix}
a + c & b + d \\
x + u & y + v
\end{vmatrix}
& =
\begin{vmatrix}
a + c & b + d \\
x & y
\end{vmatrix}
+
\begin{vmatrix}
a + c & b + d \\
u & v
\end{vmatrix}\\
& =
\begin{vmatrix}
a & b \\
x & y
\end{vmatrix}
+
\begin{vmatrix}
c & d \\
x & y
\end{vmatrix}
+
\begin{vmatrix}
a & b \\
u & v
\end{vmatrix}
+
\begin{vmatrix}
c & d \\
u & v
\end{vmatrix}
\end{align*}

\item 
$\begin{vmatrix}
\textbf{x} & \textbf{y} & \textbf{z} \\
u & v & 0 \\
\textbf{w} & \textbf{0} & \textbf{0}
\end{vmatrix}
= -
\begin{vmatrix}
\textbf{w} & 0 & 0 \\
u & \textbf{v} & 0 \\
x & y & \textbf{z}
\end{vmatrix}
= -wvz$, pois o determinante de matrizes triangulares inferiores é o produto das entradas que aparecem na diagonal.

\item 
$\begin{vmatrix}
\textbf{a} & \textbf{b} & \textbf{c} & \textbf{d} \\
e & f & g & 0\\
h & i & 0 & 0\\
\textbf{j} & \textbf{0} & \textbf{0} & \textbf{0}
\end{vmatrix}
= -
\begin{vmatrix}
j & 0 & 0 & 0\\
\textbf{e} & \textbf{f} & \textbf{g} & \textbf{0}\\
\textbf{h} & \textbf{i} & \textbf{0} & \textbf{0}\\
a & b & c & d
\end{vmatrix}
=
\begin{vmatrix}
\textbf{j} & 0 & 0 & 0\\
h & \textbf{i} & 0 & 0\\
e & f & \textbf{g} & 0\\
a & b & c & \textbf{d}
\end{vmatrix}
= jigd$

\item 
$\begin{vmatrix}
\textbf{a} & \textbf{b} & \textbf{c} & \textbf{d} & \textbf{e} \\
f & g & h & i & 0 \\
x & y & z & 0 & 0 \\
u & v & 0 & 0 & 0 \\
\textbf{w} & \textbf{0} & \textbf{0} & \textbf{0} & \textbf{0}
\end{vmatrix}
= -
\begin{vmatrix}
w & 0 & 0 & 0 & 0 \\
\textbf{f} & \textbf{g} & \textbf{h} & \textbf{i} & \textbf{0} \\
x & y & z & 0 & 0 \\
\textbf{u} & \textbf{v} & \textbf{0} & \textbf{0} & \textbf{0} \\
a & b & c & d & e
\end{vmatrix}
=
\begin{vmatrix}
\textbf{w} & 0 & 0 & 0 & 0 \\
u & \textbf{v} & 0 & 0 & 0 \\
x & y & \textbf{z} & 0 & 0 \\
f & g & h & \textbf{i} & 0 \\
a & b & c & d & \textbf{e}
\end{vmatrix}
= wvzie$
\end{enumerate}
\item
\begin{enumerate}
\item
%$\det(A)
%= \begin{vmatrix}
%  7 &  2 \\
%-14 & -3
%\end{vmatrix}
%= -21 + 28 = 7$
%
%$\det(B)
%= \begin{vmatrix}
%\textbf{-5} & \textbf{1} & \textbf{2} \\
%-10 & 3 & 6 \\
%-15 & 7 & 15
%\end{vmatrix}
%= -5(45-42) -1(-150+90)+2(-70+45)
%= -5$

\begin{align*}
\begin{vmatrix}
A & 0^T \\
0 & B
\end{vmatrix}
& =
\begin{vmatrix}
  7 &  2 &   0 & 0 & 0 \\
-14 & -3 &   0 & 0 & 0 \\
  0 &  0 &  -5 & 1 & 2 \\
  0 &  0 & -10 & 3 & 6 \\
  0 &  0 & -15 & 7 & 15
\end{vmatrix}
=
7
\begin{vmatrix}
-3 &   0 & 0 & 0 \\
 0 &  -5 & 1 & 2 \\
 0 & -10 & 3 & 6 \\
 0 & -15 & 7 & 15
\end{vmatrix}
-2
\begin{vmatrix}
-14 &   0 & 0 & 0 \\
  0 &  -5 & 1 & 2 \\
  0 & -10 & 3 & 6 \\
  0 & -15 & 7 & 15
\end{vmatrix} \\
& =
7 (-3)
\begin{vmatrix}
 -5 & 1 & 2 \\
-10 & 3 & 6 \\
-15 & 7 & 15
\end{vmatrix}
-2(-14)
\begin{vmatrix}
 -5 & 1 & 2 \\
-10 & 3 & 6 \\
-15 & 7 & 15
\end{vmatrix} \\
& =
( 7 (-3) -2(-14) )
\begin{vmatrix}
 -5 & 1 & 2 \\
-10 & 3 & 6 \\
-15 & 7 & 15
\end{vmatrix} \\
& =
\begin{vmatrix}
  7 &  2 \\
-14 & -3
\end{vmatrix}
\cdot 
\begin{vmatrix}
-5 & 1 & 2 \\
-10 & 3 & 6 \\
-15 & 7 & 15
\end{vmatrix}
= \det(A) \det(B)
\end{align*}

\item 
$\det(C-D)=
\begin{vmatrix}
-4 & 3 \\
-5 & 2
\end{vmatrix}
= 7$, $\det(C+D)=
\begin{vmatrix}
-2 & -1 \\
-1 & -2
\end{vmatrix}
= 3$ e
\begin{align*}
\begin{vmatrix}
C & D \\
D & C
\end{vmatrix}
& =
\begin{vmatrix}
-3 &  1 &  1 & -2 \\
-3 &  0 &  2 & -2 \\
 1 & -2 & -3 &  1 \\
 2 & -2 & -3 &  0
\end{vmatrix}
\overset{ L_3 + 2L_1 }{=}
\begin{vmatrix}
-3 &  1 &  1 & -2 \\
-3 &  0 &  2 & -2 \\
-5 &  0 & -1 & -3 \\
 2 & -2 & -3 &  0
\end{vmatrix}
\overset{ L_3 + 2L_1 }{=}
\begin{vmatrix}
-3 & \textbf{1} &  1 & -2 \\
-3 & \textbf{0} &  2 & -2 \\
-5 & \textbf{0} & -1 & -3 \\
-4 & \textbf{0} & -1 & -4
\end{vmatrix}\\
& =-
\begin{vmatrix}
-3 &  2 & -2 \\
-5 & -1 & -3 \\
-4 & -1 & -4
\end{vmatrix}
= -(-12 + 24 -10 + 8 +9 -40)
= 21 = 7 \cdot 3.
\end{align*}
\end{enumerate}

\item
\begin{enumerate}
\item $\displaystyle T^{-1}
= \frac{1}{\det(T)} \adj{T}
= \frac{1}{-1} \begin{bmatrix}
-1 &  0 & 0 \\
 0 & -1 & 0 \\
-3 & -2 & 1
\end{bmatrix}^T
= \begin{bmatrix}
1 & 0 & 3 \\
0 & 1 & 2 \\
0 & 0 & -1
\end{bmatrix}
= T$.
\item $R^{-1}
\displaystyle T^{-1}
%= \frac{1}{\det(R)} \adj{R}
= \frac{1}{-1} \begin{bmatrix}
0 & -1 & 0 \\
-\cos(\theta) & 0 & \sen(\theta) \\
-\sen(\theta) & 0 & -\cos(\theta)
\end{bmatrix}^T
= \begin{bmatrix}
0 & \cos(\theta) & \sen(\theta) \\
1 & 0 & 0 \\
0 & -\sen(\theta) & \cos(\theta)
\end{bmatrix}
=R^T$
\end{enumerate}
\item Se $A$ é ortogonal então $A A^T = I$. Neste caso,
\[
1
= \det(I)
= \det(A A^T)
= \det(A) \det( A^T)
= \det(A) \det( A )
= ( \det(A) )^2
\]
Mas os únicos números reais cujo quadrado é $1$ são $1$ e $-1$, logo $\det(A) = \pm1$.

\item Se $A$ é antissimétrica então $A^T = -A$. Consequentemente,
\[
\det(A)
= \det(A^T)
= \det( -A )
= (-1)^n \det(A).
\]
Assim, caso $n$ seja ímpar, $(-1)^n = -1$ e $\det(A) = -\det(A)$, ou seja, $2\det(A) = 0$ e portanto $\det(A) = 0$. Isto significa que $A$ não é inversível.

\item O sistema $(A - tI)X = 0$ possui mais de uma solução se, e somente se, a matriz $(A - tI)$ não for inversível, isto é, se $\det(A - tI) = 0$. Em cada um dos casos, esta condição resultará em uma equação polinomial na variável $t$, cujas soluções são dadas a seguir:
\begin{enumerate}
\item $t=3$ ou $t=-1$
\item $t=-5$ ou $t=0$ ou $t=4$
\item $t=-3$ ou $t=0$ ou $t=1$
\end{enumerate}
\item
\begin{enumerate}
\item $x_1 = \frac{
\begin{vmatrix}
1 &   1 \\
2 & -10
\end{vmatrix}
}{
\begin{vmatrix}
-2 &   1 \\
14 & -10
\end{vmatrix}
}
=\frac{-12}{6}
= -2$
e
$x_2 = \frac{
\begin{vmatrix}
-2 & 1 \\
14 & 2
\end{vmatrix}
}{
\begin{vmatrix}
-2 &   1 \\
14 & -10
\end{vmatrix}
}
=\frac{-18}{6}
=-3$
\item $x_1 = \frac{
\begin{vmatrix}
0 &  2 &  1 \\
0 & -5 & -2 \\
0 &  2 &  3
\end{vmatrix}
}{
\begin{vmatrix}
 2 &  2 &  1 \\
-4 & -5 & -2 \\
 2 &  2 &  3
\end{vmatrix}
}
=\frac{0}{-4}
= 0$,
$x_2 = \frac{
\begin{vmatrix}
 2 & 0 &  1 \\
-4 & 0 & -2 \\
 2 & 0 &  3
\end{vmatrix}
}{
\begin{vmatrix}
 2 &  2 &  1 \\
-4 & -5 & -2 \\
 2 &  2 &  3
\end{vmatrix}
}
%=\frac{0}{-4}
= 0$
e
$x_3 = \frac{
\begin{vmatrix}
 2 &  2 & 0 \\
-4 & -5 & 0 \\
 2 &  2 & 0
\end{vmatrix}
}{
\begin{vmatrix}
 2 &  2 &  1 \\
-4 & -5 & -2 \\
 2 &  2 &  3
\end{vmatrix}
}
%=\frac{0}{-4}
= 0$
\item
$x_1 = \frac{
\begin{vmatrix}
-3 &  0 &  3 \\
 8 &  1 & -5 \\
 4 & -1 &  1
\end{vmatrix}
}{
\begin{vmatrix}
 2 &  0 &  3 \\
-4 &  1 & -5 \\
 0 & -1 &  1
\end{vmatrix}
}
=\frac{-24}{4}
= -6$,
$x_2 = \frac{
\begin{vmatrix}
 2 & -3 &  3 \\
-4 &  8 & -5 \\
 0 &  4 &  1
\end{vmatrix}
}{
\begin{vmatrix}
 2 &  0 &  3 \\
-4 &  1 & -5 \\
 0 & -1 &  1
\end{vmatrix}
}
=\frac{-4}{4}
= -1$ e \\
$x_3 = \frac{
\begin{vmatrix}
 2 &  0 & -3 \\
-4 &  1 &  8 \\
 0 & -1 &  4
\end{vmatrix}
}{
\begin{vmatrix}
 2 &  0 &  3 \\
-4 &  1 & -5 \\
 0 & -1 &  1
\end{vmatrix}
}
=\frac{12}{4}
= 3$
\end{enumerate}
\end{enumerate}
\end{document}
