\documentclass[12pt,a4paper]{article}
\usepackage{cmap} % Makes the PDF copiable. See http://tex.stackexchange.com/a/64198/25761
\usepackage[T1]{fontenc}
\usepackage[brazil]{babel}
\usepackage[utf8]{inputenc}
\usepackage{amsmath}
\usepackage{amsfonts}
\usepackage{amssymb}
\usepackage{amsthm}
\usepackage{textcomp} % \degree
\usepackage{gensymb} % \degree
\usepackage[usenames,svgnames,dvipsnames]{xcolor}
\usepackage{hyperref}
\usepackage{multicol}
\usepackage{graphicx}
\usepackage[margin=2cm]{geometry}
\usepackage{systeme}

\hypersetup{
    colorlinks = true,
    allcolors = {blue}
}

\newcommand{\fixme}{{\color{red}(...)}}
\newcommand*\sen{\operatorname{sen}}
\newcommand*\tr{\operatorname{tr}}

\newcommand*\R{\mathbb{R}}

\newcommand{\IconPc}{\includegraphics[width=1em]{computer.png}}
\newcommand{\IconCalc}{\includegraphics[width=1em]{calculator.png}}
\newcommand{\IconThink}{\includegraphics[width=1em]{pencil.png}}
\newcommand{\IconCheck}{\includegraphics[width=1em]{checkmark.png}}
\newcommand{\IconConcept}{\includegraphics[width=1em]{edit.png}}

\newlength{\SmileysLength}
\setlength{\SmileysLength}{\labelwidth}\addtolength{\SmileysLength}{\labelsep}

\newcommand{\calc}{\hspace*{-\SmileysLength}\makebox[0pt][r]{\IconCalc}%
   \hspace*{\SmileysLength}}
\newcommand{\software}{\hspace*{-\SmileysLength}\makebox[0pt][r]{\IconPc}%
   \hspace*{\SmileysLength}}
\newcommand{\teoria}{\hspace*{-\SmileysLength}\makebox[0pt][r]{\IconThink}%
   \hspace*{\SmileysLength}}
\newcommand{\conceito}{\hspace*{-\SmileysLength}\makebox[0pt][r]{\IconCheck}%
   \hspace*{\SmileysLength}}
\newcommand{\concept}{\hspace*{-\SmileysLength}\makebox[0pt][r]{\IconCheck}%
   \hspace*{\SmileysLength}}

% Loop Space / CC BY-SA-3.0 / https://tex.stackexchange.com/a/2238/25761
\newenvironment{amatrix}[1]{%
  \left[\begin{array}{@{}*{#1}{c}|c@{}}
}{%
  \end{array}\right]
}

% Loop Space / CC BY-SA-3.0 / https://tex.stackexchange.com/a/3164/25761
%--------grstep
% For denoting a Gauss' reduction step.
% Use as: \grstep{\rho_1+\rho_3} or \grstep[2\rho_5 \\ 3\rho_6]{\rho_1+\rho_3}
\newcommand{\grstep}[2][\relax]{%
   \ensuremath{\mathrel{
       {\mathop{\longrightarrow}\limits^{#2\mathstrut}_{
                                     \begin{subarray}{l} #1 \end{subarray}}}}}}
\newcommand{\swap}{\leftrightarrow}

\newcommand*\disciplina{ALI0001}
\newcommand*\tipo{Lista de Exercícios - Determinantes}
\newcommand*\eu{Helder G. G. de Lima}
\newcommand*\data{\today}

\author{\eu}
\title{\tipo}
\date{\data}

\begin{document}

\begin{center}
\includegraphics[width=9.0cm]{marca} \\
\textbf{\tipo} \\
Prof. \eu\footnote{
Este é um material de acesso livre distribuído sob os termos da licença \href{https://creativecommons.org/licenses/by-sa/4.0/deed.pt_BR}{Creative Commons BY-SA 4.0}}
\end{center}

\section*{Legenda}
\begin{multicols}{4}
\begin{itemize}
\item[] \hspace*{\SmileysLength} \calc \hspace*{-\SmileysLength} Cálculos
\item[] \hspace*{\SmileysLength} \conceito \hspace*{-\SmileysLength} Conceitos
\item[] \hspace*{\SmileysLength} \teoria \hspace*{-\SmileysLength} Teoria
\item[] \hspace*{\SmileysLength} \software \hspace*{-\SmileysLength} Software
\end{itemize}
\end{multicols}

\section*{Questões}

\begin{enumerate}
\item \calc Encontre uma matriz triangular superior equivalente por linhas a cada matriz $P$ indicada a seguir, e utilize-as para calcular o determinante de $P$.
\begin{multicols}{2}
\begin{enumerate}
\item $P =
\begin{bmatrix}
 0 &  5 &  21\\
-3 & -7 & -13\\
 1 &  2 &   3
\end{bmatrix}$
\item $P =
\begin{bmatrix}
0 & 6 & 0 & 10\\
0 & 3 & -1 & 5\\
0 & 9 & -3 & 14\\
5 & 0 & 0 & 1
\end{bmatrix}$
\item $P =
\begin{bmatrix}
 1 & 0 & 0 & 0 & 2 \\
 0 & 2 & 4 & 2 & 0 \\
 0 & 3 & 0 & 3 & 0 \\
 0 & 1 & 2 & -2 & 0 \\
 5 & 0 & 0 & 0 & 1
\end{bmatrix}$
\end{enumerate}
\end{multicols}

\item \calc Supondo que a matriz $M =
\begin{bmatrix}
a & b \\
c & d
\end{bmatrix}$
satisfaz $\det{M} = 9$, calcule $\begin{vmatrix}
a+c & a+b+c+d \\
2a & 2(a+b)
\end{vmatrix}$.

\item \conceito Dê exemplos de matrizes não nulas $A$ e $B$ de tamanho $n \times n$ (com $n \geq 2$) tais que:
\begin{multicols}{2}
\begin{enumerate}
\item $\det(A + B)    = \det(A) + \det(B)$
\item $\det(A + B) \neq \det(A) + \det(B)$
\item $\det(c A) = c \det(A)$, para algum $c \neq 0$
\item $\det(c A) \neq c \det(A)$, para algum $c \neq 0$
\end{enumerate}
\end{multicols}

\item \calc Verifique que as matrizes $P = \begin{bmatrix}
2 &  0 &  0 & 0\\
0 & -2 &  0 & 0\\
0 &  1 &  1 & 0\\
2 & -5 & -1 & 2
\end{bmatrix}$ e $Q = \begin{bmatrix}
 1 &  0 &  0 & -1\\
 0 &  1 & -1 &  2\\
 0 & -1 &  5 &  0\\
-1 &  2 &  0 &  7
\end{bmatrix}$ satisfazem:
\begin{enumerate}
\item $\det(PQ) = \det(P) \det(Q)$
\item $\det(QP) = \det(P) \det(Q)$
\item $\det(R^T) = \det(R)$, sendo $R = P + Q$
\end{enumerate}

\item \calc Calcule o determinante das seguintes matrizes:
\begin{enumerate}
\item $Q =
\begin{bmatrix}
 \frac{1}{12} & \frac{2}{3} & -\frac{1}{4} \\
 \frac{1}{3}  & \frac{1}{6} & -\frac{3}{4} \\
-\frac{1}{12} & \frac{1}{6} & 0
\end{bmatrix}$

\item $R = \begin{bmatrix}
\frac{\sqrt{3}}{2} & \frac{\sqrt{2}}{4} & -\frac{\sqrt{2}}{4} \\
-\frac{1}{2} & \frac{\sqrt{6}}{4} & -\frac{\sqrt{6}}{4} \\
0 & \frac{\sqrt{2}}{2} & \frac{\sqrt{2}}{2}
\end{bmatrix}$

\item $T = D D^{T}$, sendo $D =
\begin{bmatrix}
-1 & 1 & 1 &  0 \\
 2 & 2 & 0 & -1
\end{bmatrix}$

\item $U = D^{T} D$, sendo $D$ como no item anterior

\item $A = L U$, sendo $L =
\begin{bmatrix}
2 & 0 & 0 \\
1/2 & 3 & 0 \\
-5/4 & 9/2 & 1
\end{bmatrix}$ e $U = \begin{bmatrix}
-1 & 1/2 & 0 \\
 0 & 2 & -3/2 \\
 0 & 0 & 1
\end{bmatrix}$
\end{enumerate}

\item \calc Mostre que
\begin{enumerate}
\begin{multicols}{3}
\item
$\begin{vmatrix}
x & y & z \\
u & v & 0 \\
w & 0 & 0
\end{vmatrix}
= -wvz$

\item
$\begin{vmatrix}
a & b & c & d \\
e & f & g & 0\\
h & i & 0 & 0\\
j & 0 & 0 & 0
\end{vmatrix}
= jigd$

\item
$\begin{vmatrix}
a & b & c & d & e \\
f & g & h & i & 0 \\
x & y & z & 0 & 0 \\
u & v & 0 & 0 & 0 \\
w & 0 & 0 & 0 & 0
\end{vmatrix}
= wvzie$
\end{multicols}
\end{enumerate}

\item \calc Determine para que valores de $t$ o sistema linear $(A - tI)X = 0$ possui mais de uma solução, sendo $I$ a matriz identidade, $A$ a matriz definida nos casos a seguir, $(A - tI)$ a matriz de coeficientes do sistema, e $0$ uma matriz coluna de ordem apropriada.
\begin{multicols}{3}
\begin{enumerate}
\item $A = \begin{bmatrix}
0 & 3 \\
1 & 2
\end{bmatrix}$
\item $A = \begin{bmatrix}
0 & 0 &  0 \\
1 & 0 & 20 \\
0 & 1 & -1
\end{bmatrix}$
\item $A = \begin{bmatrix}
0 & 0 & 0 &  0 \\
1 & 0 & 0 &  0 \\
0 & 1 & 0 & 3 \\
0 & 0 & 1 & -2
\end{bmatrix}$
\end{enumerate}
\end{multicols}
\end{enumerate}

\newpage
\section*{Respostas}
\begin{enumerate}
\item \begin{enumerate}
\item
$\det{P} =
-
\begin{vmatrix}
0 & 5 & 21\\
3 & 7 & 13\\
1 & 2 &  3
\end{vmatrix}
=
\begin{vmatrix}
1 & 2 &  3\\
3 & 7 & 13\\
0 & 5 & 21
\end{vmatrix}
=
\begin{vmatrix}
1 & 2 &  3\\
0 & 1 &  4\\
0 & 5 & 21
\end{vmatrix}
=
\begin{vmatrix}
1 & 2 & 3\\
0 & 1 & 4\\
0 & 0 & 1
\end{vmatrix}
= 1$

\item
$\det{P}
%=
%\begin{vmatrix}
%0 & 6 & 0 & 10\\
%0 & 3 & -1 & 5\\
%0 & 9 & -3 & 14\\
%5 & 0 & 0 & 1
%\end{vmatrix}
=
-
\begin{vmatrix}
5 & 0 & 0 & 1\\
0 & 3 & -1 & 5\\
0 & 9 & -3 & 14\\
0 & 6 & 0 & 10
\end{vmatrix}
=
-
\begin{vmatrix}
5 & 0 & 0 & 1\\
0 & 3 & -1 & 5\\
0 & 0 & 0 & -1\\
0 & 0 & 2 & 0
\end{vmatrix}
=
\begin{vmatrix}
5 & 0 & 0 & 1\\
0 & 3 & -1 & 5\\
0 & 0 & 2 & 0\\
0 & 0 & 0 & -1
\end{vmatrix}
= -30$

\item
\begin{align*}
\det{P}
& =
2 \cdot 3 \cdot
\begin{vmatrix}
 1 & 0 & 0 & 0 & 2 \\
 0 & 1 & 2 & 1 & 0 \\
 0 & 1 & 0 & 1 & 0 \\
 0 & 1 & 2 & -2 & 0 \\
 5 & 0 & 0 & 0 & 1
\end{vmatrix}
=
6
\begin{vmatrix}
 1 & 0 & 0 & 0 & 2 \\
 0 & 1 & 2 & 1 & 0 \\
 0 & 1 & 0 & 1 & 0 \\
 0 & 1 & 2 & -2 & 0 \\
 0 & 0 & 0 & 0 & -9
\end{vmatrix}
=
6
\begin{vmatrix}
 1 & 0 & 0 & 0 & 2 \\
 0 & 1 & 2 & 1 & 0 \\
 0 & 0 & -2 & 0 & 0 \\
 0 & 0 & 0 & -3 & 0 \\
 0 & 0 & 0 & 0 & -9
\end{vmatrix} \\
& =
6 \cdot 1 \cdot 1 \cdot (-2) \cdot (-3) \cdot (-9)
= -324
\end{align*}
\end{enumerate}

\item Usando as propriedades dos determinantes relacionadas ao uso de operações elementares sobre as linhas e colunas da matriz $S$, resulta que:
\begin{align*}
\det{S}
& = \begin{vmatrix}
a+c & a+b+c+d \\
2a & 2(a+b)
\end{vmatrix}
= 2
\begin{vmatrix}
a+c & a+b+c+d \\
a & a+b
\end{vmatrix}
= 2
\begin{vmatrix}
a+c & b+d \\
a & b
\end{vmatrix} \\
& = 2
\begin{vmatrix}
c & d \\
a & b
\end{vmatrix}
= -2
\begin{vmatrix}
a & b \\
c & d
\end{vmatrix}
= -2 \det{M}
=-18.
\end{align*}

\item \begin{enumerate}
\item Considere $A = I \in M_{3 \times 3}(\R)$ e $B = -I \in M_{3 \times 3}(\R)$. Então $\det(A+B) = \det(0) = 0 = 1 + (-1) = \det(A) + \det(B)$.
\item Considere $A = B = I \in M_{2 \times 2}(\R)$. Então $\det(A) = \det(B) = 1$ e $\det(A+B) = \det(2I) = 4$, enquanto que $\det(A) + \det(B) = 1 + 1 = 2$.
\item Sabe-se que para $A \in M_{n \times n}(\R)$, vale $\det(cA) = c^n\det(A)$. Deste modo, $\det(cA) = c \det(A)$ se, e somente se, $c^n\det(A) = c\det(A)$. Ou seja, pode-se escolher qualquer matriz que tenha determinante nulo, e o resultado será \[c^n\det(A) = c^n 0 = 0 = c 0 = c\det(A).\] Outra opção é escolher $c = 1$ e qualquer matriz $A$.
\item Seguindo o raciocínio do item anterior, basta escolher uma matriz $A$ com determinante não nulo e qualquer $c \in \R$ tal que $c^n \neq c$, ou seja, $c \neq 1$ e $c \neq 0$.
\end{enumerate}

\item \begin{enumerate}
\item $\det(PQ) = -32 = (-8) \cdot 4 = \det(P) \cdot \det(Q)$
\item $\det(QP) = -32 = (-8) \cdot 4 = \det(P) \cdot \det(Q)$
\item $R = P+Q = \begin{bmatrix}
3 &  0 &  0 & -1\\
0 & -1 & -1 &  2\\
0 &  0 &  6 &  0\\
1 & -3 & -1 &  9
\end{bmatrix}$ e $\det(R)^T = -60 = \det(R)$.
\end{enumerate}

\item
\begin{enumerate}
\item $\det(Q) = \frac{5}{144}$
\item $\det(R) = 1$
\item $T = DD^T = \begin{bmatrix}
3 & 0\\
0 & 9
\end{bmatrix}$ e $\det(T) = 27$
\item $U = D^TD = \begin{bmatrix}
 5 &  3 & -1 & -2\\
 3 &  5 &  1 & -2\\
-1 &  1 &  1 &  0\\
-2 & -2 &  0 &  1
\end{bmatrix}$ e $\det(T) = 0$
\item $A = LU = \begin{bmatrix}
          -2 &            1 &             0 \\
-\frac{1}{2} & \frac{25}{4} &  -\frac{9}{2} \\
 \frac{5}{4} & \frac{67}{8} & -\frac{23}{4}
\end{bmatrix}$ e $\det(T) = \det(L) \det(U) = (2 \cdot 3 \cdot 1) \cdot (-1 \cdot 2 \cdot 1)= -12$
\end{enumerate}

\item \begin{enumerate}
\item
$\begin{vmatrix}
\textbf{x} & \textbf{y} & \textbf{z} \\
u & v & 0 \\
\textbf{w} & \textbf{0} & \textbf{0}
\end{vmatrix}
= -
\begin{vmatrix}
\textbf{w} & 0 & 0 \\
u & \textbf{v} & 0 \\
x & y & \textbf{z}
\end{vmatrix}
= -wvz$, pois o determinante de matrizes triangulares inferiores é o produto das entradas que aparecem na diagonal.

\item
$\begin{vmatrix}
\textbf{a} & \textbf{b} & \textbf{c} & \textbf{d} \\
e & f & g & 0\\
h & i & 0 & 0\\
\textbf{j} & \textbf{0} & \textbf{0} & \textbf{0}
\end{vmatrix}
= -
\begin{vmatrix}
j & 0 & 0 & 0\\
\textbf{e} & \textbf{f} & \textbf{g} & \textbf{0}\\
\textbf{h} & \textbf{i} & \textbf{0} & \textbf{0}\\
a & b & c & d
\end{vmatrix}
=
\begin{vmatrix}
\textbf{j} & 0 & 0 & 0\\
h & \textbf{i} & 0 & 0\\
e & f & \textbf{g} & 0\\
a & b & c & \textbf{d}
\end{vmatrix}
= jigd$

\item
$\begin{vmatrix}
\textbf{a} & \textbf{b} & \textbf{c} & \textbf{d} & \textbf{e} \\
f & g & h & i & 0 \\
x & y & z & 0 & 0 \\
u & v & 0 & 0 & 0 \\
\textbf{w} & \textbf{0} & \textbf{0} & \textbf{0} & \textbf{0}
\end{vmatrix}
= -
\begin{vmatrix}
w & 0 & 0 & 0 & 0 \\
\textbf{f} & \textbf{g} & \textbf{h} & \textbf{i} & \textbf{0} \\
x & y & z & 0 & 0 \\
\textbf{u} & \textbf{v} & \textbf{0} & \textbf{0} & \textbf{0} \\
a & b & c & d & e
\end{vmatrix}
=
\begin{vmatrix}
\textbf{w} & 0 & 0 & 0 & 0 \\
u & \textbf{v} & 0 & 0 & 0 \\
x & y & \textbf{z} & 0 & 0 \\
f & g & h & \textbf{i} & 0 \\
a & b & c & d & \textbf{e}
\end{vmatrix}
= wvzie$
\end{enumerate}

\item O sistema $(A - tI)X = 0$ possui mais de uma solução se, e somente se, a matriz $(A - tI)$ tiver determinante nulo, isto é, se $\det(A - tI) = 0$. Em cada um dos casos, esta condição resultará em uma equação polinomial na variável $t$, cujas soluções são dadas a seguir:
\begin{enumerate}
\item $t=3$ ou $t=-1$
\item $t=-5$ ou $t=0$ ou $t=4$
\item $t=-3$ ou $t=0$ ou $t=1$
\end{enumerate}
\end{enumerate}
\end{document}
